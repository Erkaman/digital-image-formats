\documentclass[a4paper]{article}
\usepackage[T1]{fontenc}
\usepackage[swedish]{babel}
\usepackage[charter]{mathdesign}
\usepackage[kerning,spacing]{microtype}
\usepackage{longtable}
\hyphenpenalty=100000
\title{Schema projektarbete}
\author{Eric Arneb�ck}
\date{}
\setlength{\parskip}{1ex plus 0.5ex minus 0.2ex}
\setlength{\parindent}{0pt}

\newcounter{week}
\setcounter{week}{38}
\newcommand{\nextweek}{%
   \theweek%
   \stepcounter{week}%
}

\newcommand{\weekschedule}[1]{\nextweek & #1 \\\\}

\begin{document}

\maketitle

(Om en vecka �r tom betyder det att samma sak g�rs som f�rra veckan.)

\centering
\begin{longtable}{p{1cm}p{10cm}}
  \textbf{v.} & \textbf{Att g�ra} \\

  \weekschedule{G�r f�rdigt skriftliga delen om RLE. Renskriv och
    st�da upp allt nuvarande material. Eftersom jag tidigare inte
    f�ljde ett riktigt schema �r det nuvarande materialet lite r�rigt,
    d�rf�r kr�vs det en vecka f�r att renskriva det. }

  \weekschedule{Sammanfatta TGA skriftligt. }

  \weekschedule{Utforska och f�rst� kompressionsalgoritmen LZW}

  \weekschedule{LZW sammanfattas skriftligt.}

  \weekschedule{Koda GIF-laddare. }

  \weekschedule{G�r f�rdigt GIF laddare. }

  \weekschedule{Sammanfatta GIF formatet skriftligt. }
  \weekschedule{}

  \weekschedule{Renskrivning och av uppst�ddning  av det upp till nu
    skrivna materialet. Om nytt material beh�vs f�r att f�rtydliga vissa saker framskaffas mer material och fler k�llor.}


  \weekschedule{Jobbet p� PNG p�b�rjas med att DEFLATE algoritmen
    utforskas. Algoritmerna LZZ7 och Huffmankodning(som DEFLATE �r en
    kombination av) utforskas}

  \weekschedule{}
  \weekschedule{}

  \weekschedule{DEFLATE algoritmen sammanfattas skriftligt}
  \setcounter{week}{2}
  \weekschedule{DEFLATE algoritmen sammanfattas skriftligt}
  \weekschedule{DEFLATE algoritmen sammanfattas skriftligt}

  \weekschedule{PNG specifikation skummas igenom och boken ``PNG - the
  definitive guide'' l�ses som ett st�d vid sidan av.}
  \weekschedule{}

  \weekschedule{Koda PNG-laddaren}
  \weekschedule{}
  \weekschedule{}
  \weekschedule{}

  \weekschedule{PNG b�rjas att sammanfattas skriftligt}
  \weekschedule{}
  \weekschedule{}
  \weekschedule{}

  \weekschedule{Renskrivnign och uppst�dning av allt material.}

  \weekschedule{}

  \weekschedule{}

  \weekschedule{}

\end{longtable}

\end{document}
