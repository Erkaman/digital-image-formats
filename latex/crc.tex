\begin{comment}
  \bibliography{project.bib}
\end{comment}

\begingroup
% arrays are used to display the CRC polynomial divisions.
% This value is temporarily set in this chapter
% to make a smaller distance between the separate bits in polynomial
% divisions.
\arraycolsep=0.05em

\chapter{Checksums}
\label{cha:crc}

\section{Checksums and error detection}

The following introduction to checksums is based on
\cite{Williams_1993_crc_painless,barr:_addit_check,tanenbaum2003computernetworks_crc,Nelson:1992:FVU:135011.135017_crc32}.

Say I have an image file on my computer. Now imagine that I, using a
hex editor, changed a couple of numbers in the numerical image data of
this file. Now the file is no longer represents the original
image. But if now the computer attempted loading this image, it would
have no way of knowing that the data was fiddled with. If furthermore
the image data was compressed, it is even highly likely that it would
fail in the decompression of the data, and hence fail the loading of
the image.

But is there any for the computer in beforehand to know that I had
fiddled with the file? Suppose that before saving this image the
program producing this image calculated a check number using all of
the numerical data in the image and that this check number was then
added at the end of the image data. Then, any program attempting to
open the image can as a quick way of checking the validity of the
image data recalculate this number on the data and compare it with the
check number at the end of the image data. And since I fiddled with
the image, the same number would no longer be calculated. In this way
the computer could see that the image data was invalid.

But \textit{how} would this number be calculated? Let us say that the
image data simply consisted of the bytes $32,12,241$. What sort of
number can we calculate from these? An easy answer would simply be to
sum them modulo $256$ and add this number as a byte at the end of the
data. It is necessary that we do this modulo 256, because otherwise
the sum would not be able to fit in a 8-bit byte. This sum would be
calculated to $(32 + 12 + 241) \bmod 256 = 29$. This number is then
appended to the end of the data: $32,12,241,29$. The computer could
then validate this check number by recalculating this sum. But this
system has one big flaw: what if I had changed these numbers to
$33,11,241, 29$ when fiddling with the image data?  Because this sum
also calculates to 29: $(33 + 11 + 241) \bmod = 29$.

Thus, the main problem of this method is that since there are only
$256$ possible values for a byte, there is a $\frac{1}{2^8} =
\frac{1}{256}$ chance that an error would go undetected. But on the
other hand, there is a $1 - \frac{1}{256} \approx 99.6\%$ chance that
this method would indeed work.

We could of course make this method stronger by giving it a less
change of failure. Instead of storing the sum in a 8-bit number, we
could for example store it in a 16-bit number. This would mean a much
lower probability of failure to detect errors, namely
$\frac{1}{2^{16}} = \frac{1}{65536}$. But this does not really fix the
original problem; the computed check numbers would still be equal:
$(32 + 12 + 241) \bmod 65535 = 285$ and $(33 + 11 + 241) \bmod 65535 =
285$.

Both of these two computed numbers are known as \textit{checksums}. A
checksum is simply a value calculated from a sequence of numerical
data that is used to verify that the data is error free. Some
checksums have been shown to more secure and less prone to not
detecting data corruptions, and a \crc is one them:

\section{CRC}

One very well used checksum is a \crc, Cyclic Redundancy Check. We will
in the following section discuss the computation of a \crc.

The following description of \crc is based on the references
\cite{Ritter:1986:GCM:12647.12648,Williams_1993_crc_painless,tanenbaum2003computernetworks_crc,Nelson:1992:FVU:135011.135017_crc32,Stigge06reversingcrc,barr:_crc_implem_code_c,black:_fast_crc32_softw,geremia99:_cyclic}. Note
that I will not discuss the mathematics behind \crc in depth nor
rigorously in this text. I will however assume that the reader is
moderately familiar with long division.

\subsection{Polynomial arithmetic}

\subsubsection{Polynomial Representation}

Data in a file is simply a sequence of 8-bit bytes. The number $200$,
for example, is in binary \bin{1100\ 1000}. However, in the world of \crc,
this binary string is instead seen as the polynomial

\begin{equation*}
  1x^7 + 1x^6 + 0x^5 + 0x^4 + 1x^3 + 0x^2 + 0x^1 + 0x^0 = x^7 + x^6 + x^3
\end{equation*}

So that a bit of value $b$ in the position $n$ is represented by
$bx^n$.

\begin{Exercise}[label={bin-to-poly}]

  What polynomials does the following numbers represent:

  \begin{enumerate}[(a)]
  \item The binary number \bin{1011\ 0011}
  \item The number $35$
  \item The hexadecimal number \hex{13}
  \end{enumerate}

\end{Exercise}

\begin{Exercise}[label={poly-to-bin}]

  What binary numbers does the following polynomials represent:

  \begin{enumerate}[(a)]
  \item $x^7 + 1$
  \item $x^3 + x^2 + x + 1$
  \item $x^{15} + x^{12} + x^{11} + x^4 + x^3 + x^2 + x + 1$
  \end{enumerate}

\end{Exercise}

\subsubsection{Polynomial Width}

The width $w$ of such a polynomial is defined as the value of its
highest exponent. The binary number \bin{110}, represented by the
polynomial $x^2 + x^1$, has width $w=2$. The width could also be said to
be the position of the highest toggled bit. Stated more generally,
this means that the polynomial
\begin{equation*}
  P(x) = a_wx^w + a_{w-1}x^{w-1} + \ldots + a_{1}x + a_{0}
\end{equation*}

has a width of $w$, despite the fact that the number it represents is
$w + 1$ bits long.

\begin{Exercise}[label={width-poly}]

  What is the width of the polynomial

  \begin{enumerate}[(a)]
  \item $x^{32} + x^{31} + x^{30} + x^{29} + x^{28} + x^{27} + 1 $
  \item The polynomial represented by the binary number \bin{1000}
  \item The polynomial represented by the binary number \bin{101\ 0101\
    1111\ 0000}
  \end{enumerate}

\end{Exercise}

\subsubsection{Polynomial Addition}

Yet another property holds for these polynomials: all operations are
done modulo 2. To demonstrate this property, we will introduce the
concept of addition. And to make things simpler and cleaner during
these explanations, we will write the polynomials as their binary
equivalents.

First off we describe addition. While the ordinary relationships

\begin{equation*}
  0 + 0 = 0
\end{equation*}

and

\begin{equation*}
  1 + 0 = 0 + 1 = 1
\end{equation*}

hold true, the relationship $1 + 1 = 2$
does not. Instead, the relationship

\begin{equation*}
  1 + 1 = 0
\end{equation*}

is true. Why? Because $(1 + 1) \bmod 2 = 2 \bmod 2 = 0$.

The sum of the numbers $\bin{1100}$ and $\bin{0101}$ can thus be calculated

\begin{center}
  \begin{tabular}{lr}
    & 1100  \\
    $+$ & 0101 \\
    \hline
    & 1001 \\
  \end{tabular}
\end{center}

There is another operation for which the above properties also holds
true for: bitwise xor. To see why this must be true, replace $+$ by
$\BitXor$ in the above calculations. So bitwise xor and polynomial
addition are the exact same things. This discovery is something that
will greatly simplify the implementation of \crc.

\begin{Exercise}[label={poly-add}]
  Using \crc arithmetic, compute the sum  $\bin{1100} + \bin{1001} + \bin{1111}$
\end{Exercise}

\subsubsection{Polynomial Subtraction}

Polynomial subtraction is defined in the exact way as polynomial
addition is, so the following relations hold true:

\begin{equation*}
  1 - 1 = 0 - 0 = 0
\end{equation*}

and

\begin{equation*}
  1 - 0 = 0 - 1 = 1
\end{equation*}

So negative numbers wrap in \crc arithmetic.

So given the two polynomials $P_1(x)$ and $P_2(x)$ the following will
always hold true:

\begin{equation*}
  P_1(x) + P_2(x) = P_1(x) - P_2(x)
\end{equation*}


Since subtraction is equivalent to addition, this has the important
consequence that subtraction is also equivalent to bitwise xor.

\subsubsection{Polynomial Multiplication}

\crc multiplication is, on the other hand, defined just like it is in
ordinary arithmetic. So $\bin{1100} \cdot \bin{1} = \bin{1100}$ and $\bin{1100} \cdot \bin{0} =
\bin{0000}$ hold true. We will not discuss the multiplication of two
operands both longer than one bit, because that is not necessary for
our purposes.

\subsubsection{Polynomial Division}

Having defined addition, subtraction, and multiplication, we can now
define division. Here is the long division of the number $\bin{10011}$ by
$\bin{10}$:

\begin{equation*}
  \newcommand{\divline}{\cline{3-8}}
  \renewcommand\arraystretch{1.2}
  \begin{array}{*2r @{\hskip\arraycolsep}c@{\hskip\arraycolsep} *5r}
    &    &&    &  1 &  0 & 0 & 1\\
    \divline
    1 & 0 &\big)& 1 &  0 & 0 & 1 & 1 \\
    &&& 1 & 0 \\

    \divline

    &&&& 0 & 0 \\
    &&&& 0 & 0 \\

    \divline

    &&&&& 0 & 1 \\
    &&&&& 0 & 0 \\

    \divline

    &&&&&& 1 & 1 \\
    &&&&&& 1 & 0 \\

    \divline

    &&&&&&& 1 \\

  \end{array}
\end{equation*}

So $\frac{\bin{10011}}{\bin{10}}$ is $\bin{1001}$ with a reminder of $\bin{1}$. Stated
differently:

\begin{equation*}
  \bin{10011} = \bin{1001} \cdot \bin{10} + \bin{1}
\end{equation*}

the division operation is easy to understand. In the first step, since
the highest bit of the numerator is toggled, the two highest bits of
the numerator are subtracted by the denominator, $\bin{10} \cdot
\bin{1} = \bin{10}$. In the second step, since the highest bit of the
numerator is now cleared, it is divided by a cleared bit pattern with
the same length as the denominator, $\bin{10} \cdot \bin{0} =
\bin{00}$. These multiples are then noted at the top. This algorithm
keeps going until the bits of the numerator are exhausted. After this
the remaining bits are noted as the remainder at the bottom.

\begin{Exercise}[label={poly-div}]
  Perform the division $\frac{\bin{11100}}{\bin{11}}$ using \crc arithmetic.
\end{Exercise}

\subsection{The CRC computation}

The polynomial division operation is especially interesting for our
purposes because it turns out that the \crc computation is simply a
polynomial division.

To compute the \crc of some data, first you need to consider all the
data a long sequence of bits. Meaning that, for example, the data that
consisted of the numbers 16($\bin{10000}$, 5 binary digits) and 2($\bin{10}$, 2
binary digits) would form the sequence $\bin{1000010}$(We use these rather
arbitrary binary number sizes to make the following example easier to
follow).

This sequence of binary digits is now to be divided by something known
as the \textit{generator polynomial}, $G(x)$. The generator polynomial
is the key parameter of the algorithm and there are many different to
chose from. The only requirement that this polynomial has to satisfy,
is that it has to begin and end with a toggled bit. So $\bin{1001}$ is a valid
generator polynomial, while $\bin{0001}$ is not.

In the following example, we will use the arbitrarily chosen generator
polynomial $\bin{1101}$. The data that we wanted to compute the \crc of was
$\bin{1000010}$.  First, $w=3$ cleared bits are end to the end of the
input bit sequence, so it gets transformed to
$\bin{1000010000}$. Having done all of the preparatory steps, we now
perform the final calculation: the binary sequence is divided by the
generator polynomial, and the reminder of this division is the \crc:

\begin{equation*}
  \newcommand{\divline}{\cline{5-15}}
  \renewcommand\arraystretch{1.2}
  \begin{array}{*4r @{\hskip\arraycolsep}c@{\hskip\arraycolsep} *{10}r}
    &&&&&&&& 1 & 1 & 1 & 0 & 1 & 1 & 1 \\

    \divline
    1 & 1 & 0 & 1 & \big) & 1 & 0 & 0 & 0 & 0 & 1 & 0 & 0 & 0 & 0 \\
    &&&&& 1 & 1 & 0 & 1 && \\
    \divline
    &&&&&& 1 & 0 & 1 & 0 \\
    &&&&&& 1 & 1 & 0 & 1 \\
    \divline
    &&&&&&& 1 & 1 & 1 & 1 \\
    &&&&&&& 1 & 1 & 0 & 1 \\
    \divline
    &&&&&&&& 0 & 1 & 0 & 0 \\
    &&&&&&&& 0 & 0 & 0 & 0 \\
    \divline
    &&&&&&&&& 1 & 0 & 0 & 0 \\
    &&&&&&&&& 1 & 1 & 0 & 1 \\
    \divline
    &&&&&&&&&& 1 & 0 & 1 & 0 \\
    &&&&&&&&&& 1 & 1 & 0 & 1 \\
    \divline
    &&&&&&&&&&& 1 & 1 & 1 & 0 \\
    &&&&&&&&&&& 1 & 1 & 0 & 1 \\
    \divline
    &&&&&&&&&&&& 0 & 1 & 1 \\
  \end{array}
\end{equation*}

In this case the checksum is $\bin{011}$, or simply $3$. Notice that
the final bit size of the checksum ends up being equal to the width
$w$ of the generator polynomial. Also note that the computed quotient
is not at all interesting for computing the \crc.

\subsection{Generator Polynomials}

Authors of texts describing generator polynomials like to write out
them as hexadecimal numbers. In this text, we represent the polynomial
$x^3 + x^2 + x^0$ by the hexadecimal number \hex{d}($=\bin{1101}$). However,
since the last highest and lowest bits are mandatory, and could
therefore be seen as implicit, several authors like to leave them
out. The authors of \cite{press2007numerical_recipes} would for
example have written that last polynomial as \hex{5}($=\bin{0101}$), because
they considered the highest bit($x^{3}$) implicit; on the other hand,
the people behind \cite{Koopman04cyclicredundancy_embedded_networks}
would have written it as \hex{9}($=\bin{101}$), because they first of all
reversed the bit string to $\bin{1011}$, and they then chose to simply leave
out the lowest bit of the resulting bit string!

There are \textit{many} different \crc generator polynomials used. So
many that I quickly realized that it would become a too great of a
task to even summarize a few of them. This is mainly because the names
of these polynomials tend to be very inconsistent. According to
\cite{cook:_catal_crc} there are around 16 different \crcst
polynomials! These are polynomials whose width is $w=16$.

The advantages of various generators is very complex issue that we will
not cover in this text.

\subsection{Implementation}

Now we will finally discuss how to implement the \crc computation in
code.

Please inspect algorithm~\ref{alg:crc8-comp}. It implements
polynomial division for generator polynomials of width 8.

\begin{algorithm}[H]
  \caption{Computing a \crc of width 8}
  \label{alg:crc8-comp}
  \begin{algorithmic}[1]
    \Procedure{crc8}{$\var{bytes}, G(x)$}
    \Let{$\var{result}$}{$0$}

    \ForEach{$\var{byte}$}{$\var{bytes}$}
    \Let{$\var{result}$}{$\var{result} \BitXor \var{byte}$}

    \Repeatn{$8$}
    \linecomment{\hex{80} is the value of the highest bit in a byte.}
    \If{$\var{result} \BitAnd \hex{80}$}
    \Let{$\var{result}$}{$(\var{result} \ShiftLeft 1) \BitXor G(x)$}
    \Else
    \Let{$\var{result}$}{$\var{result} \ShiftLeft 1$}
    \EndIf
    \EndRepeatn

    \EndForEach

    \State \textbf{return}  $\var{result} \BitAnd \hex{FF}$
    \EndProcedure

  \end{algorithmic}
\end{algorithm}

Notice first how the numbers that we want to find the \crc for are not
even concatenated together to form one large binary number.  But the
reader will soon realize that it is not even necessary to do this to
calculate the \crc.  For a generator polynomial of width $w=8$, $8$
cleared bits will be added to the end of the data. It is these last
$8$ bits that will in the end contain the \crc. These 8 bits are in the
algorithm represented by the variable $\var{result}$. It is in other words
only the reminder we are after, and not the quotient.

So what will happen to the very first byte in the data of this
algorithm? Let us first consider what would happen with the byte $14 =
\bin{0000\ 1110}$. This bit pattern first gets bitwise xored into the
bit pattern of the resulting \crc, $\var{result}$. Since the initial
value of this variable was set to $0$ in the beginning of this
algorithm, this simply means that, by the definition of the bitwise
xor operator, the $\var{result}$ variable is set to the bit pattern
$\bin{0000\ 1110}$.

As familiar from the representation of long division in \crc, if the
highest bit of the numerator is cleared, then the numerator is divided
by the cleared bit pattern, and then the next bit highest bit of the
numerator is dealt with, meaning that the bit pattern is shifted left,
which is exactly what is done in this algorithm. This means that after
the first four steps in the repeat loop the four highest bits of the
$\var{result}$ variable are set to $1110$.

You may now suggest that we simply bitwise xor the generator
polynomial into $\var{result}$, because if the highest bit of the numerator
is toggled then it is subtracted by the generator polynomial. While it
is true that we now need to subtract by the generator polynomial, we
cannot simply use a bitwise xor operation in order to do this. This
would not work, because the length of the generator polynomial is $9$
bits, and the bit length of $\var{result}$ is $8$. So this would not
work. However, because the highest bit of the polynomial must always
be one, you can instead just bitwise left shift $\var{result}$ one
step. This way, the highest bit disspears of $\var{result}$, so this has the
exact same effect as subtracting would have. Then $\var{result}$ is bitwise
xored with the generator polynomial, to perform the rest of the
subtraction. Since The $8$:th bit of the generator polynomial is
outside the size boundaries of the $\var{result}$ variable it will not
affect this computation, hence why we had to bitwise left shift
$\var{result}$ before doing this.

This general algorithm is repeated for all the 8 bits of the byte. So
what this subalgorithm basically does, is that it divides one byte of
the data by the generator polynomial and gets the reminder of this
operation. For dealing with the next byte, this reminder stored in
$\var{result}$ is bitwise xored with the next byte, and the result of
this operation is processed using the repeat loop.

It may be a bit hard to see why this works, so consider the following:
we want to compute the \crc for the binary numbers $111$ and $011$ for
the generator polynomial $1100$. Using the traditional long division
polynomial algorithm, this would be done as such:

\begin{equation*}
  \newcommand{\divline}{\cline{5-14}}
  \renewcommand\arraystretch{1.2}
  \begin{array}{*4r @{\hskip\arraycolsep}c@{\hskip\arraycolsep} *9r}
    &&&&&&&& 1 & 0 & 1 & 1 & 0 & 1\\

    \divline
    1 & 1 & 0 & 0 & \big) & 1 & 1 & 1 & 0 & 1 & 1 & 0 & 0 & 0 \\
    &&&&& 1 & 1 & 0 & 0 & \\
    \divline

    &&&&&& 0 & 1 & 0 & 1 &&&\\
    &&&&&& 0 & 0 & 0 & 0 &&& \\

    \divline

    &&&&&&& 1 & 0 & 1 & 1 &&\\
    &&&&&&& 1 & 1 & 0 & 0 &&\\

    \divline

    &&&&&&&& 1 & 1 & 1 & 0 &&\\
    &&&&&&&& 1 & 1 & 0 & 0 &&\\

    \divline

    &&&&&&&&& 0 & 1 & 0 & 0 &\\
    &&&&&&&&& 0 & 0 & 0 & 0 &\\

    \divline

    &&&&&&&&&& 1 & 0 & 0 & 0\\
    &&&&&&&&&& 1 & 1 & 0 & 0\\

    \divline

    &&&&&&&&&&& 1 & 0 & 0\\

  \end{array}
\end{equation*}

However, using our algorithm, the checksum would be computed in the
following way: first we divide $111$ by the generator polynomial and
compute the reminder:

\begin{equation*}
  \newcommand{\divline}{\cline{5-11}}
  \renewcommand\arraystretch{1.2}
  \begin{array}{*4r @{\hskip\arraycolsep}c@{\hskip\arraycolsep} *6r}

    &&&&&&&& 1 & 0 & 1 \\

    \divline

    1 & 1 & 0 & 0 & \big) & 1 & 1 & 1 & 0 & 0 & 0 \\
    &&&&& 1 & 1 & 0 & 0 & \\

    \divline

    &&&&&& 0 & 1 & 0 & 0 \\
    &&&&&& 0 & 0 & 0 & 0 \\

    \divline

    &&&&&&& 1 & 0 & 0 & 0 \\
    &&&&&&& 1 & 1 & 0 & 0 \\

    \divline

    &&&&&&&& 1 & 0 & 0 \\

  \end{array}
\end{equation*}

Then, the next byte is subtracted by the reminder

\begin{equation*}
  \bin{100} - \bin{011} = \bin{111}
\end{equation*}

The result of this subtraction is divided by the generator polynomial:

\begin{equation*}
  \newcommand{\divline}{\cline{5-11}}
  \renewcommand\arraystretch{1.2}
  \begin{array}{*4r @{\hskip\arraycolsep}c@{\hskip\arraycolsep} *6r}

    &&&&&&&& 1 & 0 & 1 \\

    \divline

    1 & 1 & 0 & 0 & \big) & 1 & 1 & 1 & 0 & 0 & 0 \\
    &&&&& 1 & 1 & 0 & 0 & \\

    \divline

    &&&&&& 0 & 1 & 0 & 0 \\
    &&&&&& 0 & 0 & 0 & 0 \\

    \divline

    &&&&&&& 1 & 0 & 0 & 0 \\
    &&&&&&& 1 & 1 & 0 & 0 \\

    \divline

    &&&&&&&& 1 & 0 & 0 \\

  \end{array}
\end{equation*}

If we consider this example, then it is easy to see that dividing the
bytes by the generator polynomial separately will not affect the end
result. And therefore, algorithm~\ref{alg:crc8-comp} works. The
advantage that our algorithm has, is that does not require
concatenating the numbers into one large binary number, but rather
only requires being able to perform the operations on comparatively
small numbers.

It also turns out that this algorithm is general enough to allow for
$\var{result}$ to have bit sizes different than 8 bits. Bits outside of the
$7$:th byte boundary(the highest possible bit position of a byte) will
always stay outside this boundary and will not interfere with the 8
lowest bits, those that actually matter to the end result. But since
only the 8 lowest bits of the $\var{result}$ variable are part of the actual
\crc, you will need to find a way to only include these bits. This
trivially done using the bitwise and operation. By bitwise anding
$\var{result}$ with the pattern $\bin{1111\ 1111}$(=\hex{FF}), you can make sure
that you only get the 8 lowest bits.

There are many ways of varying the algorithm I just described. It is
for example common for $\var{result}$ to have different initial values
than zero. This results in a different \crc being computed. It is also
common to use different return values than \mbox{$\var{result} \BitAnd
  \hex{FF}$}. Many different values can be used instead of
\hex{FF}. Our advise is to always read the specification of a \crc
\textit{carefully} before starting to implement it.

The algorithm for computing \crc we just described turns out to be far
from optimal; there is a table based algorithm that is almost always
faster than the rather na\"{i}ve method we just described. But to keep
things simple we will not discuss it here.

Reasoning in very much the same, way can also implement the \crc
computation for generator polynomials of width $32$, as is shown in
algorithm~\ref{alg:crc32-comp}. As you can see, it is largely the same
algorithm.

\begin{algorithm}[H]
  \caption{Computing a \crc of width 32}
  \label{alg:crc32-comp}
  \begin{algorithmic}[1]
    \Procedure{crc32}{$\var{bytes}, G(x)$}
    \Let{$\var{result}$}{$0$}

    \ForEach{$\var{byte}$}{$\var{bytes}$}
    \Let{$\var{result}$}{$\var{result} \BitXor (\var{byte} \ShiftLeft 24)$}

    \Repeatn{$8$}
    \If{$\var{result} \BitAnd \hex{80000000}$}
    \Let{$\var{result}$}{$(\var{result} \ShiftLeft 1) \BitXor G(x)$}
    \Else
    \Let{$\var{result}$}{$\var{result} \ShiftLeft 1$}
    \EndIf
    \EndRepeatn

    \EndForEach

    \State \textbf{return}  $\var{result} \BitAnd \hex{FFFFFFFF}$
    \EndProcedure

  \end{algorithmic}
\end{algorithm}

\subsection{CRC-32}
\label{sec:pngcrc}

We will end our discussion on \crc by briefly discussing the \crc that
is used in the \png format as it is defined in
\cite{boutel:_png_portab_networ_graph_specif_version12}. The \crc
computation algorithm is given in
algorithm~\ref{alg:crc-png-comp}. The generator polynomial has a width
of $32$, so the \crc calculated by this algorithm is a 32-bit
number. As can be seen, the generator polynomial used is
\hex{EDB88320}. But this number represent the polynomial written out
in reversed order with the highest bit left out, so the polynomial
represented by this number is in fact

\begin{equation*}
  x^{32} + x^{26} + x^{23} + x^{22} + x^{16} + x^{12} + x^{11} + x^{10} + x^8 + x^7 + x^5 + x^4 + x^2 + x + 1
\end{equation*}

The peculiar thing about this \crc is that rather than working from the
highest bit to the lowest, this one works from the lowest bit to the
highest. For this reason, bitwise right shifts are used rather than
left shifts, and the lowest bit is checked rather than the
highest.

\begin{algorithm}[H]
  \caption{\crc computation for the \png format}
  \label{alg:crc-png-comp}
  \begin{algorithmic}[1]
    \Procedure{crcpng}{$\var{bytes}$}

    \Let{$G(x)$}{\hex{EDB88320}}
    \Let{$\var{result}$}{\hex{FFFFFFFF}}

    \ForEach{$\var{byte}$}{$\var{bytes}$}
    \Let{$\var{result}$}{$\var{result} \BitXor \var{byte}$}
    \Repeatn{$8$}
    \If{$\var{result} \BitAnd \hex{01}$}
    \Let{$\var{result}$}{$(\var{result} \ShiftRight 1) \BitXor G(x)$}
    \Else
    \Let{$\var{result}$}{$\var{result} \ShiftRight 1$}
    \EndIf
    \EndRepeatn

    \EndForEach

    \State \textbf{return}  $\var{result} \BitAnd \hex{FFFFFFFF}$
    \EndProcedure

  \end{algorithmic}
\end{algorithm}


\section{Adler-32}
\label{sec:adler32}

\newcommand{\adlerchk}{Adler-32\xspace}

Another example of a checksum is \adlerchk. It is for example used to
verify data integrity in the \zlib format
\cite{gailly96:_zlib_compr_data_format_specif}. The \zlib format is of
interest to us because that is the container format used to contain
the color data in \png images. It was named after its inventor Mark
Adler, one of the creators of the \zlib format.

Let us now describe \adlerchk based of the references
\cite{gailly96:_zlib_compr_data_format_specif,maxino:_revis_fletc_adler_check,DBLP:journals-tdsc-MaxinoK09_koopman}.

The algorithm to calculate the checksum is very simple: two variables
$a=1$ and $b=0$ are set in the beginning. Then, for every byte $D_i$
in the data the checksum is to be computed for, $a$ is set to $a +
D_i$ modulo the constant $65521$. This constant is the largest prime
number smaller than $2^{16}$. $b$ is then set to $a + b$ modulo the
same constant.  This process is repeated for every byte $D_i$ in the
data. Once every byte has been processed, the 16 highest bits of the
resulting checksum are set to $b$ and the 16 lowest bits are set to
$a$. Hence, the \adlerchk checksum is always a 32-bit number, as is
indicated by its name. In other words, for some data that has a length
of $n$ bytes the checksum is calculated as follows:

\begin{align*}
    & a = 1 + D_1 + D_2 + \ldots + D_n (\bmod{65521}) = 1 +
    \displaystyle\sum_{i = 1}^{n}D_i (\bmod{65521}) \\
    & b = (1 + D_1) + (1 + D_1 + D_2) + \ldots + (1 + D_1 + D_2 + \ldots +
    D_n) (\bmod{65521}) = \\
    & = n + \displaystyle\sum_{i = 1}^{n}(D_i \cdot (n + 1 - i))
    (\bmod{65521}) \\
    & \text{adler32} = (b \ShiftLeft 16) \BitOr a = (b \cdot 2^{16}) + a
\end{align*}

The last line may however need some explanation. It was back in
exercise~\ref{bit-equiv} on page~\pageref{bit-equiv} shown that $b
\ShiftLeft 16$ is equivalent to $b \cdot 2^{16}$. Bitwise or is
equivalent to plain addition in this case only because $b$ and $a$
were constantly taken modulo $65521$, meaning that they can never be
longer than $16$ bits. Hence, it is impossible for the separate bits
in the numbers $a$ and $b$ to interfere with each other in the bitwise
or operation, and therefore bitwise or is equivalent to plain addition
in this case.

It is also shown in algorithm~\ref{alg:adler32} how to calculate the
\adlerchk checksum.

\begin{algorithm}[H]
  \caption{Computation of the \adlerchk checksum}\algohack{}
  \label{alg:adler32}
  \begin{algorithmic}[1]
    \Procedure{adler32}{$\var{data}$}

    \Let{$a$}{$1$}
    \Let{$b$}{$0$}
    \ForEach{$\var{d}$}{$\var{data}$}
    \Let{$a$}{$a + \var{d} \bmod{65521}$}
    \Let{$b$}{$a + b \bmod{65521}$}
    \EndForEach
    \State \Return{$(b \ShiftLeft 16) \BitOr a$}
    \EndProcedure
  \end{algorithmic}
\end{algorithm}

\begin{Exercise}[label={ex-adler32}]

  Compute the \adlerchk checksum for the data $42,43,67$.

\end{Exercise}

\section{Answers To The Exercises}

\begin{Answer}[ref={bin-to-poly}]
  \begin{enumerate}[(a)]
  \item $x^7 + x^5 + x^4 + x + 1$
  \item $x^5+ x + 1$
  \item $x^4 + x + 1$
  \end{enumerate}
\end{Answer}

\begin{Answer}[ref={poly-to-bin}]
  \begin{enumerate}[(a)]
  \item \bin{1000\ 0001}
  \item \bin{1111}
  \item \bin{1001\ 1000\ 0001\ 1111}
  \end{enumerate}
\end{Answer}

\begin{Answer}[ref={width-poly}]

  \begin{enumerate}[(a)]
  \item $32$
  \item $3$
  \item $14$
  \end{enumerate}

\end{Answer}

\begin{Answer}[ref={poly-add}]

  $\bin{1010}$

\end{Answer}

\begin{Answer}[ref={poly-div}]

  \begin{equation*}
    \newcommand{\divline}{\cline{3-8}}
    \renewcommand\arraystretch{1.2}
    \begin{array}{*2r @{\hskip\arraycolsep}c@{\hskip\arraycolsep} *5r}
      &&&&  1 &  0 & 1 & 1\\
      \divline
      1 & 1 &\big)& 1 &  1 & 1 & 0 & 0 \\
      &&& 1 & 1 \\

      \divline

      &&&& 0 & 1 \\
      &&&& 0 & 0 \\

      \divline

      &&&&& 1 & 0 \\
      &&&&& 1 & 1 \\

      \divline

      &&&&&& 1 & 0 \\
      &&&&&& 1 & 1 \\

      \divline

      &&&&&&& 1 \\

    \end{array}
  \end{equation*}

\end{Answer}

\begin{Answer}[ref={ex-adler32}]

  $1848105$, because

  \begin{align*}
      & a = 1 +  \displaystyle\sum_{i = 1}^{3}D_i (\bmod{65521}) = 1 + 42 +
      43 + 67 (\bmod{65521}) = 153 \\
      & b = 3 + \displaystyle\sum_{i = 1}^{3}(D_i \cdot (3 + 1 - i))
      (\bmod{65521}) = 3 + 42 \cdot 3 + 43 \cdot 2 + 67  = 282 \\
      & §\text{adler32} = (b \cdot 65536) + a = 1848105
  \end{align*}

\end{Answer}

\endgroup