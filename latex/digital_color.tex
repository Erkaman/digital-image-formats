\begin{comment}
  \bibliography{project.bib}
\end{comment}

% http://www.w3.org/Conferences/WWW4/Papers/53/gq-gamut.html#fncol

\chapter{Digital Color}
\label{cha:color}

\cite{niederst1999webdesign}

% useful glossary.
% http://tigger.uic.edu/~hilbert/Glossary.html#Brightness

% Charles Poynton - Color technology. list of poyntons writings on color
% http://www.poynton.com/Poynton-color.html

% great source of papers.
% http://www.student.cs.uwaterloo.ca/~cs781/

% ``The colour gamut of a sample monitor (an HP A1097C)''
% http://www.w3.org/Conferences/WWW4/Papers/53/hp3.png

% the trichromatics color theory
% the opponent theory.

% GREAT walkthough of the trichromatic color theory, the eye, and the
% CIE color spaces.

\cite{Cotton95colour}

\section{Red, Blue and Green}
\label{sec:color-models}

% encylopedia.
% http://netghost.narod.ru/gff/graphics/book/ch02_02.htm

\subsection{What Is Color?}
\label{sec:what-color}

\newcommand{\bluewave}{\ensuremath{\SI{400}{\nano\meter}}}

Light is composed of tiny particles traveling at different
wavelengths\index{wavelength} \cite{neider93:_openg_progr_guide}.
Different colors are light traveling at different wavelengths. For
example, blue has a wavelength of \bluewave
\cite{ohlsson99:_digit_bild_kreat}.

But how do our eyes \textit{see} these light waves? In our eyes,
there are cells for perceiving three different kinds of wavelengths
of light: red, blue and green. When these cells absorb red, green or
blue light, we see color. And when these cells absorb mixed amounts
of red, blue and green light, we are able to perceive all the other
possible colors.

\subsection{RGB}
\label{sec:rgb}

Color models\index{color model} are ways of specifying colors
numerically \cite{fadgi11:color_model}. A very widely used color
model for representing color in computers is RGB.

\begin{figure}
  \centering
  \inputtikz{rgb.tex}
  \caption{The RGB color model}
  \label{fig:rgb}
\end{figure}

The RGB color model is based on how our eyes perceive color, which
we discussed in section \ref{sec:what-color}. As it can be seen in
figure \ref{fig:rgb}, the colors cyan ,magenta and yellow can be
achieved by mixing either red, blue and green. Also note that white
is achieved by mixing all of three colors. Black is represented by
no light at all. The rest of the colors can be achieved by mixing
different amounts of red,green and blue.

The RGB color model is called an additive\index{additive color model}
because it makes different colors by adding together light.

The different amounts of red, green and blue in a color, are called
their respective color channels\index{color channel}.

\subsection{RGBA}

% png definitive guide alpha channel
% http://www.vias.org/pngguide/chapter01_03_02.html

But there is actually even more to RGB. There is an extended color
model of RGB that's called RGBA. In this model, a new channel is
added: the alpha channel\index{alpha channel}. This new channel
represents the opacity of a color. Opacity is simply the opposite of
transparency \cite{porter84_compos_dig_img}.

Take a good look at figure \ref{fig:alpha}. This figure demonstrates
how a the color blue gets a lower and lower value of its alpha
channel until it turns fully transparent. In the leftmost part of
the image, the color fully has its maximum possible alpha channel
value, meaning that it's fully visible. On the other hand, the
leftmost color has its lowest possible alpha value, $0$, and as a
result is fully transparent.

\begin{figure}
  \centering
  \inputtikz{alpha.tex}
  \caption{The color blue with an increasingly lower alpha channel.}
  \label{fig:alpha}
\end{figure}

\section{Color Depth}
\label{sec:color-depth}

But up until now, I have said nothing about how the computer
represents these color models.

\subsection{24-bit Color}
\label{sec:24-bit-color}

In the RGB color model, each color is just a combination red,green
and blue channels. So, let us represent colors as a triple of three
numbers, like this: \rgbtrip{123}{21}{91}, which is
\textcolor[RGB]{123,21,91}{this color}.

An 8-bit number has, as you should be familiar with, only $2^8 =
256$ possible values. If we assign an 8-bit number to each color
channel, the total size of a single color will be 24-bits. This way
of representing color is called 24-bit color. You also refer to
24-bit colors as having a color depth\index{color depth} of
24-bits. This is also called the pixel depth\index{pixel depth} of
the color.

\newcommand{\selfcolor}[1]{\textcolor{#1}{#1}}

A full \selfcolor{red} color is represented by the triple \rgbtrip{255}{0}{0}. In
that case, have a guess at what color is represented by
\rgbtrip{255}{0}{255}. If haven't guessed it already, please study
figure \ref{fig:rgb}. As you probably already have guessed, this color
is \selfcolor{yellow}. And in the same number mixing and matching way, many other
colors can be represented. See table \ref{tab:color-examples} for
examples of other color values.

\begin{table}
  \newcommand{\colorrow}[4]{  \rgbtrip{#1}{#2}{#3} &
    \textcolor[RGB]{#1,#2,#3}{#4} \\}
  \centering
  \begin{tabular}{lr}
    \toprule
    RGB triple & Resulting color \\
    \midrule
    \colorrow{255}{215}{0}{Gold}
    \colorrow{165}{42}{42}{Brown}
    \colorrow{255}{0}{255}{Purple}
    \colorrow{255}{192}{203}{Pink}
    \colorrow{255}{165}{0}{Orange}
    \colorrow{0}{0}{139}{Dark Blue}
    \colorrow{105}{105}{105}{Dim Grey}
    \colorrow{0}{100}{0}{Dark Green}
    \colorrow{205}{92}{92}{Indian Red}
    \bottomrule
  \end{tabular}
  \caption{Colors}
  \label{tab:color-examples}
\end{table}

But of course, not all colors can be represented using just
24 bits. The number of \textit{all} possible colors is infinite. But you can still
represent an awful lot colors with only 24-bits. How many? Well, every
color channel can have $256$ different values. There are $3$
channels. Hence, there are $256^3 = 16777216$ different colors that
can represented using only 24-bits!

I was unable to find any good numbers of how many different colors us
humans can, but according to \cite{roth:_tetrachromat} we can see
about a million different colors, assuming  that we can pick 100
different gradations of red, blue and green.

As I stated in the introduction, raster graphics is going to be the
main subject of this book. And images represented as raster graphics
is simply just a grid of colors. Therefore, in this chapter well
discuss how these color are represented.


\newcommand{\rgbaquad}[4]{
  \mbox{(\textcolor{red}{#1},\textcolor{green}{#2},\textcolor{blue}{#3},\textcolor{gray}{#4})}}

Adding alpha channels to this color representation is trivial, just
add a fourth channel and 8-bit number. The only real
difference is that a color will require 32-bits of storage if an
alpha channel is needed.

\subsection{Grayscale Color}

\newcommand{\selfrgbtrip}[3]{\mbox{\textcolor[RGB]{#1,#2,#3}{(#1,#2,#3)}}}
\newcommand{\selfrgbtripgray}[1]{\selfrgbtrip{#1}{#1}{#1}}

If every color channel of a RGB color is set to the same value, the
resulting color will be a grayscale color. Using 24-bit color, we can for example make the following
grayscales:
\selfrgbtripgray{100},\selfrgbtripgray{210},\selfrgbtripgray{60}. Using
24-bits you can really only represent 256 different possible
grayscale colors because $256$ is how many different values can be
represented by an 8-bit number. So if we assign each of these RGB
triples to an 8-bit number, we have thus invented 8-bit
grayscale\index{8-bit grayscale}! So in this system, the number 80
would represent the color \selfrgbtripgray{80}, 180 would represent
\selfrgbtripgray{180}, and so on \cite{puglia00:_handbook_dig_proj}.

8-bit grayscale means that we have $2^8=256$ different shades of
gray. This system is actually applicable to any bit count, as is
demonstrated in table \ref{tab:grayscale}. You will probably meet at least one of
these grayscale color depths out in the wild.

\begin{table}
  \centering
  \begin{tabular}{ll}
    \toprule
    Bit count & Possible shades of gray \\
    \midrule
    $2$ & $2^2 = 4$ \\
    $4$ & $2^4 = 16$ \\
    $8$ & $2^8 = 256$ \\
    $16$ & $2^{16} = 65536$ \\
    \bottomrule
  \end{tabular}
  \caption{Different bit counts of grayscale}
  \label{tab:grayscale}
\end{table}

\section{Parsing Color Data}
\label{sec:parsing-color-data}

The number $26$ as a 8-bit number could be written as a string of
ones and zeroes, 00011010. From now on we will refer to such a
string as a bit pattern\index{bit pattern}.

Color is specified by bit ranges in numbers of certain
sizes. So how do we extract these bits? That is what are going to
discuss in the section.

\subsection{Grayscale}

The most trivial color type to parse is grayscale color. Since a
grayscale color is specified by just one number, there's no need to
do any parsing. As we discussed before, the number $x$ specifies the
grayscale color \mbox{($x,x,x$)}. \todo{discuss different bit depths
  of grayscale}

\subsection{RGB(A) Color}

Things do however become more interesting when we try to parse RGB
and RGBA color. There are several bit depths possible for RGB
color but we'll only be covering some of the most common ones.

\subsubsection{24-bit color}

\begin{figure}
  \centering
  {\huge\textcolor{red}{10001011}\textcolor{green}{01010010}\textcolor{blue}{\fullbyte}
    = \textcolor[HTML]{8B52FF}{purple}}
  \caption{The RGB bit pattern for a purplish color.}
  \label{fig:24-bit-colors-bits}
\end{figure}

The number 24-bit \textcolor{red}{1000 1011}\textcolor{green}{0101
  0010}\textcolor{blue}{\fullbyte} represents a purplish color. The
separate values of the color channels of this number can trivivally
be pased by reading them byte for byte. Because, remember, the each
color channel in 24-bit number is stored in a separate
byte. Algorithm \ref{alg:read-24-bit-rgb} demonstrates how to read
all the separate color channels.

\todo{discuss endianess}

\begin{algorithm}[H]
  \caption{Reading the color channels of RGB 24-bit number.}
  \label{alg:read-24-bit-rgb}
  \begin{algorithmic}[1]
    \Let{$R$}{\VoidCall{ReadByte}}
    \Let{$G$}{\VoidCall{ReadByte}}
    \Let{$B$}{\VoidCall{ReadByte}}
  \end{algorithmic}
\end{algorithm}

\subsubsection{32-bit color}

32-bit color is very much like 24-bit color. But the alpha channel
can either come before the RGB color values, ARGB\index{ARGB}, or
they can occur after the RGB color values, RGBA\index{RGBA}. Figure
\ref{fig:32-bit-colors-bits} shows both the variants. The first one
is ARGB och the second one RGBA. The separate alpha channel can be
parsed in the same way the separate color channels are parsed in
algorithm \ref{alg:read-24-bit-rgb}.

\begin{figure}
  \centering
  {\Large%
    \textcolor{gray}{11110000}%
    \textcolor{red}{10001011}%
    \textcolor{green}{01010010}%
    \textcolor{blue}{\fullbyte}

    \textcolor{red}{10001011}%
    \textcolor{green}{01010010}%
    \textcolor{blue}{\fullbyte}%
    \textcolor{gray}{11110000}%
  }

  \caption{The ARGB and RGBA variants of 24-bit color depth. The
    gray color represents the alpha channel }
  \label{fig:32-bit-colors-bits}
\end{figure}

\subsubsection{16-bit color}

Colors can also be stored in 16 bits, but it is relatively
rare. There are several variants of 16-bit color storage, so we'll
be covering some of them here.

\paragraph{TGA version}

The kind of 16-bit color used in the TGA format is shown in figure
\ref{fig:tga-16-bit-colors-bits}. It can be trivially parsed as is
shown in algorithm \ref{alg:read-16-bit-rgb}. What is unsual here is
that the values of the color channel's are stored in 5-bit number,
meaning that the maximum value of a color channel in this case is
$2^5 - 1 = 31$. However, here only one bit is reserved for the alpha
channel. A better name for this bit would a flag. And what this flag
is specify whether a color is visible or not.

\begin{figure}
  \centering
  {\huge\textcolor{gray}{0}%
    \textcolor{red}{10101}\textcolor{green}{11100}\textcolor{blue}{01010}}
  \caption{Sample 16-bit color.}
  \label{fig:tga-16-bit-colors-bits}
\end{figure}

\todo{discuss the bit patterns}

\begin{algorithm}[H]
  \caption{Reading TGA 16-bit number.}
  \label{alg:read-16-bit-rgb}
  \begin{algorithmic}[1]
    \Let{$B$}{$data \BitAnd 31$}
    \Let{$G$}{$(data \BitAnd (31 \ShiftLeft 5)) \ShiftRight 5$}
    \Let{$R$}{$(data \BitAnd (31 \ShiftLeft 10)) \ShiftRight 10$}
    \Let{$A$}{$(data \BitAnd (1 \ShiftLeft 16)) \ShiftRight 16$}
  \end{algorithmic}
\end{algorithm}

\section{Gamma}

\cite{roelofs99:_png}
\cite{boutel:_png_portab_networ_graph_specif_version11}
\cite{boutel:_png_portab_networ_graph_specif_version1}
\cite{boutel:_png_portab_networ_graph_specif_version12}
\cite{Pascale2003_ReviewRGBColourSpaces}
\cite{srgb}

% not just dcecoration w3c guide.
\cite{lilley:_not_just_decor}

\cite{motta1991analytical_crt}

% poynton book
\cite{poynton2003digital}

In computer and television screens the color from them was output
using a technology known as a CRT(cathode-ray tube). Although they are
now getting replaced by LCDs(Liciud Crytal Displays). But while in the
following discussion we'll be refering to CRT, all of things said
about CRTs can also be said about CRT
\cite{hearn1997computer_graphics,roelofs99:_png}.

For the CRT to light out a color of certain luminance, say a very dim
red color, it has to be given a certain voltage $V$. After it is given
a certain voltage, a corresponding luminance $L$ is shone out from the
CRT. Now, it turns that the luminance $L$ is \textit{not} directly
proportional to the voltage $V$. That is, there exists no constant $k$
such that $L = kV$. It rather turns out that the relationship between
these two can be expressed in the form of a power law. A power law is
any function $f(x)$ that can be written on the form

\begin{equation*}
  f(x) = Cx^k
\end{equation*}

Where $C$ and $k$ are arbitrary constants (negative sign necessary?)\cite{newman05power,easley2010networks_powerlaw}.

So the relationship between $L$ and $V$ is \textit{not} $L = kL$ for
some constant $k$, but it is $L = CV^k$ for two constant $k$ and
$C$. It turns out that the value of the constant $C$ is $0$, so the
expression can be written on an simple form as $L =
V^k$. \cite{motta1991analytical_crt,Pascale2003_ReviewRGBColourSpaces,boutel:_png_portab_networ_graph_specif_version1,boutel:_png_portab_networ_graph_specif_version11,boutel:_png_portab_networ_graph_specif_version12,roelofs99:_png}.

In practically all digital image literature the constant $k$ is
assigned the letter $\gamma$, so we'll be using that letter from now
on. Our final expression for describing the relationship the emitted
luminosity $L$ and the inputted voltage $V$ is thus:

\begin{equation}
  \label{eq:gamma}
  L = V^\gamma
\end{equation}

The rest of this section is now spend on discussing the significance
of equation (\ref{eq:gamma}).

Real imaging systems will have several components, and more than one
of these can be nonlinear. If all of the components have transfer
characteristics that are power functions, then the transfer function
of the entire system is also a power function. The exponent (gamma) of
the whole system's transfer function is just the product of all of the
individual exponents (gammas) of the separate stages in the system.

Also, stages that are linear pose no problem, since a power function
with an exponent of 1.0 is really a linear function. So a linear
transfer function is just a special case of a power function, with a
gamma of 1.0.

Thus, as long as our imaging system contains only stages with linear
and power-law transfer functions, we can meaningfully talk about the
gamma of the entire system. This is indeed the case with most real
imaging systems.

crt gamma = 2.5