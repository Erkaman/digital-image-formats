\begin{comment}
  \bibliography{project.bib}
\end{comment}

\chapter{Digital Color}
\label{cha:color}

% \cite{niederst1999webdesign}

As we stated in the introduction, raster graphics is going to be the
main subject of this text. And images represented as raster graphics
is simply a grid of colors. Therefore, in this chapter we'll discuss
how these color are represented.

\section{What is color?}

The following introductory discussion on color is based on
\cite{neider93:_openg_progr_guide,ohlsson99:_digit_bild_kreat,schanda97:_colorimetry,Cotton95colour,Pascale2003_ReviewRGBColourSpaces}

\newcommand{\bluewave}{\ensuremath{\SI{400}{\nano\meter}}}

Light is composed of tiny particles traveling at different
wavelengths. When light of different wavelengths hit
our eyes, they are processed so that we perceive these as
colors. Different colors we perceive as traveling at different
wavelengths. For example, blue has a wavelength of \bluewave.

Now, how do our eyes \textit{see} these light waves? In our eyes,
there are cone cells for perceiving three different kinds of
wavelengths of light: red, blue and green. When these cells absorb
red, green or blue light, we see color. And when these cells absorb
mixed amounts of red, blue and green light, we are able to perceive
colors that are mixtures of red, green and blue color.

\subsection{RGB}
\label{sec:rgb}

Color models are ways of specifying color numerically
\cite{hearn1997computer_graphics,Pascale2003_ReviewRGBColourSpaces}. A
very widely used color model for representing color in computers is
RGB.

\begin{figure}
  \centering
  \inputtikz{rgb.tex}
  \caption{The RGB color model}
  \label{fig:rgb}
\end{figure}

The RGB color model is based on how our eyes perceive color, which we
discussed in the previous section, so in RGB color made by mixing
different amounts of red, green and blue. As it can be seen in figure
\ref{fig:rgb}, the colors cyan, magenta and yellow can in this model
be achieved by mixing red, blue and green. Also note that white is
achieved by mixing all of three colors. Black is represented by no
color at all. The rest of the colors can be achieved by mixing
different amounts of red,green and blue.

Color channels describe the different amounts of the basic colors red,
green and blue in a color. So the amount of red in a color is
described by the red color channel. These are represented by numbers,
and how exactly this is done is discussed in section
\ref{sec:color-depth}.

\subsection{RGBA}

% png definitive guide alpha channel
% http://www.vias.org/pngguide/chapter01_03_02.html

But there is actually even more to RGB. There is an extended color
model of RGB that's called RGBA. In this model, a new channel is
added: the alpha channel. This new channel
represents the opacity of a color. Opacity is simply the opposite of
transparency \cite{porter84_compos_dig_img,murray1996encyclopedia,niederst1999webdesign}.

Take a good look at figure \ref{fig:alpha}. This demonstrates how a
color with the same values for its red, blue and green channels gets a
lower and lower value for its alpha channel. In the leftmost part of
the image the value of the alpha channel is at its maximum, so the
color is fully opaque. The more to the right we get in the figure the
lower and lower the value of the alpha channel gets, and the more
transparent the color becomes. And when the value of the alpha channel
is at its minimum the color turns fully transparent.

\begin{figure}
  \centering
  \inputtikz{alpha.tex}
  \caption{The color blue with an increasingly lower alpha channel.}
  \label{fig:alpha}
\end{figure}

\newcommand{\rgbtrip}[3]{\mbox{(#1,#2,#3)}}

\section{Color Depth}
\label{sec:color-depth}

But up until now, we have said nothing about how the computer
numerically represents these color models. In the following section
this is discussed for the RGB color model. The discussion is based on
\cite{murray1996encyclopedia,neider93:_openg_progr_guide,niederst1999webdesign,sitts2000handbook}.

\subsection{24-bit Color}
\label{sec:24-bit-color}

\subsubsection{RGB triplets}


In the RGB color model, each color is just a combination of red, green
and blue color, So every color can be represented as a RGB triplet
$(R, G, B)$. Since RGB triplets are specified by numbers they are
quite difficult to imagine visually, but they are on the other hand
very easy for the computer to interpret. It can trivially display a
RGB triplet on the screen by combining different ammounts of red, blue
and green light.

In 24-bit color every color channel is assigned to an 8-bit byte,
meaning that every single color will take up 24 bits of storage, since
there are $3$ color channel. You also refer to 24-bit colors as having
a color depth of 24 bits. This is also called the pixel
depth of the color.

Using this color depth, the maximum value of a channel will be $2^8 -
1 = 255$, so the reddest of red is represented by the RGB triplet
$(255,0,0)$.

\begin{Exercise}[label={rgb-triplet}]
  What color does each of these RGB triplets represent?

  \begin{enumerate}[(a)]
  \item $(0,0,255)$
  \item $(0, 255, 0)$
  \item $(255,255,0)$
  \item $(255,255,255)$
  \item $(0,0,0)$
  \item $(0,255,255)$
  \item $(255,0,255)$

  \end{enumerate}
\end{Exercise}

\subsubsection{Numbers instead of triplets}

The way this RGB triplet is stored in a file is simple: since every
channel in the triplet is a byte, it is stored as a sequence of 3
bytes.

But you can also see it this way: the triplet is stored as the 24-bit
number $R \cdot 256^2 + G \cdot 256 + B$. Here the 8 highest bits of
the number stores the red channel, the 8 middle bits store the green
channel, and the 8 lowest bits stores the blue channel, so that the
color yellow,$(255,255,0)$, would be stored as the 24-bit number $255
\cdot 256^2 + 255 \cdot 256 + 0 = 16776960$, which is equal to the
binary number $11111111\ 11111111\ 00000000$.

\subsubsection{Channel ordering}

As is stated in \cite{murray1996encyclopedia}, there's nothing from
stopping you from storing the color channels in another than order R,G
then B. B,G,R is an order that is also commonly used. The
specification of the image format usually specifies in which order the
color channels are stored, so do remember to read the specification
carefully.

\subsubsection{Good enough?}

So how many different colors can you represent using 24-bit color?
Every color channel can have $2^8 = 256$ different values and there
are $3$ channels; hence, $256^3 = 16777216 \approx 16\ 000\ 000$ different colors
can represented using 24-bits.

Then how many different colors can us humans see? Scientists do not
really know the answer to that question yet. Popular guesses include
one million \cite{roth:_tetrachromat} and $2^{24} = 256^3$
\cite{murray1996encyclopedia}.

\subsection{RGBA}

Adding alpha channels to this color representation is trivial, just
add a fourth channel by adding another 8-bit byte. So RGBA color data
is in other words stored as a quadruplet $(R,G,B,A)$. And as I stated
before, there is absolutely nothing stopping you from storing these
channels in a different order. The $(A,R,G,B)$ and $(A,B,G,R)$ orders
are also fully acceptable.

\subsection{Other Color Depths}
\label{sec:other-channel-sizes}

\subsubsection{48- and 64-bit color}

Other kinds of numbers than 8-bit numbers can be assigned to the
separate channels as well. Assigning a 16-bit number to each channel
is also common, and is supported by the PNG
format\cite{boutel:_png_portab_networ_graph_specif_version12}, which
we'll discuss in chapter \ref{cha:png}. This is for obvious reasons
known as 48-bit color. By adding an alpha channel to this model get
64-bit color.

In 48-bit color the maximum value of a color channel is obviously
$2^{16} - 1 = 65535$, meaning that the reddest of red is represented
by the triplet $(65535, 0, 0)$

\subsubsection{16-bit color}

A color depths of 16 is also used sometimes. TGA format supports
16-bit color \cite{91:_truev_tga_file_format_specif}. Note that there
are actually several ways of representing 16-bit color, but in this
section we'll only discuss how this is done in the TGA format.

In the TGA version of 16-bit color, the single highest bit number 15
is used to store transparency information. If this bit is toggled, the
color is visible, otherwise it is fully invisible. The RGB color
channels are given 5 of storage bits each. Bits 0-4 are reserved for
blue, 5-9 are given to green, and 10-14 are assigned to red.

Most programming languages do not support reading numbers from a file
whose bit lengths are not multiples of 8. It is therefore not possible
to directly read the 1-bit and 5-bit numbers that 16-bit color
consists of. These numbers have to be extracted using the bitwise
operators. In algorithm \ref{alg:read-16-bit-rgb} it shown how to
extract the separate channel values out of a 16-bit color value that
has been stored in a variable named $color$.

\begin{algorithm}[H]
  \caption{Reading the separate color channels of a TGA 16-bit color
    value.}
  \label{alg:read-16-bit-rgb}
  \begin{algorithmic}[1]
    \Let{$R$}{\Call{getbits}{$color, 10,14$}}
    \Let{$G$}{\Call{getbits}{$color, 5,9$}}
    \Let{$B$}{\Call{getbits}{$color, 0,4$}}
    \Let{$A$}{\Call{getbits}{$color, 15,15$}}
  \end{algorithmic}
\end{algorithm}

\begin{Exercise}[label={tga-16-bit}]

  In a TGA image file the following 16-bit color values were found:

  \begin{enumerate}[(a)]
  \item $64435$
  \item $31744$
  \item $31775$
  \end{enumerate}

  What colors does these numbers represent? Also, indicate whether
  these colors are transparent or opaque.

\end{Exercise}

\subsection{Grayscale Color}

\newcommand{\selfrgbtrip}[3]{\mbox{\textcolor[RGB]{#1,#2,#3}{(#1,#2,#3)}}}
\newcommand{\selfrgbtripgray}[1]{\selfrgbtrip{#1}{#1}{#1}}

Using 24-bit as an example, if a color is grayscale that means that
the color is represented by a RGB triplet $(n,n,n)$, where $0 \le n
\le 255$; that is, a shade of gray is a mixture of equal amounts of
all the color channels. In 8-bit grayscale color the triplet $(n,n,n)$
is simply represented by the number $n$, because this is much more
space efficient. So in this new system, the number $43$ will in
reality represent the triplet $(43,43,43)$, $12$ is really the triplet
$(12,12,12)$ and so on.

In 8-bit grayscale there are $2^8 = 256$ different shades of gray. In
1-bit grayscale, which is also quite common, there are only $2^1 = 2$
different shades of gray: black and white. And stated as generally as
possible, this means that in n-bit grayscale there are $2^n$ different
shades of gray. So there can indeed be many different kind of
greyscale, the PNG format, for example, supports supports grayscale of
the bit depths $1$, 2, 4, 8 and
16\cite{boutel:_png_portab_networ_graph_specif_version12}.

\subsection{Color palettes}

And finally, one last way of storing color is by using color
palettes. A color palette is an array of colors that is specified at
the beginning of the image. These colors are most typically stored as
24-bit RGB triplets. The image data in an image that uses a color
palette consists not of a sequence of RGB triplets nor grayscale
color, but of a sequence of indexes to the palette.

For example, say an image had the palette $(Red, Purple, Black,
Green)$. Then the color Purple in the image would be represented by
the number $1$. And note that $1$ is a \textit{zero based} index to
the palette, and not one based. The color data using this palette can
use a sequence of 2-bit values to store the color data.

Color palettes are most useful when an image uses a limited subset of
colors. Say a 10 by 10 image uses only 4 different colors and that
these colors are stored in 24-bit RGB triplets. Then the color data
requires $100 \cdot 3 \cdot 8 = 2400$ bits of storage. If the image on
the other hand put these four colors into a palette at the beginning
of the image, then the these colors could be represented as 2-bit
numbers in the image data instead. Then it would only take $2 \cdot
100 + 4 \cdot 3 \cdot 8 = 296$ bits to store the image. Note that
storing the palette requires an extra overhead of $96$ bits in this
case. So for the usage of palettes to really pay off, the space gained
by using a palette has to compensate for the extra overhead of storing
it. This has the important consequence that for small images a palette
is rarely a good idea.

\section{Color Layout}
\label{sec:color-layout}

Now how are these colors laid out in a typical image file? They tend
to be stored in a linear fashion rather than a two-dimensional
way. This means that the separate rows of an image are not at all
indicated.

Let us for example again consider the image on the front page. The
image data(which uses a palette) of this image would be stored as the
sequence $4,4,4,4,,0,4,4,3,4,\dots$ and so on in an entirely linear
fashion. The program reading the image can then split it up into rows
using the earlier given measurement values(width:$5$, height:$6$). It
should from this be obvious that there is no real need for an image
format to split up an image into separate rows.

It is also easy to see how the separate colors can be accessed in this
model. Let us say that we stored the color data of the image on the
front page in the array $data$. The indexes of this array are
zero-based, meaning that, for example, $data[3]=4$, while
$data[4]=0$. The index to the array of the color in row $r$ and column
$c$(these numbers are also zero-based) is then given by $r \cdot 5 +
c$. The index of the red color in the middle of the flower is then
given by $2 \cdot 5 + 2 = 12$. This is also the number you end up with
if you start counting(from $0$) down from the top left corner of the
image row by row to that color.

\answers{}

\begin{Answer}[ref={rgb-triplet}]
  \begin{enumerate}[(a)]
  \item Blue
  \item Green
  \item Yellow
  \item White
  \item Black
  \item Cyan(light blue)
  \item Magenta(purplish red)
  \end{enumerate}
\end{Answer}

\begin{Answer}[ref={tga-16-bit}]

  \begin{enumerate}[(a)]
  \item A fully opaque white color.
  \item A transparent red color.
  \item A transparent magenta color.
  \end{enumerate}

\end{Answer}
