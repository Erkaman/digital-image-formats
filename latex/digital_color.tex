\begin{comment}
  \bibliography{project.bib}
\end{comment}

\chapter{Digital Color}
\label{cha:color}

\begin{refsection}

  \section{Color models}
  \label{sec:color-models}

  \subsection{What is color?}
  \label{sec:what-color}

  \newcommand{\bluewave}{\ensuremath{\SI{400}{\nano\meter}}}

  Light is composed of tiny particles traveling at different
  wavelengths \index{wavelength}. They are also called light waves
  \index{light waves} \cite{neider93:_openg_progr_guide} Different colors
  are light traveling at different wavelengths. For example, blue has
  a wavelength of \bluewave. \cite{ohlsson99:_digit_bild_kreat}

  But how do our eyes \textit{see} these light waves? In our eyes,
  there are cells for perceiving three different kinds of wavelengths
  of light: red, blue and green. When these cells absorb red,green or
  blue light, we see color. And when these cells absorb mixed amounts
  of red, blue and green light, we are able to perceive \textit{all}
  the other possible colors.

  \subsection{RGB}
  \label{sec:rgb}

  Color models \index{color model} are ways of specifying colors
  numerically \cite{fadgi11:color_model}. A very widely used color
  model for representing color in computers is \textbf{RGB}
  \index{RGB}.

  \begin{figure}[h]
    \centering
    \inputtikz{rgb.tex}
    \caption{RGB color model}
    \label{fig:rgb}
  \end{figure}

  The color model is, of course, based on how our eyes perceive color,
  as mentioned in section \ref{sec:what-color}. Observe figure
  \ref{fig:rgb}. As it can be seen, all the colors cyan ,magenta and
  yellow can be achieved by mixing either red, blue and green. Also note
  that white is achieved by mixing all of three colors. Black is
  represented by no light at all.

  The RGB color model is called additive \index{additive color model} because makes different color
  by adding together light.

  The different amounts of red ,blue and green in a color, are called
  their respective color channels \index{color channel}.

  \subsection{CMYK}
  \label{sec:cmyk}

  Instead of mixing light,we can spread color out on paper to make all
  the different colors. Doing this, we have created a new color model:
  \textbf{CMYK} \index{CMYK}. It stands for \textbf{C}yan,
  \textbf{M}agenta, \textbf{Y}ellow and blac\textbf{K}.

  It is subtractive \index{subtractive color model}, meaning that
  instead of adding light, we subtract the light by adding more and
  more ink to the paper.

  \subsection{Alpha channels}
  \label{sec:alpha_chan}

  But there is actually even more to RGB. There is an extended color
  model of RGB that's called RGBA \index{RGBA}(it is sometimes also
  referred to as ARGB \index{ARGB}). In this model a new channel is
  added: the alpha channel \index{alpha channel}. This new channel
  represents the opacity of a color. Opacity is simply the opposite of
  transparency \cite{porter84_compos_dig_img}. Take
  a good look at figure \ref{fig:alpha}. It demonstrates how a square
  gets a lower and lower value of its alpha channel until it turns
  transparent. In the leftmost part of the rectangle , the color has
  its maximum possible alpha channel, meaning that it's fully
  visible. On the other hand, the leftmost color has its lowest
  possible alpha value, $0$, meaning that it's fully transparent.

  \begin{figure}[h!]
    \centering
    \inputtikz{alpha.tex}
    \caption{A rectangle with a lower and lower alpha channel.}
    \label{fig:alpha}
  \end{figure}

  \section{Color depth}
  \label{sec:color-depth}

  \newcommand{\rgbtrip}[3]{( \textcolor{red}{#1},\textcolor{green}{#2},\textcolor{blue}{#3})}

  But up until now, I have said nothing about how the computer
  represents these color models. Let us start with:

  \subsection{24-bit color}
  \label{sec:24-bit-color}

  RGB color model each color is just a combination red,blue and
  green. In the RGB color model these channels are given values. So let
  us represent colors as a triple of three numbers, like this:
  \rgbtrip{123}{21}{91}, which is \textcolor[RGB]{123,21,91}{this color}.

  An 8-bit number has, as you should be familiar with, only $256$ possible values. If we
  assign an  8-bit number to each color channel, the color's total size
  will be 24-bits. This way of representing color is called 24-bit color
  \index{24-bit color}. You can also say that its color depth \index{color
    depth} is 24 bits.

  \newcommand{\selfcolor}[1]{\textcolor{#1}{#1}}

  A full \selfcolor{red} color is represented by the triple \rgbtrip{255}{0}{0}. In
  that case, have a guess at what color is represented by
  \rgbtrip{255}{0}{255}. If haven't guessed it already, please study
  figure \ref{fig:rgb}. As you probably already have guessed, this color
  is \selfcolor{yellow}. And in the same number mixing and matching way, many other
  colors can be represented. Study table \ref{tab:color-examples} for
  examples of other color mixes.



  \begin{table}[h!]
    \newcommand{\colorrow}[4]{  \rgbtrip{#1}{#2}{#3} &
      \textcolor[RGB]{#1,#2,#3}{#4} \\}
    \centering
    \begin{tabular}[h!]{lr}
      \toprule
      RGB triple & Resulting color \\
      \midrule
      \colorrow{255}{215}{0}{Gold}
      \colorrow{165}{42}{42}{Brown}
      \colorrow{255}{0}{255}{Purple}
      \colorrow{255}{192}{203}{Pink}
      \colorrow{255}{165}{0}{Orange}
      \colorrow{0}{0}{139}{Dark Blue}
      \colorrow{105}{105}{105}{Dim Grey}
      \colorrow{0}{100}{0}{Dark Green}
      \colorrow{205}{92}{92}{Indian Red}
      \bottomrule
    \end{tabular}
    \caption{Colors}
    \label{tab:color-examples}
  \end{table}

  But of course, not all colors can be represented using just
  24 bits. The number of \textit{all} possible colors is infinite. But you can still
  represent an awful not colors with only 24-bits. How many? Well, every
  color channel can have $256$ different values. There are $3$
  channels. Hence, there are $256^3 = 16777216$ different colors that
  can represented using only 24-bits!
g
  \newcommand{\rgbaquad}[4]{(
    \textcolor{red}{#1},\textcolor{green}{#2},\textcolor{blue}{#3},\textcolor{gray}{#4} )}

  Adding alpha channels to this color representation is trivial, just
  add a fourth channel. Here is for example the color green halve
  transparent: \rgbaquad{0}{0}{255}{125}. The only real difference is
  that a color will require 32-bits of storage if an alpha channel is needed.

  \subsection{Grayscale}
  \label{sec:other-colors-depths}

  \newcommand{\selfrgbtrip}[3]{\textcolor[RGB]{#1,#2,#3}{(#1,#2,#3)}}
  \newcommand{\selfrgbtripgray}[1]{\selfrgbtrip{#1}{#1}{#1}}

  If every color channel of a RGB color is set to the same value, the
  resulting color will be a grayscale color \index{grayscale color},
  and nothing else. Using 24-bit color, we can for example make the
  following grayscales:
  \selfrgbtripgray{100},\selfrgbtripgray{210},\selfrgbtripgray{60}. Using
  24-bits you can really only represent 256 different possible
  grayscale colors. 256 is how many different values can be
  represented by an 8-bit number. So if we assign each of thse rgb
  triples to a number that can be fitted into 8 bits, we have thus
  invented 8-bit grayscale \index{8-bit grayscale}! So in this system,
  the number 80 would represent the color \selfrgbtripgray{80}, 180 by
  \selfrgbtripgray{180}, and so on. \cite{puglia00:_handbook_dig_proj}

  So 8-bit grayscale means that we have $2^8=256$ different shades of
  gray. This system is actually applicable to any bit count, as is
  demonstrated in table \ref{tab:grayscale}. And you will probably meet at least one of
  these grayscale color depths out in the wild. 8-bit grayscale is not
  really the grayscale that is used.

  \begin{table}[h!]
    \centering
    \begin{tabular}{ll}
      \toprule
      Bit count & Possible shades of gray \\
      \midrule
      $2$ & $2^2 = 4$ \\
      $4$ & $2^4 = 16$ \\
      $8$ & $2^8 = 256$ \\
      $16$ & $2^{16} = 65536$ \\
      \bottomrule
    \end{tabular}
    \caption{Different bit counts of grayscale}
    \label{tab:grayscale}
  \end{table}

  \subsection{Indexed Color}
  \label{sec:indexed-color}

  % 8-bit color 1-bit color, palettes need to be covered.-

  \printbibliography[heading=subbibliography]
\end{refsection}