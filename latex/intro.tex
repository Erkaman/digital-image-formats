
\chapter{Introduction and Background}

\section{Introduction}

One day a couple of years ago, I accidentally opened an executable
file(.exe) on my computer in a text editor. In front of my eyes, I saw
an absolutely incomprehensible mess of characters and numbers that no
matter how much you examined it, you could see no structure in the
chaos. This event really got me thinking about one thing: how exactly
are all the data stored in the computer actually stored?

A couple of months after this, I learned about the concept of a hex
dump.A hex dump is a hexadecimal representation of the data in a
file. It was at this point I finally realized something that should
have been obvious to me: since the language that the computer talks in
is just a bunch of ones and zeroes, then obviously all the data stored
in a computer is also just a bunch of ones and zeroes. And
furthermore, saying that a computer speaks in ones and zeroes, is just
another way of saying that it speaks in \textit{numbers}. Since I have
always been very much fascinated by mathematics, this got me even more
interested in how data is stored in a computer.

Ever since I came upon this realization, I very often started to make
hexdumps of the files on the computer and start examining their
contents, and trying to figure out exactly how they work. But I
quickly realize that doing this would require great knowledge of how
different kinds of data worked, something which I did not yet have.

Every kind of way of storing a kind of data is called a \textit{file
  format}. How these formats are structured is almost always defined
in their respective specifications. But alas, reading these documents
often required the reader already being knowledgeable in the kinds of
data that the formats stores to begin with.

So I realized that it would require a great ammount of research in
order to truly understand how image formats work. In the summer of
2011, s small part was spent trying to understand how image of and
sound formats work. It was very much fun learning how these kinds of
formats work, but at the same time I realized that time was running
out, and that I needed to quickly come up with an idea of what kinds
school project I should do for the next term.

And then it suddenly struck me: exploring how file formats work would
be a great idea for a school project! But since image formats are very
complex and often require a great of ammount research in order to be
understood, I realized that I needed to limit myself to one specific
area of image formats, in order to make the project more manageable.

In the end, image formats seemed to be the most manageable choice and
that is how this text came to be.

\section{Background}

Designing an image format is not as easy it seems to be; first of all,
you need decide in how flexible you want your format to be, that is,
how many different kind of color storage you want your format to
support. And second, you need to decide on what compression scheme to
use. Compression schemes are for making sure that the image doesn't
take up as much space as it usually would. If images were never
compressed, they would up way too much space and we would be unable
store but a few on our harddrives. The method used does not only have
to be efficient, it has to be fast. The restoration of the compressed
data to its original form has to be fast.???

There are other miscellaneous things image formats are often expected
to support. Some kind of gamma suppot for example??

Another useful thing are checksums. A checksum doesn't necessaly have
to be sum, but it has to be computed from the numerical color data to
be one. Checksums are very useful for veryfying that the image is not
corrupt and that it is in it's original form.

\section{Sample source code}

Accompanying most chapters are command line demo programs that
demonstrates the techniques discussed in the chapter. These programs
are written in \C because I thinks that language fits the subject of
this book the best. The source of these can be found in the
\path|code| directory. If you know how to use the \verb|make| program
you can easily compile these programs. Otherwise I'll assume that you know how
to compile these programs.

\todo{give a list of which source directories are associated with
  their respective chapter. Put instructions in a textfile on how to
  use the code in each directory. }

To get instructions of how to use these programs, pass the command
line flag \verb|--help| to them.
