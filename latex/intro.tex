\begin{comment}
  \bibliography{project.bib}
\end{comment}


\chapter{Introduction and Abstract}

\section{Introduction}

One day a couple of years ago, I accidentally opened an executable
file(.exe) on my computer in a text editor. In front of my eyes I saw
an absolutely incomprehensible mess of characters and numbers. No
matter how much I examined this mess, I could see no order in this
absolute chaos. This event really got me thinking about one thing:
\textit{how} exactly are all the data stored in a computer actually
\textit{represented}?

A couple of months after this, I learned about the concept of a hex
dump. A hex dump is a hexadecimal representation of the data in a
file. It was at this point I finally realized something that should
have been obvious to me: since the language that the computer talks in
is just a bunch of ones and zeroes, then obviously all the data stored
in a computer is also just a bunch of ones and zeroes! And
furthermore, saying that a computer speaks in ones and zeroes is just
another way of saying that it speaks in \textit{numbers}. Since
mathematics have always been one of my great interests in life, this
episode served greatly to strengthen my interest in how data is
represented in a computer.

Ever since I came upon this realization, something I very much like to
do have been viewing the hexdumps of the files on my computer, and
trying to figure out exactly how they work. %But I quickly realized that
%doing this would require great knowledge of how the data stored in the
%file is

Every kind of way of storing a certain type of data is called a
\textit{file format}. The exact structuring of a file format tends to
be specified in something called that file formats
\textit{specification}. But alas, reading these documents often
requires the reader already being moderately knowledgeable about the
data kind of data that the file format stores.

And so I realized that it would probably require a great amount of
research in order to truly understand how file formats work. But I
still didn't want to give up; my thirst for knowledge was to great for
such an action.  For this reason a spent a great deal of the summer of
2011 in exploring how two kinds of file formats work: image and sound
formats. And while I made progress in exploring how these kinds of
formats work, progress was being made too slow. The Time of the summer
was running out, and I was at the same time desperately trying to come
up with what sort school project I should do for my last term.

And then it suddenly struck me: exploring how one kind of file format
works would be a great idea for a school project! I needed to make a
choice: image or sound formats? but the real questiosn was rather:
which kind of format would be the most manageable to explore? Then the
choice was simple: image formats, because while sound formats are
intesting the math behind them is way beyond at the current I was
at. And while image formats are relatively simple, you need to have
some moderate knowledge from several different areas in order to fully
understand them. These areas are:

\begin{itemize}
\item Information Theory
\item Color Science
\end{itemize}

And Color Science was by far the hardest of these two areas to study,
which is why only a small segment of this text is dedicated to this
area, while pretty must the rest of the book is dedicated to information
theory.

\section{Abstract}

The cover picture of this project was made to illustrate the main
point of this project: the representation of images in a computer is
just a list of numbers. The cover image is a very simplified model of
all image formats are built and I will spend the rest of this section
on explaining the numbers that this image consists of. I will here
write a specifciation of the image format on the front page,that will
from now on be known as IM, the \textit{IM}aginary image format; that
is to say, it is a made up format that was designed for me for
illustrative purposes.

\subsection{Image Header}

The first gray section of the image is known as the image header of
the image. Let us go through to parts of the header number by number:

\subsubsection{Magic Numbers}

The first two numbers are the two letters IM. Even letters are
represented by numbers in a computer. How these letters are mapped to
different numers is determined by the encoding of these letters. An
encoding is basically a table that maps all the letters and
miscellaneous symbols of the human language to respective numbers. One
common encoding for is known as ASCII \cite{rfc20}, and in ASCII the
letter I is represented by the number 73 and the letter I is mapped to
the number 77. So if these letters where encoded in ASCII a better way
to represent them would have been to use the numbers 73 and 77. But I
did not do this because it is much easier for the reader to understand
letters than numbers.

Another reason why I used letters instead of number is that there are
so many different encoding. ASCII only has support for english
language character, and there are other encodign that ar designed to
support all the symbols for all human languages, the most of example
is Unicode\cite{allen2007unicode}. But since it would have been
bothersome for the reader to guess the encoding, I prefered to simply
use letters instead.

So these two first numbers are known as magic numbers \index{magic
  numbers}. At the beginning of every image file, there are a couple
bytes called magic numbers that are common to all images of that
type. They are sort like file name extensions, meaning that they help
you identify the type of the file you are currently reading.

\subsubsection{Measurements}

And following the magic number is the width and the height of this
image. In practically every image format this information is
included.

\subsubsection{Metadata}

But images can not only contain image data. They can also contain
something known as metadata. Metadata is simply data that describes
the image data. It contains no data that is necessary to understand
how the image data within image is represented, it rather gives
informtion about things like:

\begin{itemize}
\item The program used to create the image
\item A descriptive name for the image.
\item The creator/producer of the image.
\item The creation date of the image.
\item The camera that was used to take the image.
\end{itemize}

One relatively well known method for embedding metadata about the
camera used to take an image is known as
Exif\cite{camera:_cipa_dc_trans_exchan}. But since Exif is a quite
complex method for storing such data, we will not discuss it in this
text. We will only be concerned in exploring how to extract simple
metadata, like the creation date of the image and the producer of the
image.

\subsection{Color Palette}

Color palettes \index{color palette} are also very common, but not all
images out in the wild use them, but as good all image formats support
them, hence why they are included in this imaginary format. A color
palette is basically an array or list of colors that is to be used by
the image. The colors themselves are represented numerically using a
so called color model. Color models are something we will discus much
in depth in chapter \ref{cha:color}. The numbers in the palette is the
numbers that the color will be represented by in the color data. In
reality, only the color data is stored in the palette, and no numbers
are they given. They are only written to make the image easier to
understand, and not something that is found in ordinary images.

\subsection{The Color Data}

Following the header and the optional color palette is the actual
color data. Now let us talk about how the colors are stored as an
image.

An image is simply a grid of colors. As you can see on the front cover
image, an image really only consists of a lot of colored
squares. These squares as referred to as pixels
\index{pixel}\cite{murray1996encyclopedia}.

So as we can see from our sample image format, IM, the flower image
is just a couple of colored pixels. But in this case a color palette
is used. This means the image indexes the color palette like an
array or list. So instead of the image containing its actual color,
it in this case just contains a bunch of index numbers.

If a color palette were not to be used in this case, the image would
just contain a grid of raw colors. The color would in that case have
represened like they were in the palette; that is, using a color
model.

\subsection{Done\dots?}

But it's not really that simple. A majority of all image format have
had some sort of compression applied to their color data. Compression
is necessary because the raw color data usually tends to take up a lot
of space. And performing the decompression of that image data is
usually the most difficult part of understanding an image format. So
most time of this text will be dedicated on understanding how these
compression algorithms work. Compression algorithm is a major part of
the area of \textit{information theory}, which will be the main topic
of this text. To understand color models however, you will some
knowledge of the area of \textit{color science}, which is what chapter
\ref{cha:color} of this text will be dedicated to.
