\begin{comment}
  \bibliography{project.bib}
\end{comment}

\chapter{Introduction and Abstract}

\section{Introduction}

One day a couple of years ago, I accidentally opened an executable
file(.exe) on my computer in a text editor. In front of my eyes I saw
an absolutely incomprehensible mess of characters and numbers. No
matter how much I examined this mess, I could see no order in it. This
event really got me thinking about one thing: \textit{how} exactly is
all the data stored in a computer actually \textit{represented}?

A couple of months after this, I learned about the concept of a hex
dump. A hex dump is a hexadecimal representation of the data in a
file. It was at this point I finally realized something that should
have been obvious to me: since the language that the computer talks in
is just a bunch of ones and zeroes, then obviously all the data stored
in a computer is also just a bunch of ones and zeroes! And
furthermore, saying that a computer speaks in ones and zeroes is just
another way of saying that it speaks in \textit{numbers}. Since
numbers, and mathematics in general, have always been one of my
greatest interests in life, this episode served to greatly strengthen
my interest in how data is represented in a computer.

A way of storing a certain type of data is called a \textit{file
  format}. The exact structure of a file format tends to be specified
in something called that file format's \textit{specification}. But
alas, reading these documents often requires the reader already being
moderately knowledgeable about the kind of data that the file format
stores.

And so I realized that it would probably require a great amount of
research in order to truly understand how file formats work. But I
still didn't want to give up; my thirst for knowledge was too great
for such an action.  For this reason a spent a great deal of the
summer of 2011 in exploring how two different categories of file
formats work: image and sound formats. And while it is true that I
made progress in exploring how these kinds of formats work, progress
was being made too slow. The summer was coming to an end, and I was at
the same time desperately trying to come up with what sort school
project I should do for my last term.

And then it suddenly struck me: exploring how one category of file
format works would be a great idea for a school project! I needed to
make a choice: image or sound formats? But the real question was
rather: which kind of format would be the most manageable to explore?
Then the choice was simple: image formats, because while it is true
that sound formats are very interesting, the math behind them was, and
still is, simply beyond me. And while image formats are relatively
simple, you still need to have some moderate knowledge from several
different areas in order to fully understand them, so a project based
on image formats would still be quite challenging. These areas are:

\begin{itemize}
\item Data Compression
\item Color Science
\end{itemize}

pAnd Color Science was by far the hardest of these two areas to study,
which is why only a small segment of this text is dedicated to it,
while pretty must the rest is dedicated to the art of data
compression.

\section{Abstract}

The cover picture of this text was made to illustrate the main point
of this text: digital images are just numbers. The cover image is a
very simplified model of how all image formats in general are built
and I will spend the rest of this section explaining the numbers that
this image consists of. I will in other words write a
\textit{specification} of the image format that the image on the front
page is stored in. This format will from now on be known as IM, the
IMaginary image format.

\subsection{Image Header}

The first gray section of the image is known as the image header of
the image. It will \textit{always} be found first in an image. Let us
go through to parts of the header number by number:

\subsubsection{Magic Numbers}

The first two numbers are the two letters IM. Even letters are
represented by numbers in a computer. How these letters are mapped to
different numbers is determined by the \textit{encoding} of them. An
encoding is basically a table that maps all the letters and
miscellaneous symbols of the human language to numbers. One very
common encoding is known as ASCII, and in ASCII the letter I is
represented by the number 73 and the letter I is mapped to the number
77. So if these letters where encoded in ASCII, a better way to
represent them would have been to simply use the numbers 73 and
77. But I did not do this because it is much easier for the reader to
understand letters than numbers.

These two first numbers are known as the magic numbers of the file
format. At the beginning of every image file, there are a couple
numbers called magic numbers that are common to all images of that
specific format. They are sort like file name extensions(for example,
a file named ``essay.doc'' has a file name extension ``doc''), meaning
that they help you identify the type of the file you are currently
reading.

\subsubsection{Measurements}

And following the magic number is the width and the height of this
image. In practically every image format this information is
included. This information is included to make it easier for programs
dealing with digital images to parse the image data in the file.

\subsubsection{Metadata}

But images can not only contain image data, they can also contain
something known as \textit{metadata}. Metadata contains no data that
is necessary to understand how the image data within the image is
represented, but it rather gives information about things like:

\begin{itemize}
\item The program used to create the image
\item A descriptive name for the image.
\item The name of the creator of the image.
\item The creation date of the image.
\item The camera that was used to take the image.
\end{itemize}

One much used method for embedding metadata about the camera used to
take an image is known as Exif\cite{camera:_cipa_dc_trans_exchan}. But
since Exif is a quite complex method for storing such data, we will
not discuss it in this text. We will only be concerned in exploring
how to extract simple metadata, like the creation date of the image
and the creator of the image.

\subsection{Color Palette}

Color palettes are also very common, but far from all images out in
the wild use them, but as good all image formats support them, hence
why they are included in this imaginary format. A color palette is
basically a list of colors that is to be used by the image. The colors
themselves are represented numerically using a so called color
model. Color models are something we will discus in much depth in
chapter \ref{cha:color}. The numbers in the palette is the numbers
that the colors will be represented by in the color data. In other
words, the color data in this image doesn't contain any actual color
data, but rather contains indexes to a list of colors known as the
color palette. But in real image formats the colors in the palette are
not actually assigned numbers. It is the order in which they appear in
the palette that decide which numbers they are assigned.

\subsection{The Color Data}

Following the header and the color palette is the actual color data.
An image is simply a grid of colors. As you can see on the front cover
image, an image really only consists of a lot of colored
squares. These squares as referred to as
pixels\cite{murray1996encyclopedia}. Every pixel has \textit{one} and
only one color. It is easy to see that even very big images can be
broken down to small colored squares.

Since a color palette was used, the image data simply consists of a
small sequence of indexes to the palette. If the image didn't use a
color palette, the image would just contain a grid of raw colors. The
color would in that case have been represented like they were in the
palette, using a color model.

\subsection{Done\dots?}

But it's not really that simple. A majority of all image formats have
had some sort of compression applied to their color data. Compression
is usually necessary because raw color data usually tends to take up a
\textit{lot} of space. Compression algorithms are hands down the most
difficult part in understanding how image formats work, which is why
the majority of this text will be spent explaining them.
