
\chapter{Introduction and Background}

\section{Introduction}

One day a couple of years ago, I accidentally opened an executable
file(.exe) on my computer in a text editor. In front of my eyes I saw
an absolutely incomprehensible mess of characters and numbers. No
matter how much I examined this mess, I could see no order in this
absolute chaos. This event really got me thinking about one thing:
\textit{how} exactly are all the data stored in a computer actually
\textit{represented}?

A couple of months after this, I learned about the concept of a hex
dump. A hex dump is a hexadecimal representation of the data in a
file. It was at this point I finally realized something that should
have been obvious to me: since the language that the computer talks in
is just a bunch of ones and zeroes, then obviously all the data stored
in a computer is also just a bunch of ones and zeroes! And
furthermore, saying that a computer speaks in ones and zeroes is just
another way of saying that it speaks in \textit{numbers}. Since
mathematics have always been one of my great interests in life, this
episode served greatly to strengthen my interest in how data is
represented in a computer.

Ever since I came upon this realization, something I very much like to
do have been viewing the hexdumps of the files on my computer, and
trying to figure out exactly how they work. %But I quickly realized that
%doing this would require great knowledge of how the data stored in the
%file is

Every kind of way of storing a certain type of data is called a
\textit{file format}. The exact structuring of a file format tends to
be specified in something called that file formats
\textit{specification}. But alas, reading these documents often
requires the reader already being moderately knowledgeable about the
data kind of data that the file format stores.

And so I realized that it would probably require a great amount of
research in order to truly understand how file formats work. But I
still didn't want to give up; my thirst for knowledge was to great for
such an action.  For this reason a spent a great deal of the summer of
2011 in exploring how two kinds of file formats work: image and sound
formats. And while I made progress in exploring how these kinds of
formats work, progress was being made too slow. The Time of the summer
was running out, and I was at the same time desperately trying to come
up with what sort school project I should do for my last term.

And then it suddenly struck me: exploring how one kind of file format
works would be a great idea for a school project! I needed to make a
choice: image or sound formats? but the real questiosn was rather:
which kind of format would be the most manageable to explore? Then the
choice was simple: image formats, because while sound formats are
intesting the math behind them is way beyond at the current I was
at. And while image formats are relatively simple, you need to have
some moderate knowledge from several different areas in order to fully
understand them. These areas are:

\begin{itemize}
\item Information Theory
\item Color Science
\end{itemize}

And Color Science was by far the hardest of these two areas to study,
which is why only a small segment of this text is dedicated to this
area, while pretty must the rest of the book is dedicated to information
theory.

\section{Background}

Designing an image format is not as easy it seems to be; first of all,
you need decide in how flexible you want your format to be, that is,
how many different kind of color storage you want your format to
support. And second, you need to decide on what compression scheme to
use. Compression schemes are for making sure that the image doesn't
take up as much space as it usually would. If images were never
compressed, they would up way too much space and we would be unable
store but a few on our harddrives. The method used does not only have
to be efficient, it has to be fast. The restoration of the compressed
data to its original form has to be fast.???

There are other miscellaneous things image formats are often expected
to support. Some kind of gamma suppot for example??

Another useful thing are checksums. A checksum doesn't necessaly have
to be sum, but it has to be computed from the numerical color data to
be one. Checksums are very useful for veryfying that the image is not
corrupt and that it is in it's original form.
