\begin{comment}
  \bibliography{project.bib}
\end{comment}

\chapter{TGA -- Truevision Graphics Adapter}
\label{cha:tga}

\begin{refsection}

  \section{Introduction}
  \label{sec:tga-introduction}

  The whole purpose of this book was to give a good understanding of
  how image formats really work under the hood. But we couldn't just
  have jumped right in, we required knowledge of how color is stored
  digitally and how compression algorithms work. Now it's time to put
  that knowledge to good use. \textit{All} the compression algorithms
  we just covered are used in the image formats we are going to
  explore. And furthermore, the digital color storage model we have
  learned about is used extensively in image formats.

  Now it's time to explore our real first image format:
  TGA\index{TGA}. But why TGA of all image formats? Because the format
  is an extremely simple one at that, yet it is actually widely used
  in certain areas, like games and video processing. And it has all the
  parts common to all image formats and it even features a very simple
  compression algorithm: PackBits RLE, which was one of the algorithm we
  covered in chapter \ref{cha:compress-techn}.

  But the TGA format is also has its share of complexity. To make the
  format more extensible for programmers such us, Several features
  were added in the second version of the format specification. But we
  will ignore these features as they are not really needed to just
  make an ordinary image, and few images out in the field use them. To
  not make this chapter a complete rip-off of the TGA specification
  \cite{91:_truev_tga_file_format_specif} I will only cover the
  features that are required to load the color data out of the
  image. The rest of the features of the TGA format are free for you
  to explore. This shouldn't be very hard for you to do, as the TGA
  specification is the most well written image format specifications I
  have read.

  The TGA format consists things called fields\index{field}. A field
  gives information about the image you're trying to open. It could
  give information like image name, color depth, the entire color data
  of the image and so soon. It could really contain of anything. And
  field also has a specific size.

  \newcommand{\plural}[3]{\ifstrequal{#1}{1}{#2}{#3}}
  \newcommand{\fieldlength}[1]{\ifstrequal{#1}{0}{variable}{#1 \plural{#1}{byte}{bytes}}}
  \newcommand{\imgfield}[2]{\subsection*{#1(\fieldlength{#2})}}

  The contents of fields will be organized under subsections. The heading:

  \imgfield{Name}{3}

  Specifies that we will be discussing a field named Name with a length
  of 3 bytes.

  On the other hand, the heading:

  \imgfield{Name}{0}

  Shows that we will be covering a field named Name that a length
  that varies depending on the image we're dealing with. The color
  data field is an perfect example of this; it just contains all the
  color of the image and it thus varies with the size and color depth
  of the image. The sizes of these fields could also be specified in
  earlier fields, as we will see example of later.

  The TGA format consists of three major sections: The File Header,
  The Color Data, The Developer Area, The Extension Area, The TGA File
  Footer.

  \section{File Header}

  \imgfield{ID Length}{1}

  Surprisingly enough, this is one the few image formats that doesn't
  have a set of magic numbers at the beginning. Instead, something
  known as the of the image is used. In a later field of
  the format, there is a field called the image ID. This field
  specified the length of that field. It is disscussed
  in the future section.

  \imgfield{Color Map Type}{1}

  The TGA format supports palettetes. An alternative term for palette
  is Color map\index{color map} and that word is used in the TGA
  standard. This field could simply be seen a boolean value indicating
  whether or not the image has a color map or not. So if it is $1$,
  there is indeed a color map in the image, otherwhise; there's no
  color map.

  \imgfield{Image Type}{1}

  There are several ways of storing color as we talked about in
  chapter \ref{cha:color}. In the TGA format there's
  pseudo-color\index{pseudo-color}, which means that the color data
  uses a palette and there's true-color\index{true-color}, meaning that the raw
  color data is stored in the image. And furthermore, the color data
  may either be stored in RGB values, grayscale values and the data
  may even be compressed.

  In the end, this all amounts to $6$ different kinds of images. The
  image type is specified by the number value of this field. Table \ref{tab:imgtype}
  shows all of these different kinds of images. If you have been
  reading carefully up to this point, all the terms of the table
  should make perfect sense to you.

  \begin{table}
    \centering
    \begin{tabular}{ll}
      \toprule
      Image Type & Properties \\
      \midrule
      1 & Uncompressed, Color-mapped \\
      2 & Uncompressed, True-color \\
      3 & Uncompressed, Grayscale\\
      9 & Run-length encoded, Color-mapped \\
      10 & Run-length encoded, True-color \\
      11 & Run-length encoded, Grayscale\\
      \bottomrule
    \end{tabular}
    \caption{Image Type}
    \label{tab:imgtype}
  \end{table}

  \imgfield{Color map specification}{5}

  \section{Color Data}

  \section{Other Fields}

  The extension area includes other miscellaneous information about an
  image. But this information really has nothing to do with the way
  the color data is stored, so we won't explore these fields. But you
  are of course free to explore this area.

  \printbibliography[heading=subbibliography]

\end{refsection}