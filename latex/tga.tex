\begin{comment}
  \bibliography{project.bib}
\end{comment}

\chapter{TGA -- Truevision Graphics Adapter}
\label{cha:tga}

\begin{refsection}

  \cite{91:_truev_tga_file_format_specif}


  \section{Introduction}
  \label{sec:introduction}

  The whole purpose of this book, was to give a good understanding of
  how image formats really work. But we couln't just have jumped right
  in, we required knowledge of how color is stored digitally and how
  some compressin algorithms work. Now it's time to put that knowledge
  to good use. \textit{All} the compressin algorithms we just coverede
  are all used in the image formats we are going to explore. And
  furthermore, the digital color storage model is used extensively in
  image formats.

  Now it's time to explore our first image format: TGA \index{TGA}. But why TGA of
  all image format? Because the format is an extremely simple one at
  that, yet it is actually videly used in some areas, including games
  and video processing. It has all the parts common to all image
  formats and it even features a very simple compression algoriothm:
  RLE, which was one of the algoriothms we covered in the last
  chapter.

  Before we dive let me clarify one thing for you: all darta that is
  stored in a computer is just a sequence of number. How this seuqnece
  of numbers is to be parsed is defined by it's \textit{format}
  \index{format}. Thus, the rest of this book will simply be about how
  you parse the meaning of the numbers that an image file consists
  of. 

  \section{Image headers}
  \label{sec:image-headers}

  Common for all graphics file formats is the image header.

  \printbibliography[heading=subbibliography]

\end{refsection}