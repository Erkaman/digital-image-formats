\begin{comment}
  \bibliography{project.bib}
\end{comment}

\chapter{TGA -- Truevision Graphics Adapter}
\label{cha:tga}

\begin{refsection}

  \section{Introduction}
  \label{sec:tga-introduction}

  The whole purpose of this book was to give a good understanding of
  how image formats really work under the hood. But we couldn't just
  have jumped right in, we required knowledge of how color is stored
  digitally and how compression algorithms work. Now it's time to put
  that knowledge to good use. \textit{All} the compression algorithms
  we just covered are used in the image formats we are going to
  explore. And furthermore, the digital color storage model we have
  learned about is used extensively in image formats.

  Now it's time to explore our real first image format:
  TGA\index{TGA}. But why TGA of all image formats? Because the format
  is an extremely simple one at that, yet it is actually widely used
  in certain areas, like games and video processing. And it has all the
  parts common to all image formats and it even features a very simple
  compression algorithm: PackBits RLE, which was one of the algorithm we
  covered in chapter \ref{cha:compress-techn}.

  But the TGA format is also has its share of complexity. To make the
  format more extensible for programmers such as us, several features
  were added in the second version of the format. But we will ignore
  these features as they are not really needed to just make an
  ordinary image, and few images out in the wild use them. To not
  make this chapter a complete rip-off of the TGA specification
  \cite{91:_truev_tga_file_format_specif} I will only cover the
  features that are required to load the color data out of the
  image. The rest of the features of the TGA format are free for you
  to explore. This shouldn't be very hard for you to do, as the TGA
  specification is an easy read, at least compared to other image
  formats specifications.

  \section{Some Terminology}

  The TGA format consists things called fields\index{field}. A field
  gives information about the image you're trying to open. It could
  give information like the name of the image, color depth, the entire color data
  of the image and so soon. It could really contain of anything. And
  field also has a specific size.

  \newcommand{\plural}[3]{\ifstrequal{#1}{1}{#2}{#3}}
  \newcommand{\fieldlength}[1]{\ifstrequal{#1}{0}{variable}{#1 \plural{#1}{byte}{bytes}}}
  \newcommand{\imgfield}[2]{\subsection*{#1(\fieldlength{#2})}}
  \newcommand{\imgsubfield}[2]{\subsubsection*{#1(\fieldlength{#2})}}

  We will describe these fields in the order that they occur. Each
  field will given it's own subsection, organized under headings:

  \imgfield{Name}{3}

  This heading specifies that we will be discussing a field named Name
  that has a length of 3 bytes. The byte size of course very useful
  for writing the loading code.

  On the other hand, the heading:

  \imgfield{Name}{0}

  Shows that we will be covering a field named Name with a length that
  varies depending on the image we're dealing with. The color data
  field is an perfect example of this; it just contains all the color
  data of the image and it thus varies with the size, width, height
  and color depth of the image. The sizes of these fields could also
  be specified in earlier fields, as we soon will see an example of.

  The fields are also organized into so called sections\index{section} in the
  file. We will now be covering these section one after one:

  \section{File Header}

  \imgfield{ID Length}{1}

  Surprisingly enough, this is one the few image formats that doesn't
  have a set of magic numbers at the beginning. Instead, something
  known as the ID Length of the image is used. In a later field of the
  format, there is a field called the image ID. This field specifies
  the length of that field. It is discussed in depth in section
  \ref{sec:color-data}.

  \imgfield{Color Map Type}{1}

  The TGA format supports color palettes. An alternative term for a
  palette is Color map\index{color map}. This field could simply be
  seen as a boolean value indicating whether or not the image has a color
  map or not. So if it is $1$, there is a color map in the
  image, otherwise; there's no color map is used.

  \imgfield{Image Type}{1}

  There are several ways of storing color as we talked about in
  chapter \ref{cha:color}. In the TGA format there's
  pseudo-color\index{pseudo-color}, which means that the color data
  uses a palette and there's true-color\index{true-color}, meaning that the raw
  color data is stored in the image. And furthermore, the color data
  may either be stored in RGB values, grayscale values and the data
  may even be compressed.

  In the end, this all amounts to $6$ different kinds of images. The
  image type is specified by the number value of this field. Table \ref{tab:imgtype}
  shows all of these different kinds of images. If you have been
  reading carefully up to this point, all the terms of the table
  should make perfect sense to you.

  \begin{table}
    \centering
    \begin{tabular}{ll}
      \toprule
      Image Type & Properties \\
      \midrule
      1 & Uncompressed, Color-mapped \\
      2 & Uncompressed, True-color \\
      3 & Uncompressed, Grayscale\\
      9 & Run-length encoded, Color-mapped \\
      10 & Run-length encoded, True-color \\
      11 & Run-length encoded, Grayscale\\
      \bottomrule
    \end{tabular}
    \caption{Image Type}
    \label{tab:imgtype}
  \end{table}

  \imgfield{Color Map Specification}{5}

  The color map specification fields consists of a bunch of subfields
  describing the color map include in this image. If the image doesn't
  even have color map to begin with, these fields will be totally
  useless to you, but on to the subject:

  \imgsubfield{Color map offset}{2}

  \imgsubfield{Color map length}{2}

  \imgsubfield{Color map entry size}{1}

  \imgsubfield{Color map entry size}{1}

  \imgfield{Image Specification}{10}

  \imgsubfield{X-origin of Image}{2}

  \imgsubfield{Y-origin of Image}{2}

  \imgsubfield{Image Width}{2}

  \imgsubfield{Image Height}{2}

  \imgsubfield{Pixel Depth}{1}

  \imgsubfield{Image descriptor}{1}

  \section{Loading the file header}
  \label{sec:loading-file-header}

  These fields occur in the order that I just described
  them. Algorithm ref{alg:load-file-header} shows how to load them. The values of these fields
  will be used in the loading code for the color data, so do take note
  of their names. As can be seen, I'm almost being condescending
  towards you by even bothering to include this extremely simple
  code. I promise I won't be as condescending in the future.

  \begin{algorithm}
    \caption{Loading the file header of a TGA image.}
    \label{alg:load-file-header}
    \newcommand{\loadfield}[2]{\Let{$#1$}{\Call{Read}{$#2$}}}
  \begin{algorithmic}[1]
    \loadfield{IDLength}{1}
    \loadfield{colorMapType}{1}
    \loadfield{imageType}{1}
    \linecomment{The Color Map Specification}
    \loadfield{colorMapOffset}{2}
    \loadfield{colorMapLength}{2}
    \loadfield{colorMapEntrySize}{2}
    \linecomment{Image Specification}
    \loadfield{xOrigin}{2}
    \loadfield{yOrigin}{2}
    \loadfield{imageWidth}{2}
    \loadfield{imageHeight}{2}
    \loadfield{pixelDepth}{1}
    \loadfield{imageDescriptor}{1}
  \end{algorithmic}
\end{algorithm}

\section{Color Data}
\label{sec:color-data}

\imgfield{Image ID}{0}

\imgfield{Color Map Data}{0}

\imgfield{Color Map Specification}{0}

\section{Developer Area}

The extension area is useful to programmers.

\section{Extension Area}

The extension area includes many different fields containing mostly
miscellaneous information about an image. We will not be covering
here, but the sample source code includes does load this area and
all of it's fields. See the TGA specification for more information
on this area.

\section{TGA File Footer}

The file footer of a TGA image always occurs in the end of it, if it
has one. Images made before the version 2 of the TGA format tend not
to have one, and neither do they have an extension area nor
developer area. Speaking of that, all this section really contains
is the offset the developer area and extension area. \todo{mention
  the fseek function}. Because you see, these fields can occur
anywhere in a TGA image.\todo{mention order implications}.

what.

\printbibliography[heading=subbibliography]

\end{refsection}