\begin{comment}
  \bibliography{project.bib}
\end{comment}

\chapter{TGA -- Truevision Graphics Adapter}
\label{cha:tga}

\begin{refsection}

  \section{Introduction}
  \label{sec:tga-introduction}

  \cite{91:_truev_tga_file_format_specif}

  The whole purpose of this book was to give a good understanding of
  how image formats really work under the hood. But we couldn't just
  have jumped right in, we required knowledge of how color is stored
  digitally and how compression algorithms work. Now it's time to put
  that knowledge to good use. \textit{All} the compression algorithms
  we just covered are used in the image formats we are going to
  explore. And furthermore, the digital color storage model we have
  learned about is used extensively in image formats.

  Now it's time to explore our real first image format: TGA
  \index{TGA}. But why TGA of all image formats? Because the format is
  an extremely simple one at that, yet it is actually videly used in
  some areas, including games and video processing. It has all the
  parts common to all image formats and it even features a very simple
  compression algoriothm: RLE, which was one of the algoriothms we
  covered in the last chapter.

  \printbibliography[heading=subbibliography]

\end{refsection}
