\begin{comment}
  \bibliography{project.bib}
\end{comment}

% http://www.svgopen.org/2005/papers/abstractsvgopen/

\chapter{Truevision Graphics Adapter}
\label{cha:tga}

\section{Introduction}
\label{sec:tga-introduction}

In this chapter, we are to explore the first image format in this
text: TGA\index{TGA}. But why are we beginning with TGA of all image
formats?

For the same reason that the image format first became popular: its
specification is very well written; it is freely available without any
encumbering patents; and it is one extremely simple image format
because the color data is stored in an obvious and intuitive way(in
chapter \ref{cha:gif} and \ref{cha:png} we will see \textit{many}
examples of how unintuitively the color data can be stored) and the
data itself is compressed with a very simple compression algorithm:
PackBits RLE, which we discussed very much in depth in section
\ref{sec:packbits-rle}. \cite{murray1996encyclopedia}

However, there are also some more complex features of the TGA format,
but we will not cover all of them because few of them are rarely ever
used in the real world. Only the features that are actually used in a
majority of TGA images are going to be covered in depth.

The TGA specification is the document
\cite{91:_truev_tga_file_format_specif} and it is going to be the main
reference for this chapter.

\section{Conventions Used In This Chapter}

The TGA format is built up of things called fields\index{field}. A
field gives information about the image you're trying to open. It
could give information like the name of the image, color depth, the
color data of the image and so on. It could really contain any sort of
data.

A field also has a specific size. Knowing the size of a field is
essential to loading its data, since you cannot just let the computer
guess the size of a field. It's all just a sequence of numbers to the
computer, so you as the programmer will have to tell the computer how
big the respective fields are in order to load them.

\newcommand{\plural}[3]{\ifstrequal{#1}{1}{#2}{#3}}
\newcommand{\fieldlength}[1]{\ifstrequal{#1}{0}{variable}{#1 \plural{#1}{byte}{bytes}}}
\newcommand{\imgfield}[2]{\subsection*{#1(\fieldlength{#2})}}
\newcommand{\imgsubfield}[2]{\subsubsection*{#1(\fieldlength{#2})}}

Fields are organized into larger units we will call
\textit{sections}\index{section}. We will in this chapter cover these
sections one after one and we will describe the fields in the order
that they occur in their respective sections. The separate are
organized in subheadings like this:

\imgfield{Data}{3}

This subheading specifies that we will be discussing a field named
Data that has a size of 3 bytes.

On the other hand, the subheading:

\imgfield{Data}{0}

Shows that we will be covering a field named Data with a variable
size. The size of such a field varies completely on the image we're
dealing with. The color data field is a perfect example of that kind
of field; since it contains the color data of the image, it will
always vary for images of different widths and height, and therefore
the size of this field is variable.

The sizes of these fields are most often specified by \textit{earlier}
fields, as we soon will see several examples of.

\section{File Header}

The first thing that can be found in a TGA image is always the File
Header.

\imgfield{ID Length\index{ID Length}}{1}

Surprisingly enough, this is one the few image formats that doesn't
have a set of magic numbers identifying the type of the file at the
beginning. Instead, something known as the ID Length is found. In a
later field of the format, there is a field called the image ID and
this field specifies the length of that field in bytes.

\imgfield{Color Map Type\index{Color Map Type}}{1}

The TGA format supports color palettes. An alternative term for
palette used in the TGA specification is Color map\index{color
  map}. This field could simply be seen as a boolean flag indicating
whether or not the image has a color map; if the value of this field
is $1$, then the image has a color map and if it is $0$, then there is
no color map in the image.

\imgfield{Image Type\index{Image Type}}{1}

The TGA format supports three different ways of storing color:
grayscale\index{grayscale}, color-mapped color and
true-color\index{true-color}.

Grayscale color should be familiar to you by now. Color-mapped color
obviously means that the color data are indexes to a
color-map. True-color simply means that the color uses the RGB or the
RGBA color models.

%The color depths supported by the TGA format are 16,24 and 32.

Since the color data can also be PackBits RLE compressed, this all amounts to $6$
different kinds of image types. The image type is specified by the
number value of this field. Table \ref{tab:imgtype} shows the number
assigned to each image type\footnote{I will admit that to the number
  $0$ the image type ``No Image Data Included'' is assigned. However,
  since the main topic of this text is the storage of digital color, I
have deemed this image type useless for our purposes and therefore
omitted any discussion of it.}.

\begin{table}
  \centering
  \begin{tabular}{ll}
    \toprule
    Image Type & Description \\
    \midrule
    1 & Uncompressed, Color-mapped \\
    2 & Uncompressed, True-color \\
    3 & Uncompressed, Grayscale\\
    9 & Compressed, Color-mapped \\
    10 & Compressed, True-color \\
    11 & Compressed, Grayscale\\
    \bottomrule
  \end{tabular}
  \caption{TGA Image Type values}
  \label{tab:imgtype}
\end{table}

From table \ref{tab:imgtype}, we can derive that, if we let $T$
represent the value of this field, then a TGA image is compressed if
and only if the following holds true:

\begin{equation*}
  imageType = 9 \OR imageType = 10 \OR imageType = 11
\end{equation*}

\imgfield{Color Map Specification\index{Color Map Specification}}{5}

The color map specification fields consists of a set of subfields
describing the color map included in the image. If the image doesn't
even have a color map to begin with, these fields can simply be
ignored. Remember that a TGA image only has a color map if and only if
the color map type $1$.

\imgsubfield{First Entry Index\index{First Entry Index}}{2}

This entry consists of 2 bytes, so it's a 16-bit number and the
maximum length of a 16-bit number is $2^{16} - 1$

A color map essentially consists of a sequence of colors. The value
of this field is how many of those colors that should be skipped in
the beginning of that sequence. Don't know why you'd \textit{ever}
want to do this, though. This value is most usually $0$.

\imgsubfield{Color map length\index{Color map length}}{2}

In this field, the length of the color map is specified. The
length is the number of colors in the color map.

\imgsubfield{Color map entry size\index{Color map entry size}}{1}

Because the color map is just an array of colors, of course all
these colors must have a specific color depth. That color depth is
specified by this field. This number is specified in bits.

If the color map entry size is $S$ and the color map length is $L$,
then the total byte length of the color map is $\frac{S}{8} \cdot
L$. If furthermore the first entry index is $I$ and $I \neq 0$, then
the number is colors you actually end up loading from the color map
will actually be $\frac{S}{8} \cdot L - I$, since you in this case
skipped $I$ colors at the beginning of the color map. In algorithm
\ref{alg:tga-read-colormap} it is shown to load the color map to an
array.

\begin{algorithm}[H]
  \caption{Reading the color map of a TGA file.}
  \label{alg:tga-read-colormap}
  \begin{algorithmic}[1]

    \State Move $colorMapStart$ bytes forward in the file.

    \ForTo{$i$}{$colorMapStart$}{$colorMapLength$}
      \Let{$data$}{\Call{Read}{$colorMapEntrySize / 8$}}
      \Let{$colorMap[i]$}{$data$}
    \EndFor
    \end{algorithmic}
  \end{algorithm}

\imgfield{Image Specification\index{Image Specification}}{10}

Following the color map specification is the image specification. This
field contains many values that are essential to loading the image data
properly.

\imgsubfield{X-origin of Image}{2}

\imgsubfield{Y-origin of Image}{2}

These two subfields specify the coordinates of the image on the
screen. The values of both of these fields are usually $0$, and I have
never used an image viewer that actually supports the usage of these
fields. My advice is that you can simply ignore these two values.

\imgsubfield{Image Width\index{Image Width}}{2}

\imgsubfield{Image Height\index{Image Height}}{2}

These two fields specify the size of the image in pixels. Let $W$ be
the first of these two fields, and $H$ the second, then the total
number of pixels(colors) in this image is $W \cdot H$. And if the
color depth is $D$, and $D$ is measured in bits, then the total byte
size of the image data is $\frac{W \cdot H \cdot D}{8}$. Note that
this only holds true when the color data have been uncompressed.

\imgsubfield{Pixel Depth\index{Pixel Depth}}{1}

If the image doesn't use a colormap then this is color depth of the
colors in the image data. If the image uses a color palette then this
is the bit size of the indexes that the color data consists of when a
colormap is used.

\imgsubfield{Image Descriptor\index{Image Descriptor}}{1}

The image descriptor stores the data of several subfields at different
bits in a single byte.

Bits 7 and 6, the two highest bits, are unused and their values should
always be set to zero.

Bits 5 and 4 specify the screen destination of the first pixel. The
screen destination varies depending on the values of these two bits
according to table \ref{tab:image-origin}. If the image destination is
the top left corner, then the image is normally displayed. If it is
the top right corner, then the image is mirrored. If it is the bottom
left corner, then the image is flipped upside down. If it is the
bottom right corner, then the image is mirrored \textit{and} displayed
upside down.

% discuss how you can achieve horizontal reversal using this field.

\begin{table}
  \centering
  \begin{tabular}{lrr}
    \toprule
    Screen destination \\ of first pixel & bit 5 & bit 4 \\
    \midrule
    bottom left & 0 & 0 \\
    bottom right & 0 & 1 \\
    top left & 1 & 0 \\
    top right & 1 & 1 \\
    \bottomrule
  \end{tabular}
  \caption{The different image origin combinations}
  \label{tab:image-origin}
\end{table}

The last four bits are the alpha channel bits. These specify how many
bits in a separate color is used for the alpha channel. For images
without an alpha channel this value will always be zero. For 16-bit
images it will be 1, and for 32-bit images it will be 8. 32-bit color
uses the ARGB color channel ordering, meaning that the 8 highest bits,
bits 24-31, are used to store the alpha. How the alpha channel is
stored in 16-bit color was discussed in section
\ref{sec:other-channel-sizes}.

And the data of this image descriptor can trivially be parsed using
the \textproc{getbits} function.

\section{Color Data\index{Color Data}}
\label{sec:color-data}

I described the fields of the image header in the order that they must
be loaded. Following the image header is the color data section. Here
the palette, if there is one, and the color data of the image can be
found.

\imgfield{Image ID\index{Image ID}}{0}

This field contains identifying information about the image in form of
an ASCII string. It could be described as the name of the image. The
ID Length field found in the header describes the size of this
field. Since the ID Length field was stored in a byte, the maximum
length of this string is $255$. But if the ID Length field is 0, this field
doesn't exist at all and should be ignored.

\imgfield{Color Map Data\index{Color Map Data}}{0}

The color map can trivially be loaded into an array by loading the
same number colors that was specified in the color map length field. These
colors will have to be loaded $\frac{D}{8}$ bytes at a time, where $D$
is the color map entry size. Back in algorithm
\ref{alg:tga-read-colormap} it was shown to load the color map into an
array.

\imgfield{Color Data\index{Color Data}}{0}

In this field, the color data of the image is stored. The way this
data is stored depends on the values stored in the image header. In
algorithm \ref{alg:TGA-read-colordata} it is shown how to load the
color data. Note that this algorithm only loads the numbers which
represents the colors. If you don't know parse the color channels in
these loaded numbers, then I refer you back to chapter
\ref{cha:color}.

And if you don't understand the decompression step of the PackBits RLE
encoded data in the algorithm, then refer you back to algorithm
\ref{alg:packbits-dec}.

This algorithm is a combination of everything you have learned up to
this point and you should study it carefully and not go on reading until
you have fully understood it.

\begin{algorithm}[H]
  \caption{Reading and decompressing the color data of a TGA file.}
  \label{alg:TGA-read-colordata}
  \newcommand{\process}{\State Process the color $data$ \dots}
  \begin{algorithmic}[1]
    \Let{$imageSize$}{$imageWidth \mul imageHeigth$}
    \Let{$compressed$}{$imageType = 9 \OR imageType = 10 \OR imageType = 11$}
    \Let{$i$}{$0$}

    \While{$i < imageSize$}
      \If{$compressed$}

      \Let{$packet$}{\VoidCall{ReadByte}}
      \Let{$length$}{$packet \BitAnd 127$}

      \If{$packet \BitAnd 128$}

        \Let{$data$}{\Call{Read}{$pixelDepth / 8$}}

        \If{$colorMapType = 1$}
          \Let{$data$}{$colorMap[data]$}
        \EndIf

        \Repeatn{$length + 1$}
          \process
          \Inc{$i$}{1}
        \EndRepeatn

    \Else

      \Repeatn{$length + 1$}

        \Let{$data$}{\Call{Read}{$pixelDepth / 8$}}

        \If{$colorMapType = 1$}
          \Let{$data$}{$colorMap[data]$}
        \EndIf

        \process

        \Inc{$i$}{1}

        \EndRepeatn

       \EndIf
    \Else

      \Let{$data$}{\Call{Read}{pixelDepth / 8}}

      \If{$colorMapType = 1$}
        \Let{$data$}{colorMap[data]}
      \EndIf

      \process

      \Inc{$i$}{1}

    \EndIf
    \EndWhile
  \end{algorithmic}
\end{algorithm}

\section{File Footer\index{File Footer}}

TGA images might also contain a file footer. The file footer is mainly
used for getting other kinds of data than color data, mostly metadata,
about an image. But this data only exists in the image if the file has
a File Footer and not all TGA images even have one.

If there is a file footer, then the last 26 bytes of the file
constitutes of it. And if the 26 last bytes really is a file footer,
then bytes 8-23 for this footer is the ASCII string
``TRUEVISION-XFILE''. If this is true then these 26 bytes is the file
footer of the image; otherwise, the image doesn't have a file footer.

Byte 24 of this footer is the ASCII character ``.''(a dot) and byte 25
is the NULL terminator(the ASCII character whose ASCII value is
0). Only bytes 0-7 in the file footer contain information that is of
any significance and the fields that these bytes form are what I am
going to cover now:

\imgfield{Extension Area Offset}{4}

This 32-bit number gives the offset to the extension area. That is, it
gives the position of the extension area, assuming that the very first
byte in the file has position 0. I will discuss the contents of the
extension area in section \ref{sec:extension-area}.

If this value is zero, then there is no extension area in the
file. This will also hold true for any of the following offsets I am
about to discuss. In general, if the offset to a section in a file has
the value $0$, then that section does not exist for that file.

\imgfield{Developer Directory Offset}{4}

This field gives the offset of the less interesting developer area. I
will briefly discuss this area section \ref{sec:developer-area}.

\section{Developer Area\index{Developer Area}}
\label{sec:developer-area}

The developer area allows programmers to add information to the TGA
format that is specific to their applications. But this information is
basically useless if you're only trying to load the color data from
the TGA image, hence why we will not cover this section at all.

Even if you were to find a TGA image out in the wild that uses this
feature, you can only really guess on how you are supposed to use this
data, as this is entirely dependent on the program that created the
image.

But if you actually need to figure out how to use this area, then I
refer you to
\cite{murray1996encyclopedia,91:_truev_tga_file_format_specif} for
further reading.

\section{Extension Area\index{Extension Area}}
\label{sec:extension-area}

The extension area is an absolutely \textit{huge} section containing
many fields giving numerous kinds of meta data about the image. After
examining this section, you should hopefully finally understand what
metadata is.

\imgfield{Extension Size}{2}

This field gives the size of the entire extension area, which is
always $495$ bytes.

\imgfield{Author Name}{41}

This field gives the name of the author of the image. The last byte of
this string is always NULL, so you can load this field as a C
string. And if the length of the name is less than $40$, the rest of
the string will be filled with NULL characters to indicate the end of
the author name. We will from now on refer to such a string as a
NULL-padded C string.

\imgfield{Author Comment}{324}

In this field the author can leave a comment on the image. This could
for example be a shorter description of the where the image was
taken. It is stored in a NULL-padded C string.

\imgfield{Date/Time Stamp}{12}

This field contains a series of 6 16-bit numbers that are used to
store the exact time at which the image was created and saved. The
numbers are stored in this order:

\begin{enumerate}
\item Month(1-12)
\item Day (1-31)
\item Year (four digits)
\item Hour (0-23)
\item Minute (0-59)
\item Second (0-59)
\end{enumerate}

\imgfield{Job Name/ID}{41}

This is used to tie an image with a specific job. That is the best
explanation I can give you, because I didn't understand the TGA
specification's description of this field.

\imgfield{Job Time}{6}

This is the time invested in creating the image, as described by the 3
16-bit values:

\begin{itemize}
\item Hours (0 -- 65535)
\item Minutes (0--59)
\item Seconds (0--59)
\end{itemize}

\imgfield{Software ID}{41}

A NULL-padded C string storing the name of program that was used to
create the image. This string could for example be GIMP or
Photoshop(Two image editing programs).

\imgfield{Software Version}{3}

The field gives the version of the software that was used to create
the image.

\imgsubfield{Version Number}{2}

If $V$ is the version of the software that was used to create the
image, then this field is $100V$. So if the software that was
used to create this image for example had the version $1.23$, then the
number $123$ would get stored in this field.

\imgsubfield{Version Letter}{1}

In this field, an ASCII letter that is used to indicated the version
letter of the software is stored. If the software that created this
image was in alpha version when the image was created, then the ASCII
letter ``a'' would get stored in this field.

\imgfield{Key Color}{4}

The key color can be though of as a background color. It is the color
that is displayed under the rendered image, so to say, when it is
shown by an image viewer. This color will be visible only if the
transparency feature is used in the image. Under transparent pixels
this color will be visible. The key color is stored as a basic 32-bit
color in the ARGB order.

\imgfield{Pixel Aspect Ratio}{4}

This field consists of the two subfields

\imgsubfield{Pixel Width}{2}
\imgsubfield{Pixel Height}{2}

If the pixels of the original image were not squares, like pixels
should be, then these values are used by image viewer to preserve the
appearance of the pixels as they were in the original image.

\imgfield{Gamma Value}{4}

\todo{gamma}

\imgfield{Color Correction Offset}{4}

This gives the offset to the color correction table. However, this is
a feature that practically no TGA images actually use, and we will
therefore not discuss this feature in depth.

\imgfield{Postage Stamp Offset}{4}

This field gives the offset to the postage stamp image. The postage
stamp image is like a small preview of the image stored in the
file. The first value in the postage stamp is the width of the postage
stamp image and the second value is its height. Following these two
values is the postage stamp image. The color data in the postage stamp
image is stored in exactly the same way as the full-size image and
should therefore be loaded in the same way, as it is specified in
algorithm \ref{alg:TGA-read-colordata}.

The postage stamp image is mainly useful when you want to preview a
TGA image, without loading the entire full-size image. This may come
in use when the full-size image is very big and you just want a small
preview of it.

\imgfield{Scan Line Offset}{4}

This field provides the offset to the beginning of the Scan Line Table
of the image. But the Scan Line Table is practically never used in
practice, so I'll only discuss the Scan Line Table very briefly.

The Scan Line Offset table basically contains a list of offsets. These
offsets are offsets to the beginning of every single line, also called
scan line, in the image data.

\imgfield{Attributes Type}{1}

This field specifies which type of alpha channel data is stored in the
file. The following are valid values for this field:

\begin{description}
  \item[0] No alpha data is included at all.
  \item[1] The alpha data is undefined, meaning that it can be ignored.
  \item[2] Basically same as for $1$.
  \item[3] The alpha data should be used for transparency.
  \item[4] The color uses pre-multiplied alpha.

    But what does that mean? Let the quadruplet $(A,R,G,B)$ represent
    a 32-bit ARGB color. How would one specify a half transparent
    fully red color? One obvious suggestion is
    $\left(\frac{255}{2},255,0,0\right)$. However, using
    pre-multiplied alpha, the alpha channel gets multiplied into the
    other color channels, so using pre-multiplied alpha the former
    quadruplet would instead be represented as

    \begin{equation*}
    \left(
    \frac{255}{2},
    \left(255 \cdot \frac{255}{2}\right) \bmod 255,
    0 \cdot \frac{255}{2},
    0 \cdot \frac{255}{2}\right)  =  \left(\frac{255}{2},\frac{255}{2}, 0, 0\right)
    \end{equation*}

    So using pre-multiplied alpha, the quadruplet $(A,R,G,B)$ is
    instead represented by the quadruplet $\left(A, AR \bmod M, AG
      \bmod M, AB \bmod M \right)$, where $M=2^n$ and $n$ is the
    number bits assigned to one channel. The advantages of
    pre-multiplied alpha will not be treated in this text, but a brief
    discussion of this can be found in \cite{porter84_compos_dig_img}.

    % pre-multiplied alpha exercise?

\end{description}


