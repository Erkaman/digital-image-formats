\begin{comment}
  \bibliography{project.bib}
\end{comment}

\chapter{Purpose and problem}

So the main purpose of this project is to explore how image formats
works. But only the area of image formats itself is a very broad area
since there are so \textit{many} of them. Finding out exactly how many
image formats there are in existence is a very difficult, almost
impossible, task. Imagemagick, an open source project whose main
purpose could be said to be to support the maximum amount of image
formats, supports well over 100 formats
\cite{11:imagemagick_home}. And according to the \textit{open
  directory project} there are at this time about 170 different
graphics formats \cite{10:opendirectory_data_formats}.

Indeed, there are many different kinds of image formats. And
furthermore, image formats are generally divided into two main
categories: \textit{vector and raster
  graphics}\cite{murray1996encyclopedia,roelofs99:_png}. Vector
graphics consists of lines, rectangles, B\'{e}zier curves and other geometric
forms, while raster graphics could simply be described as a grid,
matrix, of colors. We will in this project focus only on raster
graphics. Some authors, like \cite{murray1996encyclopedia}, also refer
raster graphics to as \textit{bitmap images}.

But there are also many different raster formats. Yet all of them are
very similar in their overall structure. So to make the project even
more manageable, we will only cover the three specific raster formats
that are known as \tga, \gif and \png. The simplest formats will be
covered first, and the most the complex format will be covered last,
meaning that we will explore first \tga, then \gif and last \png, which is
significantly more complex than the former two.

By exploring these 3 formats, we will have gotten a good general view
of how image formats are built and structured. So this is main purpose
of the project:

\begin{quote}
  To explore how images are digitally, numerically, represented and
  structured.
\end{quote}

And by this we mean that this project will go on until we have fully
understood the specifications of the 3 former image formats.
