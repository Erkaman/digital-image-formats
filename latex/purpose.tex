\begin{comment}
  \bibliography{project.bib}
\end{comment}

\chapter{Purpose and problem}

\section{Purpose}

So the main purpose of this project is to explore how image formats
works, and how color is stored digitally. But only image formats
itself is a very broad area. Finding out exactly how many image
formats there are in existence is a very difficult, almost impossible,
task. Imagemagick, a software project whose main purpose could be said
to be to support the maximum amount if image formats, supports well
over 100 formats \cite{11:imagemagick_home}. And according to the open
directory project there are at this time about 170 different graphics
formats \cite{10:opendirectory_data_formats}.

So there are many different kinds of image formats. Furthermore, image
formats are divided into two main categories: vector and raster
graphics. Vector graphics consists of lines, rectangles and other
geometric forms, while raster graphics could be described as a grid of
colors. Neither of these is really simpler than the other, but in this
project I will focus only on raster graphics. Some author like,
\cite{murray1996encyclopedia}, also refer raster graphics to as bitmap
images \cite{murray1996encyclopedia,roelofs99:_png}.

There are many different raster formats. But they are all of them very
similar to each and tend to have the same general structure. So to
make the project more manageable, I will only focus on three specific
raster formats known as TGA, GIF and PNG. I will cover these formats
in the order of their complexity, that is, first I will cover TGA,
then GIF and last PNG, which is significantly more complex than the
former two.

By exploring these 3 formats, I will have gotten a quite good view of
how image formats are built and structured. So this is main purpose of
this project:

\begin{quote}
  To explore how images are \textit{numerically} represented.
\end{quote}