\begin{comment}
  \bibliography{project.bib}
\end{comment}

\chapter{Perfect Compression?}
\label{cha:perfect}

In this chapter, we will show that a perfect lossless compression algorithm does
not exist, where a perfect lossless algorithm is able to compress
\textit{any} input without fail.

\section{The Counting Argument}

We base the following discussion on\cite{Salomon:2004:DCC,jean-loup-comp-faq}.

Let us first discuss something known as the counting argument. It goes
as follows:

\begin{quote}
  No \textit{lossless} compressor can compressed all files of size
  $\ge N$ bits, for all integers $N \ge 0$.
\end{quote}

This statement can be proved using surprisingly simple
mathematics. Let us first of all assume that such a compressor do
indeed exists and so if find any contradictions. What would such a
compressor have to do? It would have to be able to compress down all
$2^n$ files of length n bits down to files that are \textit{at most}
$n-1$ bits long. How many possible files are at most $n-1$ bits long?
This is the sum

\begin{equation}
  \label{eq:n-minus-one}
  2^0 + 2^1 + \ldots + 2^{n-1}
\end{equation}

If we inspect this sum, then we will see that the quotitent between
each term is in fact $2$. So the sum could also be expressed as

\begin{equation*}
  \sum^{n-1}_{i = 0} 2^{i}
\end{equation*}

From this we realize that the sum is a simple geometric series! As
familiar, such sums are calculated as

\begin{equation}
  \label{eq:geometric-series}
  a + ak + ak^2 + \ldots + ak^{n-1} = \sum^{n-1}_{i = 0} ak^{i} =
  \frac{a(k^n - 1)}{k -1}
\end{equation}

So the sum \eqref{eq:n-minus-one} can from \eqref{eq:geometric-series}
simply be computed to

\begin{equation*}
  \frac{1 \cdot (2^{n} - 1)}{2 -1} = 2^{n} - 1
\end{equation*}

So, $2^n$ different $n$ bits files are by the perfect compression
algorithm supposed to be compressed down $2^n - 1$ different files. By
the pigeon hole principle, it is impossible for this compression to be
lossless, because since $2^n - 1 < 2^n$ at least two different files
will be compressed down to the same file. This simple contraction
concludes the counting argument, reaching the conclusion that perfect
lossless compression is impossible!

But on the other hand, since $2^n - (2^{} - 1) = 1$, then that means
that only one file failed to be mapped losslessly to some compressed
bit string. The algorithm can actually be made lossless by mapping
this remaining bit strings to some bit string whose length is $\ge
n$. But since this compression algorithm no longer compresses all input strings it
is no longer perfect, but it is now at least lossless!