% fonts
\usepackage[T1]{fontenc}
\usepackage{inconsolata}
\usepackage[charter]{mathdesign}
\usepackage[kerning,spacing,stretch=10]{microtype}
\microtypecontext{spacing=nonfrench}

% index creation
\usepackage{makeidx}

\makeindex

% miscellaneous packages
\usepackage{xcolor}
\usepackage{float}
\usepackage[hyperindex,colorlinks]{hyperref}
\usepackage{verbatim}
\usepackage{siunitx}
\usepackage{todonotes}
\usepackage{placeins}
\usepackage{booktabs}
\usepackage{graphicx}
\usepackage{etoolbox}
\usepackage{path}
\usepackage{enumitem}
\usepackage{xspace}
\usepackage{amsmath}

\usepackage{caption}
\captionsetup{
  margin=10pt,
  font=small,
  labelfont=bf,
  labelsep=endash,
  format=hang
}

% references
\usepackage[backend=biber,sorting=none]{biblatex}
\addbibresource{project.bib}

% tikz
\usepackage{tikz}
\usetikzlibrary{
  shapes,backgrounds,decorations.pathreplacing,
  positioning,fit,shadings,fadings,matrix,calc}

\newenvironment{indentpar}[1][1cm]%
{\begin{list}{}%
    {\setlength{\leftmargin}{#1}}%
  \item[]%
  }
  {\end{list}}

% algorithms typesetting.
\usepackage[chapter]{algorithm}
\usepackage{algpseudocode}

\newcommand{\commentsymbol}{$\triangleright$}
\algrenewcommand{\algorithmiccomment}[1]{\hfill\commentsymbol#1}
\newcommand{\linecomment}[1]{\State\commentsymbol#1}

\algrenewcommand\alglinenumber[1]{
  \tt\footnotesize\textcolor[HTML]{888888}{#1}}

\newcommand*\Let[2]{\State #1 $\gets$ #2}

\algblockdefx[REPEATN]
{Repeatn}
{EndRepeatn}
[1]{\algorithmicrepeat\ #1 \ \algorithmicdo}
{\algorithmicend \ \algorithmicrepeat}

\algnewcommand\algorithmicforeach{\textbf{for each}}

\newcommand{\algto}{\textbf{to}}



\algblockdefx[FOREACH]
{ForEach}
{EndForEach}
[2]{\algorithmicforeach \ #1 \ \textbf{in} \ #2 \algorithmicdo}
{\algorithmicend \ \algorithmicforeach}

\newcommand{\Break}{\State \textbf{break}}
\newcommand{\VoidCall}[1]{ \Call{#1}{\ensuremath{}}}
\newcommand{\ForTo}[3]{\For{ #1 $\gets$ #2 \algto\  #3}}

\newcommand{\Inc}[2]{\Let{#1}{#1 + #2}}

% miscellaneous configurations
%\pagestyle{headings}

\usepackage{fancyhdr}
\pagestyle{fancy}

\fancypagestyle{plain}{%
\fancyhf{} % clear all header and footer fields
\renewcommand{\headrulewidth}{0pt}
\renewcommand{\footrulewidth}{0pt}}

% macros

\newcommand{\inputtikz}[1]{\input{tikz_img/#1}}

\newcommand{\True}{\textbf{true}}
\newcommand{\False}{\textbf{false}}
\newcommand{\NOT}{\neg}
\newcommand{\AND}{\wedge}
\newcommand{\OR}{\vee}
\newcommand{\mul}{\ensuremath{\cdot}}

\newcommand*\BitAnd{\mathrel{\&}}
\newcommand*\BitOr{\mathrel{|}}
\newcommand*\BitXor{\mathrel{\otimes}}
\newcommand*\ShiftLeft{\ll}
\newcommand*\ShiftRight{\gg}
\newcommand*\BitNeg{\ensuremath{\mathord{\sim}}}

% Fixes list of algorithms formatting.
% Thanks Dan: http://newsgroups.derkeiler.com/Archive/Comp/comp.text.tex/2005-10/msg01179.html
\makeatletter
\let\l@algorithm\l@figure
\makeatother

\renewcommand{\listofalgorithms}{\begingroup
\tocfile{List of Algorithms}{loa}
\endgroup}

\usepackage[nottoc]{tocbibind}

\newcommand*{\tikzname}{Ti\emph{k}Z\xspace}

\newcommand{\fullbyte}{$11111111$\xspace}
\newcommand{\emptybyte}{$00000000$\xspace}

% Thanks to Philip and Konrad Rudolph:
% http://tex.stackexchange.com/questions/2933/parse-comma-separated-list-of-command-names-with-spaces

\newcommand*\acronymstyle{\scshape}
\newcommand*\newacronym[1]{
  \expandafter\newcommand\csname#1\endcsname[1][]{{\acronymstyle#1}##1\xspace}}

\newcommand*{\newacronyms}{%
  \let\do\newacronym
  \docsvlist
}
\newacronyms{tga,png,gif,C,pgf,lzw,ascii,lz,lzma,lzss}

\newcommand{\eof}{\ensuremath{\VoidCall{EndOfFileReached}}}
\newcommand{\neof}{\ensuremath{\NOT \VoidCall{EndOfFileReached}}}

\newcommand{\lzseven}{{\acronymstyle lz77}\xspace}
\newcommand{\lzeight}{{\acronymstyle lz78}\xspace}
