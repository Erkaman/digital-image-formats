\begin{comment}
  \bibliography{project.bib}
\end{comment}

\chapter{Method}
\label{cha:method}

To explore how image formats works, I decided to write small programs
that would analyse the numbers that digital images consists. My
programmign language of choice for this was C and the compiler I used
to compile the programs was GCC. For doing the memory debugging that
is often necesary in C programming, I used
Valgrind\footnote{\url{http://valgrind.org/}}, which at times was a real
life saver. For tests and atomization tasks Bash and Python were
used. My editor of choice was \textsc{Emacs}.

This document was created and typeset using
\LaTeXe{}\footnote{\url{http://www.latex-project.org/}}. \textsc{Biber} was
used for reference management. \textit{MakeIndex} was used for index
management. The typesetting software used was pdf\LaTeX{}.

The text is typeset in the fonts Charter(main Body text) and
\texttt{inconsolata}(monospace). The \LaTeX{} source files were edited
in \textsc{Emacs} with the AUC\TeX{} package.

The following \LaTeX{} packages were used to produce this document:

\texttt{inconsolata}, fontenc, mathdesign, charter, microtype,
makeidx, xcolor, float, verbatim, siunitx, todonotes, placeins,
booktabs, graphicx, caption, biblatex, \tikzname, \pgf , algorithms,
algorithmicx, fancyhdr, tocbibind.

All the source files in the project were managed with \textsc{Git},
the version control system.

