\begin{comment}
  \bibliography{project.bib}
\end{comment}

\chapter{Methods}
\label{cha:method}

\newcommand{\CC}{C\nolinebreak\hspace{-.05em}\raisebox{.4ex}{\tiny\bf +}\nolinebreak\hspace{-.10em}\raisebox{.4ex}{\tiny\bf +}}

\section{Tools}

\newcommand{\credits}[1]{\footnote{\url{#1}}}

To explore how the aforementioned image formats works, I decided to
write small programs that would analyze the numbers that images in the
formats consists of and dump the raw color data of the images in
files. My programming language of choice for this was \CC{} and my
compiler was \gcc\credits{http://gcc.gnu.org/}. For doing the memory
debugging that is often necessary in \CC{} programming I used
Valgrind\credits{http://valgrind.org/}, which at times was a real life
saver. My editor of choice for writing the code was
\emacs\credits{http://www.gnu.org/software/emacs/}.

To learn the theory necessary to write these programs I had to read
several books and many articles.

This document was created and typeset using
\LaTeX{}\credits{http://www.latex-project.org/}. I chose to use
\LaTeX{} rather than Word because I, as a programmer, find it a lot
easier to use than Word.

The body text is typeset in the font \TeX{} Gyre Pagella, a clone of
Hermann Zapf's Palatino. The monospace text uses the font
\texttt{Luxi}.

The \LaTeX{} source files were also edited in \emacs, using the
AUC\TeX{}\credits{http://www.gnu.org/software/auctex/} package.

All the graphics of this document were created using
\tikzname \& \pgf\credits{http://sourceforge.net/projects/pgf/}

All the source files for both this document and the programs were
managed using \git\credits{http://git-scm.com/}, the version control system.

\section{Structure Of This Text}

The chapters in this text are laid out as follows:

\begin{description}
\item[Chapter 4] Any preliminary knowledge that is absolutely
  necessary to understand \textit{any} chapter of this text will be
  put here. We will also establish all the conventions and glossaries
  that we will be using throughout the rest of the text in this
  chapter.

\item[Chapter 5] We will in this chapter discuss how color is
  represented digitally; that is, using numbers.

\item[Chapter 6] The main topic of this chapter will be the \rle
  compression algorithm. This algorithm is used in the \tga format.

\item[Chapter 7] The \tga format is discussed.

\item[Chapter 8] The \lzw algorithm will be explained.

\item[Chapter 9] The \gif format will be the subject of this chapter.

\item[Chapter 10] To ensure the data integrity of PNG images something
  known as a \crc is used, which is the topic of this chapter.

\item[Chapter 11] The compression algorithm used in the PNG format,
  known as \deflate, is a rather complex algorithm that consists of
  several subalgorithms. One of these subalgorithms is known as
  Huffman Coding, which is what we will be discussing in this chapter.

\item[Chapter 12] The second subalgorithm of the \deflate algorithm is
  known as \lzone, which is the algorihtm we will discuss in this
  chapter. An improvement of \lzone called \lzss is also discussed in
  this chapter.

\item[Chapter 13] In this chapter we will combine \lzone and Huffman
  Coding into the final \deflate algorihtm.

\item[Chapter 14] In this chapter the \png format will finally be
  discussed.

\item[Chapter 14] All the former chapter will be discussed and we will
  attempt to draw a final conclusion from the project.

\end{description}

Small exercises can also be found in the text. They were primarily
added so that the reader could have a way of checking that he/she
truly have understood the contents of the text.


\section{Sample source code}

All the source code of the programs that I wrote during this project
can be found at the following address:
\url{https://github.com/Erkaman/digital-image-formats}.

In the \path|code| directory the \CC source files for the programs
written during the project can be found. The \path|log| directory
contains the log book of this project. And in the \path|latex|
directory you can find the \LaTeX{} source files that were compiled to
produce this document. But even if you don't understand anything of
the contents of the \path|code| folder, don't worry, because this text
was written to explain the concepts behind image formats as
intuitively as possible, without you having to understand the source
files for the programs that demonstrates them. But it is of course not
a bad thing if you do understand the source code files in the \path|code|
directory.
