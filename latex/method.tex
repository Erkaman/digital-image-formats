\begin{comment}
  \bibliography{project.bib}
\end{comment}

\chapter{Method}
\label{cha:method}

\section{Tools and sources}

To explore how the aforementioned image formats works, I decided to
write small programs that would analyze the numbers that a image
format consists. My programming language of choice for this was C and
the compiler I used to compile the programs was
GCC\footnote{\url{http://gcc.gnu.org/}}. For doing the memory debugging that
is often necessary in C programming, I used
Valgrind\footnote{\url{http://valgrind.org/}}, which at times was a
real life saver. For tests and automatizing tasks Bash and Python were
used. My editor of choice was \textsc{Emacs}.

To get the theory necessary to write these programs, I had to read
several books on the subjects. I also read a lot of articles.

Also, this document was created and typeset using
\LaTeXe{}\footnote{\url{http://www.latex-project.org/}}. I did not
chose to use word, because I, as a programmer, prefer \LaTeX{} over
words for several reasons:

\begin{itemize}
\item The default output o \LaTeX{} is almost always typographically
  superior to that of word.
\item It makes references management a breeze.
\item It makes creating indexes trivial.
\end{itemize}

The text is typeset in the fonts Charter(main body text) and
\texttt{inconsolata}(links). The \LaTeX{} source files were edited in
\textsc{Emacs} with the AUC\TeX{} package.

And the following \LaTeX{} packages were used to produce this document:

\texttt{inconsolata}, fontenc, mathdesign, charter, microtype,
makeidx, xcolor, float, verbatim, siunitx, todonotes, placeins,
booktabs, graphicx, caption, biblatex, \tikzname, \pgf , algorithms,
algorithmicx, fancyhdr, tocbibind.

All the source files in the of this entire project were managed with
\textsc{Git}, the version control system.

\section{Sample source code}

All the source code of the programs that I wrote during this project
can be found at the following address:
\url{https://github.com/Erkaman/digital-image-formats}. In the
\path|code| directory the C source files for the programs that
analyzes image formats can be found. The \path|log| directory contains
the log book for this project. And in the \path|latex| directory you
can find the \LaTeX{} source files that were used to produce this
document. But if you don't understand anything of the contents of the
\path|code| folder, don't worry, because this text was written to
explain the concepts behind image formats as intuitively as possible,
without you having to understand the source files for the programs
that demonstrates them. But it is of course not a bad thing if you do
understand the source files in the \path|code| directory.
