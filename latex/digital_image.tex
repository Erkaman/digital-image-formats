\begin{comment}
  \bibliography{project.bib}
\end{comment}

\chapter{Digital Image}
\label{cha:digital-image}

\begin{refsection}

  \section{Color}
  \label{sec:color}

  \subsection{What is color?}
  \label{sec:what-color}

  \newcommand{\bluewave}{\ensuremath{\SI{400}{\nano\meter}}}

  Color is light and light is composed of tiny particles traveling at
  different wavelengths. \cite{neider93:_openg_progr_guide} So light is
  just a wave. Figure \ref{fig:wave} demonstrates what a wave looks like
  and what a wavelength measures. Different colors have different
  wavelengths. For example: blue, as shown in figure \ref{fig:wave}, has
  a wavelength of \bluewave. \cite{ohlsson99:_digit_bild_kreat} That's
  $0.00000004$ meters!

  \begin{figure}[h!]
    \centering
    \inputtikz{wave.tex}
    \caption{A blue lightwave. $\lambda$ is the letter commonly used to
      represent  wavelength.}
    \label{fig:wave}
  \end{figure}

  But how do our eyes see these light waves? In our eyes, there are
  cells for perceiving three different kinds of wavelengths of light:
  red, blue and green. When these cells absorb light, we see color. And
  when these cells absorb mixed amounts and/or different amounts of red,
  blue and green light, we are able to perceive \textit{all} the other
  possible colors.

  \subsection{RGB}
  \label{sec:rgb}

  Color models \index{color model} are ways specifying color numerically \todo{mention
    additive and subtractive color models here}
  \cite{fadgi11:color_model}. They are of course very convenient for
  computers who are very good at dealing with numbers. A very widely
  used color model for representing color in computers is \textbf{RGB}
  \index{RGB}.

  \begin{figure}[h]
    \centering
    \inputtikz{rgb.tex}
    \caption{RGB color model}
    \label{fig:rgb}
  \end{figure}

  The color model is, of course, based on how our eyes perceive color,
  as mentioned in section \ref{set:what-color}. Observe figure
  \ref{fig:rgb}. As it can be seen, all the colors cyan ,magenta and
  yellow can be achieved by mixing either red, blue and green. Also note
  that white is achieved by mixing all of three colors. Black is
  represented by no light at all.

  The different amounts of red ,blue and green in a color, are called
  their respective color channels \index{color channel}.

  \subsection{Alpha channel}
  \label{sec:alpha_chan}

  But there is actually even more to RGB. There is an extended color
  model of RGB that's called RGBA \index{RGBA}(it is sometimes also
  referred to as ARGB \index{ARGB}). In this model a new channel is
  added: the alpha channel \index{alpha channel}. This new channel
  represents the opacity of a color. Opacity is simply the opposite of
  transparency \cite{porter84_compos_dig_img}. Take
  a good look at figure \ref{fig:alpha}. It demonstrates how a square
  gets a lower and lower value of its alpha channel until it turns
  transparent. In the leftmost part of the rectangle , the color has
  its maximum possible alpha channel, meaning that it's fully
  visible. On the other hand, the leftmost color has its lowest
  possible alpha value, $0$, meaning that it's fully transparent.

  \begin{figure}[h!]
    \centering
    \inputtikz{alpha.tex}
    \caption{A rectangle with a lower and lower alpha channel.}
    \label{fig:alpha}
  \end{figure}

  \subsection{CMYK}
  \label{sec:cmyk}


  \section{Color depth}
  \label{sec:color-depth}

  \newcommand{\rgbtrip}[3]{( \textcolor{red}{#1},\textcolor{green}{#2},\textcolor{blue}{#3})}

  But up until now, I have said nothing about how the computer
  represents these color models. Let us start with:

  \subsection{24-bit color}
  \label{sec:24-bit-color}

  RGB color model each color is just a combination red,blue and
  green. In the RGB color model these channels are given values. So let
  us represent colors as a triple of three numbers, like this:
  \rgbtrip{123}{21}{91}, which is \textcolor[RGB]{123,21,91}{this color}.

  An 8-bit number has, as you should be familiar with, only $256$ possible values. If we
  assign an  8-bit number to each color channel, the color's total size
  will be 24-bits. This way of representing color is called 24-bit color
  \index{24-bit color}. You can also say that its color depth \index{color
    depth} is 24 bits.

  \newcommand{\selfcolor}[1]{\textcolor{#1}{#1}}

  A full \selfcolor{red} color is represented by the triple \rgbtrip{255}{0}{0}. In
  that case, have a guess at what color is represented by
  \rgbtrip{255}{0}{255}. If haven't guessed it already, please study
  figure \ref{fig:rgb}. As you probably already have guessed, this color
  is \selfcolor{yellow}. And in the same number mixing and matching way, many other
  colors can be represented. Study table \ref{tab:color-examples} for
  examples of other color mixes.

  \newcommand{\colorrow}[4]{  \rgbtrip{#1}{#2}{#3} &
    \textcolor[RGB]{#1,#2,#3}{#4} \\ \hline}

  \begin{table}[h!]
    \centering
    \begin{tabular}[h!]{|l|l|}
      \hline
      RGB triple & Resulting color \\ \hline
      \colorrow{255}{215}{0}{Gold}
      \colorrow{165}{42}{42}{Brown}
      \colorrow{255}{0}{255}{Purple}
      \colorrow{255}{192}{203}{Pink}
      \colorrow{255}{165}{0}{Orange}
      \colorrow{0}{0}{139}{Dark Blue}
      \colorrow{105}{105}{105}{Dim Grey}
      \colorrow{0}{100}{0}{Dark Green}
      \colorrow{205}{92}{92}{Indian Red}
    \end{tabular}
    \caption{Colors}
    \label{tab:color-examples}
  \end{table}

  But of course, not all colors can be represented using just
  24-bit. The number of All possible colors is infinite. But you can
  represent an awful not colors with only 24-bits. How many? Well, every
  color channel can have $256$ different values. There are $3$
  channels. Hence, there are $256^3 = 16777216$ different colors.

  \begin{figure}[h!]
    \centering
    \begin{tikzpicture}
      % left
      \shade[yslant=-0.5,upper left=yellow,upper right=white,
      lower left=red,lower right=pink] %c
      (0,0) rectangle +(3,3);


      % right
      \shade[yslant=0.5,upper left=yellow,upper right=white,
      lower left=red,lower right=green]
      (3,-3) rectangle +(3,3);

      % top
      \shade[yslant=0.5,xslant=-1,upper left=green,upper right=white,
      lower left=yellow,lower right=white] (6,3) rectangle +(-3,-3);

    \end{tikzpicture}
    \caption{RGB color cube TODO: Make.}
    \label{fig:color-cube}
  \end{figure}

  \newcommand{\rgbaquad}[4]{(
    \textcolor{red}{#1},\textcolor{green}{#2},\textcolor{blue}{#3},\textcolor{gray}{#4} )}

  Adding alpha channels to this color representation is trivial, just
  add a fourth channel. Here is for example the color green halve
  transparent: \rgbaquad{0}{0}{255}{125}. The only real difference is
  that a color will require 32-bits of storage if an alpha channel is needed.

  \subsection{Grayscale}
  \label{sec:other-colors-depths}

  % check the handbook for this

  \subsection{Indexed Color}
  \label{sec:indexed-color}

  % 8-bit color 1-bit color, palettes need to be covered.-

  \printbibliography[heading=subbibliography]
\end{refsection}