\begin{comment}
  \bibliography{project.bib}
\end{comment}

\chapter{Conclusion}
\label{cha:conclus}

So how are images represented numerically? The answer to that question
is simple: by using a color model separate pixels are represented
numerically and by storing a long sequence of such pixels images are
formed. A way of formalizing and standardizing how and what data is to
be stored with an image is an image format. They give software
developers, such as myself, a common and unified way to talk about
images.

However, since raw color data takes up so much space, it is often
absolutely necessary to compress it. And here is where things truly
get tough. Or is it really that hard? Because when you really look
into compression algorithms like Huffman coding and \lzw, you will
notice that they have a beauty in their simplicity that is almost
\textit{frightening}. For example, the main idea behind Huffman coding
is simply that more frequent symbols are assigned binary numbers with
less bits. The \lzfam of methods could also be said to reason in the
same way; but here the main goal is to rather assign frequently
occurring formations of symbols, words, more space efficient
representations. And as was shown by the \deflate algorithm, these
word representations can also be Huffman coded to even further
increase compression.

But I will admit that the literature on data compression I had found
was very difficult to read. Often the descriptions of the compression
algorithms were very terse, and as a result very difficult to
understand. It for example took me several weeks until I had fully
understood how the \lzw algorithm worked! Looking back and rereading
my own description of the algorithm, I find it almost impossible to
believe that it took me that long of a time to understand that simple
of an algorithm! But at the beginning of 2012 it felt like I had
finally gotten over the initial learning hurdle, because I could now
with ease read and understand the literature on data compression that
I had initially found impenetrable!

But understanding data compression was not the only great difficulty I
had during the project. Understanding the computation of a \crc was
something that I found much harder than I had ever expected it to be,
and it took several weeks until I had fully understood and documented
this almost arcane subject. Because the mathematics that \crc is based
is so advanced(at least to me), I had little to no choice than just
having to accept that the algorithm works, but without understanding
why!  Having to accept that sort of thing is something I have always
found very frustrating; I want to understand \textit{why} things work,
not merely \textit{how} they are done! But in the end, I had no choice
but having to abandon the idea of fully understanding \crc, since
there were several other things that I had also to get done in the
project.

Another part that proved to be a great obstacle to me was
understanding color theory. Color theory proved very difficult for me,
because, again, the mathematics was simply beyond me. I had plans for
covering several more advanced color spaces, and not only \rgb, but in
the end I simply had to give up on this idea because I simply could
not understand the math behind them.

I learned many things during this project: how to be even more
self-reliant(to be frank, my supervisor, Aref, was basically
\textit{useless} to me throughout this entire project); how to read
scientific papers; how much fun and rewarding data compression is(!);
how to write better explanations; how to properly focus on a big
project for a longer period of time(because I have a rather short
attention span); and finally, it made me really appreciate how complex
things we easily take for granted can be. Image formats are complex
but easy to use to the same extent that things like cars and computers
are; how well software hides the underlying complexity of an image
format like \png is almost ghastlingly beautiful! But it is a bit of a
pity as well; that is, how well it hides the fact that generations
upon generations of men's work were built upon the format. Just take
the compression algorithm, \deflate, used by the format: first, Claude
Shannon founded the area of Information Theory, and building upon his
work David Huffman invented Huffman coding and Lempel and Ziv invented
\lzone(and \lztwo), and in the \deflate algorithm all of this was
combined by Phil Katz. Truly is our modern society \textit{standing on
  the shoulders of giants!}
