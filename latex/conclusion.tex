\begin{comment}
  \bibliography{project.bib}
\end{comment}

\chapter{Conclusion}
\label{cha:conclus}

\section{Answer to the question}

So how are images represented numerically? The answer to that question
is simple: by using a color model separate pixel colors are
represented numerically and by storing a long sequence of such colors
model numbers images are formed.

However, since raw color take up so much space, it often absolutely
necessary to compress it. And here is where things truly gets
tough. Or is it really that hard? Because when you really look into
compression algorithms like Huffman coding and LZW, you will notice
that they have an almsot frighting beauty in their simplicity. The
main idea behind Huffman coding is simply:

\begin{quote}
  If we assume that some symbols are more frequent in the data, then
  by making these symbols more space efficient we can achieve
  significant compression.
\end{quote}

The LZ-family methods could also be said to reason in the same way;
here the main goal is to instead frequently assign formations of
symbols, words, more space efficient representations. And as was shown
by the DEFLATE algorithm, these word representation can also be
Huffman coded to even further increase compression.

And Once you have gotten rid of your initial fright of compression
algorithm, then you will truly see how simple image formats truly
are.

\section{Difficulties}

But despite this, a great difficulty during the entire project was
understanding compression algorithms. The difficulty of the
litterature I had found on the subject of data compression is partly
to blame for this; these texts often reasoned in a very mathematical
and succinct way, often making them very hard to read.

% mention difficulties during coding.

\section{Final thoughts}

But overall I had very fun doing this project because these is nothing
I enjoy more than programming and learning new things. And as the
length of this written part probably indicates, there were quite a lot
to learn from this project!  
