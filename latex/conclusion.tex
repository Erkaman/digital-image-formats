\begin{comment}
  \bibliography{project.bib}
\end{comment}

\chapter{Conclusion}
\label{cha:conclus}

\section{Answer to the question}

So how are images represented numerically? The answer to that question
is simple: by using a color model separate pixels are represented
numerically and by storing a long sequence of such pixels images are
formed.

However, since raw color takes up so much space, it is often
absolutely necessary to compress it. And here is where things truly
get tough. Or is it really that hard? Because when you really look
into compression algorithms like Huffman coding and \lzw, you will
notice that they have a beauty in their simplicity that is almost
frightening. For example, the main idea behind Huffman coding is
simply that more frequent symbols are assigned binary numbers with
less bits.

The \lzfam of methods could also be said to reason in the same way;
but here the main goal is to rather assign frequently occurring
formations of symbols, words, more space efficient
representations. And as was shown by the \deflate algorithm, these
word representations can also be Huffman coded to even further
increase compression.

But I will admit that the sources in data compression I had found were
very difficult reads. Often the descriptions of the compression
algorithms were very terse, and as a result very difficult to
understand. It for example took me several weeks until I had fully
understood how the \lzw algorithm worked! Rereading my own description
of the algorithm, I find it almost impossible to believe that it took
me that long of a time to understand that simple of an algorithm!

But at the beginning of 2012 it felt like I had finally gotten over
the intial hurdle because now I could without any sorts of problems
read and understand the literature on data compression that I had
initially found impenetrable!

But understanding data compression was not the only great difficulty I
had during the project. Understanding the computation of a \crc was
something that I found much harder than I expected, and it took
several weeks until I had fully understood and documented this arcane
subject. Because the mathematics that \crc is based is so advanced,
you have little to no choice than just having to accept that the
algorithm works. Having to accept that is something that I have always
found very frustrating; I wanted to understand how things, not only
know merely how they are done! But in the end, I had no choice but
having to abbondon the idea of fully understanding \crc, since there
were several other things that I had also to get done in the project.

Another parts that proved to be a great obstacle to me was
understanding color theory. Color Theory proved very difficult for me,
because, again, the mathematics was simply beyond me. I had plans for
covering several more advanced color spaces that are also used in the
project, but I simply had to give up because I simply could not
understand the math behind them. This is also the reason why the
chapter on digital color is so short compared to the rest of the
chapters in the text.

But overall I had very fun doing this project because these is nothing
I enjoy more than programming and learning new things. And as the
length of this written part probably indicates, there were quite a lot
to learn from this project!
