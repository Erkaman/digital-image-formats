\begin{comment}
  \bibliography{project.bib}
\end{comment}

\chapter{Conclusion}
\label{cha:conclus}

So how are images represented numerically? The answer to that question
is simple: by using a color model separate pixels are represented
numerically and by storing a long sequence of such pixels images are
formed.

However, since raw color takes up so much space, it is often
absolutely necessary to compress it. And here is where things truly
get tough. Or is it really that hard? Because when you really look
into compression algorithms like Huffman coding and \lzw, you will
notice that they have a beauty in their simplicity that is almost
frightening. For example, the main idea behind Huffman coding is
simply that more frequent symbols are assigned binary numbers with
less bits.

The \lzfam of methods could also be said to reason in the same way;
but here the main goal is to rather assign frequently occurring
formations of symbols, words, more space efficient
representations. And as was shown by the \deflate algorithm, these
word representations can also be Huffman coded to even further
increase compression.

But I will admit that the literature on data compression I had found
was very difficult to read. Often the descriptions of the compression
algorithms were very terse, and as a result very difficult to
understand. It for example took me several weeks until I had fully
understood how the \lzw algorithm worked! Looking back and rereading
my own description of the algorithm, I find it almost impossible to
believe that it took me that long of a time to understand that simple
of an algorithm!

But at the beginning of 2012 it felt like I had finally gotten over
the intial hurdle, because now I could without any problem read and
understand the literature on data compression that I had initially
found impenetrable!

But understanding data compression was not the only great difficulty I
had during the project. Understanding the computation of a \crc was
something that I found much harder than I had ever expected it to be,
and it took several weeks until I had fully understood and documented
this arcane subject. Because the mathematics that \crc is based is so
advanced, you have little to no choice than just having to accept that
the algorithm works. Having to accept that is something that I have
always found very frustrating; I want to understand how things work,
not only merely how they are done! But in the end, I had no choice but
having to abandon the idea of fully understanding \crc, since there
were several other things that I had also to get done in the project.

Another part that proved to be a great obstacle to me was
understanding color theory. Color Theory proved very difficult for me,
because, again, the mathematics was simply beyond me. I had plans for
covering several more advanced color spaces that are also used, but in
the end I simply had to give up on this idea because I simply could
not understand the math behind them.

This project also really made me appreciate how complex things we take
for granted can be. Image formats are complex but easy to use to the
extent that things like cars and computers are. How well software
hides the underlying complexity of an image format like \png is almost
ghastlingly beautiful! But it is a bit of a pity as well; that is, how
well it hides the fact generations and generations of men's work were
built upon the format. Its compression methods could said to have been
started by Claude Shannon when he created the area of Information
Theory. Building on Shannon's early work, David Huffman invented
Huffman coding, Lempel and Ziv invented \lzone(and \lztwo), and in the
\deflate algorithm all of this was combined by Phil Katz! And the
color model used in \png, \rgb, builds is based directly of the
Young-Helmholtz theory of Thomas Young and Hermann von Helmholtz,
which was way back in 1802! Learning to appreciate this enormous
underlying complexity under things we take for granted is something we
all as humans should do, because not doing so would be ungrateful to
men behind it!

But overall I had very fun doing this project because there is nothing
I enjoy more than learning new things. It was also very fun because it
turned out to be a perfect mix of programming and mathematics! And as
the length of this written part probably indicates, there was quite a
lot to learn from this project! And I also hope that you, as the
reader, also had as much fun in reading this text as I had in writing
in it. But if you did not, then that would be a real pity\dots
