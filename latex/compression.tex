\begin{comment}
  \bibliography{project.bib}
\end{comment}
\chapter{Compression techniques}
\label{cha:compress-techn}

\begin{refsection}

\section{Some terminology}
\label{sec:some-terminology}

\subsection{Encoding and decoding}
\label{sec:encoding-decoding}

Coding data into another form is often referred to as
\textbf{encoding}\index{encoding}. Converting that encoded data back
to its original form is called
\textbf{decoding}\index{decoding}. \todo{Need sources on these two
  terms.} In the circumstances of compression techniques, encoding and
decoding are also often referred to as
\textbf{compression}\index{compression} and
\textbf{decompression}\index{decompression}.

\subsection{Compression ratio}
\label{sec:compression-ratio}

The compression ratio\index{compression ratio} is a very useful concept when talking about a
compression algorithms efficiency. The compression ratio of an
algorithm is calculated like this:

\begin{equation}
  \label{eq:compress-ratio}
  {\rm Compression\;Ratio} = \frac{\rm Compressed\;Size}{\rm Uncompressed\;Size}
\end{equation}

So the compression ratio basically tells how much data was
compressed. Of course, the lower compression ratio the better. But
there are ratios to take in heed, like performance and complexity of
the algorithm, but we will ignore these concepts in this book to book
to keep things simple. We will for the rest of the book specify the
compression ratio in percentages, $\%$\cite{Salomon:2004:DCC}.

\section{Pseudocode conventions}
\label{sec:pseud-conv}

In this chapter, we'll actually start covering some algorithms. We
will describe these using pseudocode. The pseudocode will try to kept
as traditional as possible, but we'll still need to establish some
conventions for it:

\subsection{Boolean operators}
\label{sec:boolean-operators}

For the boolean operators we'll be using the following
symbols:

\begin{description}
\item[$\NOT$] logical \textit{not}
\item[$\AND$] logical \textit{and}
\item[$\OR$] logical \textit{or}
\end{description}

\subsection{Bitwise operators}
\label{sec:bitwise-operators}

\newcommand{\C}{\textsc{C}}

We will for example also use the bitwise operators. If you
haven't even heard of them, look up them and study them on your own. I
assumed from the beginning of this book that such knowledge should be
obvious to you. We will be using the notation used in \C{} to
represent them in pseudocode:

% looks ugly when typeset. check this:
% http://www.troubleshooters.com/linux/lyx/ownlists.htm

\begin{description}
\item[$\BitNeg$] Bitwise \textit{NOT}
\item[$\BitAnd$] Bitwise \textit{AND}
\item[$\BitOr$] Bitwise \textit{OR}
\item[$\BitXor$] Bitwise \textit{XOR}
\item[$\ShiftLeft$] Left bit shift
\item[$\ShiftRight$] Right bit shift
\end{description}

Notice that we are using $\BitXor$ for typesetting bitwise
\textit{XOR} rather than the traditional \C{} notation. This is due to
the fact that we'd otherwise confuse it with logical and, $\AND$.

\subsection{Pseudocode Functions}
\label{sec:pseudocode}

We'll be dealing with files in these compression algorithms. So we'll
need to introduce several functions for dealing with files:

\begin{description}
\item[\textproc{ReadByte}] It is assumed from the beginning of the
  algorithm that a file has already been opened for reading. This
  could be the file we're either trying to compress or
  decompress. This function reads a byte from that file.

\item[\textproc{WriteByte}] At the beginning of every algorithm, we also
  assume that there is a file opened for output. This function writes
  a byte to that file.

\item[\textproc{EndOfFileReached}] \True{} if the end the file we're
  reading from have been reached.

\end{description}

Although we will be dealing mostly with bytes in these algorithms, they could
easily by generalized to compression or decompressing other integer
types likes \texttt{long} or \texttt{short}.

\subsection{Other conventions}
\label{sec:other-conventions}

We will also a be using a custom control structure called \algorithmicrepeat,
which is demonstrated in algorithm \ref{alg:repeat}.

\begin{algorithm}
  \caption{The repeat control structure.}
  \label{alg:repeat}
  \begin{algorithmic}[1]
    \Repeat{$n$}
      \State $actions\ldots$ \Comment{Repeats $actions\ldots n$ times} %
    \EndRepeat
  \end{algorithmic}
\end{algorithm}

The start of a comment is indicated by the symbol \commentsymbol.

\section{Run Length Encoding}
\label{sec:rle}

\subsection{Naive version}
\label{sec:naive-version}

\subsubsection{Description}
\label{sec:description}

One the most simple compression algorithms is known as \textit{Run
  Length Encoding}\index{Run Length Encoding}, or simply,
RLE\index{RLE}\cite{nagarajan11:_enhan_approac_run_lengt_encod_schem}.
The algorithm works like this: Sequences of consecutive data are
stored in packets containing a data count and a single value, rather
than the entire run of data.

But what does that actually \textit{mean}? Let us consider the string

\begin{indentpar}
  WWWWAAAACCCCCCQ
\end{indentpar}

As we can see, there are several runs of duplicated data in this
string. What if we instead of storing all of the duplicated data,
stored a number that tells how many times the following character
is replicated? And that's the RLE algorithm in a nutshell.

So when running the RLE encoding algorithm on this string, it is
compressed like this:

\newcommand{\pkt}[2]{\textbf{#1}#2}

\begin{indentpar}
  \pkt{4}{W}\pkt{4}{A}\pkt{6}{C}\pkt{1}{Q}
\end{indentpar}

So all the RLE encoding algorithm really does is encode long
consecutive runs of the same value as a count and a single data
value. We from now on will call a sequence of a data count and the
data itself a \textit{packet}\index{packet}.

But however, the same thing is also done for single runs, meaning that
the algorithm can potentially explode the size of the original data
rather than compress it. Consider,for example, the string

\begin{indentpar}
  eric
\end{indentpar}

this will actually get ``compressed'' down to

\newcommand{\spkt}[1]{\pkt{1}{#1}}

\begin{indentpar}
  \spkt{e}\spkt{r}\spkt{i}\spkt{c}
\end{indentpar}

So the size actually doubled for this kind of data! So for very data
with very varying values, like photos and books, this algorithm will
do just horrible. But in data in which the values are not varied much
at all, like that robot logo we showed in the last chapter, even for
such a naive algorithm, it will do quite good.

In table \ref{tab:rle-comp} the compression ratio of the algorithm for
some sample data is demonstrated. As can be seen from the table, the
comrpession algorithm is not a single value that can be assgined to an
algorithm, but it highly depends on the kind of data that is being
compressed. Some data may yield a high compression ratio for a simple
algorithm, while some data may potentially explode the size.

\begin{table}
  \centering
  \begin{tabular}{llr}
    \toprule
    Uncompressed data & compressed data & compression ratio($\%$) \\
    \midrule
    AAAEEERR & \pkt{3}{A}\pkt{3}{E}\pkt{2}{R} & $75$ \\
    ericarne & \spkt{e}\spkt{r}\spkt{i}\spkt{c}\spkt{a}\spkt{r}\spkt{n}\spkt{e} & $200$ \\
    BBBWWWBBBWWWBBBWWW & \pkt{3}{B}\pkt{3}{W}\pkt{3}{B}\pkt{3}{W}\pkt{3}{B}\pkt{3}{W} & $66.666\dots$ \\
    \bottomrule
  \end{tabular}
  \caption{RLE compression}
  \label{tab:rle-comp}
\end{table}

\subsubsection{Implementation}
\label{sec:implementation-1}

\newcommand{\eof}{\ensuremath{\VoidCall{EndOfFileReached}}}
\newcommand{\neof}{\ensuremath{\NOT \VoidCall{EndOfFileReached}}}

All of this is essentially trivial to implement in code. Let us first
consider the compression algorithm, which is shown in algorithm
\ref{alg:rle-enc}. What follows is the explanation of this algorithm.

\begin{algorithm}
  \caption{Encoding a file using RLE.}
  \label{alg:rle-enc}
  \begin{algorithmic}[1]

    \Let{$length$}{$1$}
    \Let{$passedFirstCharacter$}{\False}
    \Let{$c_2$}{ \VoidCall{ReadByte}}

    \While{\neof}

      \If{$passedFirstCharacter$}
        \If{$c_2 = c_1 \AND length < 255$}
          \Let{$length$}{$1 + length$}
        \Else
          \State \Call{WriteByte}{$length$}
          \State \Call{WriteByte}{$c_1$}
          \Let{$length$}{$1$}
        \EndIf
      \EndIf

      \Let{$passedFirstCharacter$}{\True} %
      \Let{$c_1$}{$c_2$}
      \Let{$c_2$}{ \VoidCall{ReadByte}}

    \EndWhile

    \If{$passedFirstCharacter$} \Comment{Write the last bytes.}
      \State \Call{WriteByte}{$length$}
      \State \Call{WriteByte}{$c_1$}
    \EndIf
  \end{algorithmic}
\end{algorithm}

First we need to make sure we have two characters to compare for
equality and possible pack together for this we use the flag
$passedFirstCharacter$ . Once we have two character then we can
compare then for equality and if they are equal we increase the packet
length counter. If they are not equal or the current packet is full,
we write the packet.

To write a packet, we first write the length of the packet and then
the actual data that is repeated by the package. And then the counter is
reset.

However, once we have reached the end of the file, we still haven't
written the last packet. But if the file was entiely empty, we'd have
nothing to write, so we need to check that we have passed the first
character. If we have, we simply write the last packet in the same
manner as all the other packets.

\begin{algorithm}
  \caption{Decoding a RLE encoded file.}
  \label{alg:rle-dec}
  \begin{algorithmic}[1]

    \Let{$length$}{\VoidCall{ReadByte}}
    \Let{$c$}{\VoidCall{ReadByte}}

    \While{\neof}
      \Repeat{$length$}
        \State \Call{WriteByte}{$b$}
      \EndRepeat

      \Let{$length$}{\VoidCall{ReadByte}}
      \Let{$c$}{\VoidCall{ReadByte}}

    \EndWhile
  \end{algorithmic}
\end{algorithm}

The ridiculously simple RLE decoding algorithm is shown in algorithm
\ref{alg:rle-dec}.

It is very simple: While we still haven't reached the end of the file,
we read a packet length and the packet data and then write out the
packet data to a uncompressed file $length$ the same number of times
as the packet length.



\subsection{PackBits version}
\label{sec:packbits-version}



But as we previously stated,this algorithm just do horrible on data that
is not at all repetitive. For example data such as this

\begin{indentpar}
  ERICARNE
\end{indentpar}

But we needn't to forget that even for such a simple algorithm it can do
wonders on data that is actually repetitive, like for example

\begin{indentpar}
  WWWWWWWWWWWWAAAAAAAAAAAAAAARRRRRRRRRRRR
\end{indentpar}

So the question is, can we get rid of the effect of potentially
doubling the size, yet still being able to effectively compress data
such as the above?  It turns out that we can. We just need to revise
our packet format a tiny bit.

A byte is written as the following bit pattern

\begin{indentpar}
  $00000000$
\end{indentpar}

Let us now call the first byte of a packet, the part that specifies
the length of a packet, the packet head\index{packet head}. Let us
reserve the last bit in the packet header for a flag, and use the
seven remaining bits for specifying the size as is demonstrated in
figure \ref{fig:packbits-header}.

\begin{figure}
  \centering
  $\mbox{\fontsize{30}{0} $\underbrace{0}_\text{Flag bit}\overbrace{0000000}^\text{Length bits}$}$
  \caption{PackBits header}
  \label{fig:packbits-header}
\end{figure}

But this last bit of course won't be wasted, it will specify the
\textit{type} of the packet. If the value of flag is 1, it will be a
raw packet, elsewise it will be a run length packet.

A run-length packet is just your ordinary run length packet, you know,
the one we used in the last algorithm. But we need to keep that in
this case only the first seven bytes are used for specifying the size.

But however, a raw packet means that the data following the firs byte
of the packet is just a sequence of uncompressed raw data. The length
of this sequence is the value specified by the first seven bytes.

And the length of a run-length packet is also these seven first
bytes

Both of these packet types use only seven bits for specifying their
sizes. Hence, the new maximum length of a run length packet is the
maximum number of an unsigned 7-bit number, which is $2^7 -1 = 127$.

So we just have lost some valuable storage space, but what did we get in
return? Well, now the string

\begin{indentpar}
  ERICARNE
\end{indentpar}

gets compressed down

\begin{indentpar}
  136ERICARNE
\end{indentpar}

(do keep in mind that 136 is a byte of the value 136, and not the
sequence of the numbers 1,3 and 6)

Which is a huge gain!

But why is the packet head number so huge? Well, that's obviously
because the last bit flag in a byte has the value $128$. If we toggle
off that bit we end up with the packet length: $136 \BitXor 128 = 8$.

And while \textit{not} exploding the size of certain kinds of data,
this algorithms can also reap all of the benefits the old, naive, version
of the algorithm had. So the string

\begin{indentpar}
  WWWWAAAACCCCCC
\end{indentpar}

still gets compressed down to

\begin{indentpar}
  4W4A6C
\end{indentpar}

And all of this in exchange for just one bit!

As an exercise for the reader, examine the following compression and
validate its correctness:

\begin{indentpar}
  EEEEEEERIC
\end{indentpar}

to

\begin{indentpar}
  7E131RIC
\end{indentpar}

To further demonstrate the efficiency of this algorithm, please study
table \ref{tab:packbits-comp}. As it can be seen, this algorithm
maintains the occasional efficency of RLE, while not doubling the size
of non-repetitive data.

\begin{table}
  \centering
  \begin{tabular}{llr}
    \toprule
    Uncompressed data & compressed data & compression ratio($\%$) \\
    \midrule
    AAAEEERR & \pkt{3}{A}\pkt{3}{E}\pkt{2}{R} & $75$ \\
    ericarne & \pkt{136}{ericarne} & $112.5$ \\
    BBBWWWBBBWWWBBBWWW & \pkt{3}{B}\pkt{3}{W}\pkt{3}{B}\pkt{3}{W}\pkt{3}{B}\pkt{3}{W} & $66.666\dots$ \\
    \bottomrule
  \end{tabular}
  \caption{PackBits compression}
  \label{tab:packbits-comp}
\end{table}


\subsubsection{Usage}
\label{sec:usage}

This algorithm is commonly referred to as the PackBits\index{PackBits} algorithm, but
it is really just a variant of the RLE algorithm \cite{96:_techn_note_tn102}.

This algorithm is actually the one used to compress color data in the
TGA image format \cite{91:_truev_tga_file_format_specif}.

\subsubsection{Implementation}
\label{sec:packbits-implementation}

\paragraph{Writing the packets}
\label{sec:writing-packets}

Let us first consider how to write a raw packet. This is shown in
function \textproc{writeRawPacket}(algorithm
\ref{alg:raw-packet}). The packet head is first given an initial value
of zero to specify that it's a raw packet. Then it's applied to
bitwise operator \textit{OR} with the length of the packet, to pack in
the length in the head. After that, the entire raw data which is
contained in the array $data$ is written one by one to the compressed file.

\begin{algorithm}
  \caption{Writing a raw packet.}
  \label{alg:raw-packet}
  \begin{algorithmic}[1]
    \Require $length$ is a list or array containing the data to be written
    \Function{writeRawPacket}{$length,data$}
      \Let{$packetHead$}{$0$}
      \Let{$packetHead$}{$packetHead \BitOr length$}
      \State \Call{writeByte}{$packetHead$}
      \ForEach{$raw$}{$data$}
        \State \Call{writeByte}{$raw$}
      \EndForEach
    \EndFunction
  \end{algorithmic}
\end{algorithm}

And writing a Run Length Packet is even easier! Please observe
algorithm \ref{alg:rle-packet}. Since it is a run length packet in
this case, we toggle the last bit on by assigning it to an initial
value of $128$. Then the length is packed in the head, then we simply
write the data value that is to be replicated $length$ times to the
file.

\begin{algorithm}
  \caption{Writing a run length packet.}
  \label{alg:rle-packet}
  \begin{algorithmic}[1]
    \Require $length > 0$
    \Function{writeRunLengthPacket}{$length,data$}
      \Let{packetHead}{$128$}
      \Let{packetHead}{$packetHead \BitOr length$}
      \State \Call{writeByte}{$packetHead$}
      \State \Call{writeByte}{$data$}
    \EndFunction
  \end{algorithmic}
\end{algorithm}

\paragraph{Compression and decompression}
\label{sec:compr-decompr}

Now that we know how to write the packets, let us consider how to
implement the algorithm itself.

The encoding algorithm can be seen in algorithm
\ref{alg:packbits-enc}. This algorithm has a significantly more
complex control flow than any other algorithm up to this point, so please
read the following explanation carefully.

In the same manner as the previous version of the algorithm we first
read in the first two values. First we check if the current packet is
full. If it is, we write it differently depending on whether it's a
raw or run length packet.

If the current two values are equal, we first need to check whether we
are currently writing a raw packet. If we are, we write the raw
packet. And then we increase the run length packet length by one.

If the packet is not full and it is not writing a run length packet,
it must be writing a raw packet. However, if we are transitioning from
writing run length to a raw packet, we need to, of course, write out
the run length packet first. Else we put the current value into the
data array and increase the count.

And we must of course write out the last packet after the end of the
file has been reached. The last step is done differently depending on
the type of the current packet. \todo{the explanation needs a lot more
work.}

\begin{algorithm}
  \caption{Encoding a file using PackBits.}
  \label{alg:packbits-enc}
  \begin{algorithmic}[1]
    \Require $RawPacket = 0,RunLengthPacket = 1$

    \Let{$length$}{$0$}
    \Let{$passedFirstCharacter$}{\False}
    \Let{$c_2$}{ \VoidCall{ReadByte}}
    \Let{$packetType$}{$RawPacket$}

    \While{\neof}

      \If{$passedFirstCharacter$}
        \If{$length = 127$}
          \Comment{If the packet is full.}
          \If{$length = 127$}

            \State \Call{writeRunLengthPacket}{$length,c_1$}
            \Let{$packetType$}{$RawPacket$}
            \Let{$length$}{$0$}

          \Else

            \State \Call{writeRawPacket}{$length-1,data,out$}
            \Let{$length$}{$1$}

          \EndIf
        \ElsIf{$c_2 = c_1$}

          \If{$packetType = RawPacket \AND length > 0$}
            \State \Call{writeRawPacket}{$length-1,data,out$}
            \Let{$length$}{$0$}
          \EndIf

          \Let{$length$}{$1 + length$}
          \Let{$packetType$}{$RunLengthPacket$}

        \Else

          \If{$packetType = RunLengthPacket$}

            \State \Call{writeRunLengthPacket}{$length,c_1$}
            \Let{$packetType$}{$RawPacket$}
            \Let{$length$}{$0$}

          \Else

            \Let{$data[length]$}{$c_1$}
            \Let{$length$}{$length + 1$}
            \Let{$packetType$}{$RawPacket$}

          \EndIf

        \EndIf
      \EndIf

      \Let{$passedFirstCharacter$}{\True}
      \Let{$c_1$}{$c_2$}
      \Let{$c_2$}{ \VoidCall{ReadByte}}

    \EndWhile

    \If{$passedFirstCharacter$}
      \If{$packetType = RunLengthPacket$}
        \State \Call{$writeRunLengthPacket$}{$length,c_1$}
      \Else
        \Let{$data[length]$}{$c_1$}
        \State \Call{$writeRawLengthPacket$}{$length,data$}
      \EndIf
    \EndIf
  \end{algorithmic}
\end{algorithm}

Phew! Thankfully, the decoding algorithm is not as complex. It is
shown in algorithm \ref{alg:packbits-dec}.

First we read in the packet head of the packet. If the end of the file
has not been reached, we first extract the length out of the packet
head using the operation $x \BitAnd 127$. This makes sense, as
$127$ has the bit pattern

\begin{indentpar}
  01111111
\end{indentpar}

Thus, we only extract the length part of the head, and not the highest
bit. Then we check if the highest bit is toggled, and if such is the
case, then it's a run length packet. And so we read in the data part of
the packet and repeat it $length$ times. Else, it is a raw packet, and
we just read the data and write it back $length$ times.

\begin{algorithm}
  \caption{Decoding a RLE packbits encoded file.}
  \label{alg:packbits-dec}
  \begin{algorithmic}[1]

    \Let{$head$}{\VoidCall{ReadByte}}

    \While{\neof}

      \Let{$length$}{$head \BitAnd 127$}

      \If{$head \BitAnd 128$}
        \Let{$data$}{\VoidCall{ReadByte}}

        \Repeat{$length$}
          \State \Call{WriteByte}{$data$}
        \EndRepeat
      \Else

        \Repeat{$length$}
          \Let{$data$}{\VoidCall{ReadByte}}
          \State \Call{WriteByte}{$data$}
        \EndRepeat

      \EndIf

      \Let{$head$}{\VoidCall{ReadByte}}

    \EndWhile
  \end{algorithmic}
\end{algorithm}

\subsection{Last things}

Both of these algorithm are implemented in the sample program
\path|code/tga/tga|. You can get instructions on how to use this command
line program by passing the flag \verb|-h| to it. \todo{Improve the damn
  parsing and the command line format.}

The source code to this program is in the file
\path|code/tga/tga.c|. The RLE encoding and decoding functions are
found their respective function \verb|RLE_Encode| and
\verb|RLE_Decode|. The PackBits encoding and decoding functions are
\verb|packBitsEncode| and \verb|packBitsDecode|.

\FloatBarrier

\printbibliography[heading=subbibliography]

\end{refsection}
