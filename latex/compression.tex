\begin{comment}
  \bibliography{project.bib}
\end{comment}
\chapter{Compression Techniques}
\label{cha:compress-techn}

\begin{refsection}

\section{Some Terminology}
\label{sec:some-terminology}

\subsection{Encoding And Decoding}
\label{sec:encoding-decoding}

Coding data into another form is often referred to as
\textbf{encoding}\index{encoding}. Converting that encoded data back
to its original form is called
\textbf{decoding}\index{decoding}. \todo{Need sources on these two
  terms.} In the circumstances of compression techniques, encoding and
decoding are also often referred to as
\textbf{compression}\index{compression} and
\textbf{decompression}\index{decompression}.

\subsection{Compression Ratio}
\label{sec:compression-ratio}

The compression ratio\index{compression ratio} is a very useful concept when talking about a
compression algorithms efficiency. The compression ratio of an
algorithm is calculated like this:

\begin{equation}
  \label{eq:compress-ratio}
  {\rm Compression\;Ratio} = \frac{\rm Compressed\;Size}{\rm Uncompressed\;Size}
\end{equation}

So the compression ratio basically tells how much data was
compressed. Of course, the lower compression ratio the better. But
there are ratios to take in heed, like performance and complexity of
the algorithm, but we will ignore these concepts in this book to book
to keep things simple. We will for the rest of the book specify the
compression ratio in percentages, $\%$\cite{Salomon:2004:DCC}.

Although we will be dealing mostly with bytes in these algorithms, they could
easily by generalized to compression or decompressing other integer
types likes \texttt{long} or \texttt{short}.

\FloatBarrier

\printbibliography[heading=subbibliography]

\end{refsection}
