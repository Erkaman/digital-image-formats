\begin{comment}
  \bibliography{project.bib}
\end{comment}

\chapter{Preliminaries}
\label{cha:preliminaries}

In this chapter, we will establish all the conventions that will be used
throughout his text and explain some basic terms.

\section{Glossary}
\label{sec:glossary}

\begin{description}
\item[binary number] A binary number is simply a number of base 2. For
  example, the number $139$ is represented by the binary number
  \bin{1000\ 1011}, since

  \begin{align*}
   & 1 \cdot 2^7 + 0 \cdot 2^6 + 0 \cdot 2^5 + 0 \cdot 2^4 + 1 \cdot 2^3
   + 0 \cdot 2^2 + 1 \cdot 2^1 + 1 \cdot 2^0 = \\
   & 2^7 + 2^3 + 2^1 + 2^0 = \\
   & 128 + 8 + 2 + 1 = 139
  \end{align*}

  \begin{Exercise}[label={n-to-bin}]
    Rewrite the following numbers on binary form:

    \begin{enumerate}[(a)]
    \item 4
    \item 50
    \item 500
    \end{enumerate}

  \end{Exercise}

  \begin{Exercise}[label={bin-to-n}]
    Convert the following binary numbers to ordinary numbers of base
    10:

    \begin{enumerate}[(a)]
    \item \bin{1111}
    \item \bin{1000\ 0000}
    \item \bin{1111\ 1110}
    \end{enumerate}

  \end{Exercise}

\item[bit] The separate digits of a binary numbers are often referred
  to as bits. An n-bit number is simply a number with $n$ binary digits.

  \begin{Exercise}[label={n-bits-max-val}]
    What is the maximum value of a n-bit number?  How many different
    possible values can a n-bit number have?
  \end{Exercise}

\item[toggled bit] If we say that a bit is \textit{toggled}, we mean that
  the bit has a value of $1$.

\item[cleared bit] And controversially, if a bit has the value
  $0$, then we say that the bit is \textit{cleared}.

\item[bit counting order] The bit that has the lowest value is
  referred to as bit 0\footnote{This convention is mainly used because
    us programmers like to start counting from $0$ rather than $1$}.
  In the binary number \bin{0100}, bit 0 is cleared( it has the value
  $0$), bit 2 is toggled (it has the value $1$) and the last bit 3 is
  cleared.

\item[lowest bit] Bit 0 will also commonly be referred to as the
  lowest bit.

\item[highest bit] Bit $n-1$ of a n-bit number. In order words, the
  last bit of a binary number. In the number \bin{100} the highest bit
  toggled and the 2 lowest bits are cleared.

\item[most significant bit] Alternative name for the highest bit.

\item[least significant bit] Alternative name for the lowest bit.

  \begin{Exercise}[label={bits-value-order}]
    What are the values of bits 1, 3 and the most significant bit in the
    number \bin{01001}?

  \end{Exercise}

\item[Byte] A byte is simply a binary number with 8 digits/bits. The
  max value of a byte is $2^8 - 1 = 255$ and there are $2^8 = 256$
  different values that a byte can have. We will often be referring to
  bytes throughout the entire text, since, as we soon shall see, an
  image represented digitally is just a sequence of bytes.

\item[ASCII] A very common text encoding that will be used a lot in
  this text. \ascii is a \mbox{7-bit} encoding and therefore covers
  128 different characters. But \ascii values are for the sake of
  convenience always stored in 8-bit bytes. The entire \ascii table is
  given in table \ref{tab:ascii-table}\cite{rfc20}. Characters 33--126
  are printable characters. Characters 0--32 are on the other hand
  control characters. These are used to affect how the text is
  processed. \htab(9) is for an ordinary tab while {\acronymstyle
    sp}(32) represents space. However, many of these control
  characters are obsolete and are practically never used. The control
  codes that are actually used in modern files are \nul, \htab, \lf
  and \cret \cite{maini2007digital}.

  The usage of the two special codes \cret and \lf is something that
  we need to further discuss. They are used to start a new line in
  text. But representing newlines turns out to be a surprisingly
  complex issue. In Windows based operating systems, newlines are
  represented by a \crlf; that is, a carriage return, \cret, followed
  by a linefeed, \lf. Unix based operating systems, like Linux and Mac
  OS X, simply uses a line feed, \lf, to represent newlines. But for
  Mac OS version 9 and lower, \cret was
  used\cite{robbins:_common_newline,noria:_under_newlin_oreilly,editor:_end_line_story_rfc,tancig01:_apart_no_more_newline,corporation08:_creat_telep_applic_both_window_linux}.

  If a text file is to be used on only one operating system this will
  never pose a problem. But if the file is to be shared between
  computers running on different operating systems, it will. However,
  this is something that is almost never noticeable to the common
  user. It is really only of importance to programmers who want their
  software to flawlessly run on different operating systems.

    \begin{longtable}{lll}
      \caption{The ASCII table} \\
      \toprule
      Value & Code(Character) & Code Description \\
      \midrule
      \endfirsthead
      \caption[]{(continued)}\\
      \toprule
      Value & Code(Character) & Code Description \\
      \midrule
      \endhead
      \bottomrule
      \endfoot
      \bottomrule
      \endlastfoot
      0 & \nul & Null Character \\
      1 & \soh & Start of Heading \\
      2 & \stx & Start of Text \\
      3 & \etx & End of Text \\
      4 & \eot & End of Transmission \\
      5 & \enq & Enquiry \\
      6 & \ack & Acknowledge \\
      7 & \bel & Bell(makes a sound) \\
      8 & \bs & Backspace \\
      9 & \htab & Horizontal tab(ordinary tab) \\
      10 & \lf & Line Feed \\
      11 & \vt & Vertical tab \\
      12 & \ff & Form Feed \\
      13 & \cret & Carriage Return \\
      14 & \sout & Shift Out \\
      15 & \shiftin & Shift In \\
      16 & \dle & Data Link Escape \\
      17 & \dc1 & Device Control 1 \\
      18 & \dc2 & Device Control 2 \\
      19 & \dc3 & Device Control 3 \\
      20 & \dc4 & Device Control 4\\
      21 & \nak & Negative Acknowledge \\
      22 & \syn & Synchronous Idle \\
      23 & \etb & End of Transmission Block \\
      24 & \can & Cancel \\
      25 & \emed & End of Medium \\
      26 & \sub & Substitute \\
      27 & \esc & Escape \\
      28 & \fs & File Separator \\
      29 & \gs & Group Separator \\
      30 & \rs & Record Separator \\
      31 & \us & Unit Separator \\
      32 & {\acronymstyle sp} & Space \\
      33 & ! & Exclamation Point \\
      34 & " & Quotation Mark \\
      35 & \# & Number Sign \\
      36 & \$ & Dollar Sign \\
      37 & \% & Percentage Sign \\
      38 & \& & Ampersand \\
      39 & ' & Apostrophe \\
      40 & ( & Left Parenthesis \\
      41 & ) & Right Parenthesis \\
      42 & * & Asterisk \\
      43 & + & Plus Sign \\
      44 & , & Comma \\
      45 & - & Minus Sign, hyphen \\
      46 & . & Period , Dot \\
      47 & / & Forward Slash \\
      48 & 0 & 0 \\
      49 & 1 & 1 \\
      50 & 2 & 2 \\
      51 & 3 & 3 \\
      52 & 4 & 4 \\
      53 & 5 & 5 \\
      54 & 6 & 6 \\
      55 & 7 & 7 \\
      56 & 8 & 8 \\
      57 & 9 & 9 \\
      58 & : & Colon \\
      59 & ; & Semicolon  \\
      60 & < & Less-than Sign \\
      61 & = & Equals Sign \\
      62 & > & Greater-than Sign \\
      63 & ? & Question Mark \\
      64 & @ & At Sign \\
      65 & A & A \\
      66 & B & B \\
      67 & C & C \\
      68 & D & D \\
      69 & E & E \\
      70 & F & F \\
      71 & G & G \\
      72 & H & H \\
      73 & I & I \\
      74 & J & J \\
      75 & K & K \\
      76 & L & L \\
      77 & M & M \\
      78 & N & N \\
      79 & O & O \\
      80 & P & P \\
      81 & Q & Q \\
      82 & R & R \\
      83 & S & S \\
      84 & T & T \\
      85 & U & U \\
      86 & V & V \\
      87 & W & W \\
      88 & X & X \\
      89 & Y & Y \\
      90 & Z & Z \\

      91 & [ & Left Bracket \\
      92 & \ & Backward Slash \\
      93 & ] & Right Bracket \\
      94 & $\wedge$ & Caret  \\
      95 & \_ & Underscore  \\
      96 & \`{} & Grave Accent   \\


      97 & a & a \\
      98 & b & b \\
      99 & c & c \\
      100 & d & d \\
      101 & e & e \\
      102 & f & f \\
      103 & g & g \\
      104 & h & h \\
      105 & i & i \\
      106 & j & j \\
      107 & k & k \\
      108 & l & l \\
      109 & m & m \\
      110 & n & n \\
      111 & o & o \\
      112 & p & p \\
      113 & q & q \\
      114 & r & r \\
      115 & s & s \\
      116 & t & t \\
      117 & u & u \\
      118 & v & v \\
      119 & w & w \\
      120 & x & x \\
      121 & y & y \\
      122 & z & z \\

      123 & \{ & Left Bracket \\
      124 & | & Vertical Line \\
      125 & \} & Right Bracket \\
      126 & \textasciitilde & Tilde \\
      127 & \del & Delete

      \label{tab:ascii-table}

    \end{longtable}



  \begin{Exercise}[label={ascii-to-num}]
    Convert the following \ascii characters to their corresponding \ascii
    values:

    \begin{enumerate}[(a)]
    \item A
    \item n
    \item <
    \end{enumerate}
  \end{Exercise}

  \begin{Exercise}[label={num-to-ascii}]
    Convert the following \ascii values to their corresponding characters:
    \begin{enumerate}[(a)]
    \item $35$
    \item $122$
    \item $63$
    \end{enumerate}
  \end{Exercise}

  \begin{Exercise}[label={ascii-uppercase}]

    What is always true for the \ascii values of uppercase characters(A--Z)?

    (Hint: write them out binary)

  \end{Exercise}

  \begin{Exercise}[label={uppercase-conv}]

    How do you convert an uppercase \ascii value to its corresponding
    lower case value? How do you do the reverse transformation,
    lowercase to uppercase?

    (Hint: what number has to be added or subtracted?)

  \end{Exercise}

\item[Hexadecimal] We will also be using the hexadecimal numeral system
  in this text a lot. In the hexadecimal system the numbers 0--9 are
  given their usual values, while the letters A--F are assigned to the
  values 10-15, so that the hexadecimal number \hex{D3} has the value

  \begin{equation*}
   13 \cdot 16^1 + 3 \cdot 16^0 = 13 \cdot 16 + 3 = 208 + 3 = 211
  \end{equation*}

  \begin{Exercise}[label={hex-to-n}]
    Convert the following hexadecimal numbers to ordinary numbers of
    base 10:

    \begin{enumerate}[(a)]
    \item \hex{23}
    \item \hex{FF}
    \item \hex{AA}
    \end{enumerate}

  \end{Exercise}

  \begin{Exercise}[label={n-to-hex}]
    Convert the following numbers to hexadecimal:

    \begin{enumerate}[(a)]
    \item 3
    \item 46
    \item 189
    \end{enumerate}

  \end{Exercise}


\item[String] A string is simply a sequence of letters in some
  encoding. The most commonly used encoding in this text will be \ascii
  \cite{rfc20}.

\item[C string] String as they are represented in the C programming
  language. In this language, strings are always terminated by the
  \nul character\cite{kernighan1988c}. The \nullm(alternative spelling
  of \nul) character has, as familiar, a value of $0$. This means that
  the string ``eric'' will be represented by the sequence of bytes
  $101,114,105,99, 0$ in the C programming language. This is important
  to know, because in a lot of image formats strings are stored as C
  strings.

  Why the \nullm character at the end of the string even is necessary
  is for rather complex reasons that we will not treat in this text.

\item[File] We are going to be talking a lot about files in this text,
  so it is important that we as early as possible establish a strict
  definition for what a file is. A file is just a sequence of bytes. A
  perfectly valid file could for example consist of the numbers
  $101,114,105,99$. This a file that consists of the single string
  ``eric'', where the letters use \ascii encoding. However, if you
  opened this file in a text editor, say notepad, you would only see
  the letters ``eric'' and not the numbers that represent the
  letters. This is because a text editor is programmed to see all the
  bytes in a file as text. If you on the other hand opened this file
  in a hex viewer, you would see the file for what it truly is: a
  sequence of numbers\footnote{Do note that in a hex viewer these
    numbers are, as is implied by the name, shown as hexadecimal
    numbers}.

  But since a byte can only have 256 different values, the reader may
  wonder how larger numbers are stored in a file. This is simply done
  by combining bytes. Two bytes in a sequence becomes a 16-bit number
  with a maximum value of $2^{16} - 1$. In the same fashion even
  larger numbers can be stored.

\item[Offset] When we are talking about an offset we are referring to
  a position in a file. The offset is zero based. When we are talking
  about the number at offset 0 in the file $13,2,1$ we are talking
  about the number $13$. In the same file at the offset $2$, the
  number $1$ can be found.


\item[Render] When a program displays an image on the screen this
  process is known as \textit{rendering}. To render an image is to
  display an image on the screen.

\item[Display] Synonym for render.

\end{description}

\section{Pseudocode Conventions}

Instead of showing code examples in some random programming language,
we will be using pseudocode to explain the algorithms in this book. This
will keep things as general as possible, and not force the reader into
knowing a specific programming language before reading this text.

The pseudocode will be kept as traditional as possible, but we will
still need to establish several conventions for it, which is what
we are going to do for the rest of the chapter.

\subsection{Typographical Conventions}

\begin{description}
\item[keywords] will use a \textbf{bold} font.
\item[functions] will be signified by \textsc{small caps}.
\item[variables] can noticed by their $\var{cursive\ slant}$.
\end{description}

\subsection{Boolean Operators}
\label{sec:boolean-operators}

To signify the boolean operators, or logical operators as we will often
also refer them to, we will be using the following symbols:

\begin{description}
\item[$\NOT$] logical \textit{not}
\item[$\AND$] logical \textit{and}
\item[$\OR$] logical \textit{or}
\end{description}

Logical truth is represented by \True, and falseness is represented by
\False.

The truth tables for these operators are presented in the
tables,~\ref{tab:log-and-table},~\ref{tab:log-or-table} and
\ref{tab:log-not-table}.

\begin{table}
  \centering
  \begin{tabular}{|c|c|c|}
    \hline
    $\mathbf{\AND}$ & $\mathbf{\False}$ & $\mathbf{\True}$ \\ \hline
    $\mathbf{\False}$ & $\mathbf{\False}$ & $\mathbf{\False}$ \\ \hline
    $\mathbf{\True}$ & $\mathbf{\False}$ & $\mathbf{\True}$ \\ \hline
  \end{tabular}
  \caption{Logical and truth table}
  \label{tab:log-and-table}
\end{table}

\begin{table}
  \centering
  \begin{tabular}{|c|c|c|}
    \hline
    $\mathbf{\OR}$ & $\mathbf{\False}$ & $\mathbf{\True}$ \\ \hline
    $\mathbf{\False}$ & $\False$ & $\True$ \\ \hline
    $\mathbf{\True}$ & $\True$ & $\True$ \\ \hline
  \end{tabular}
  \caption{Logical or truth table}
  \label{tab:log-or-table}
\end{table}

\begin{table}
  \centering
  \begin{tabular}{|c|c|}
    \hline
    $\mathbf{p}$ & $\mathbf{\NOT p}$ \\ \hline
    $\True$ & $\False$ \\ \hline
    $\False$ & $\True$ \\ \hline
  \end{tabular}
  \caption{Logical not truth table}
  \label{tab:log-not-table}
\end{table}

\begin{Exercise}[label={boolean}]
  Are the following expressions \True or \False?

  \begin{enumerate}[(a)]
  \item $\NOT \True \AND \True$
  \item $\NOT \True \OR (\True \AND \NOT \False)$

  \item $(\True \AND \True) \AND (\True \AND \NOT(\False \OR \NOT \True))$

  \end{enumerate}
\end{Exercise}

\subsection{Bitwise Operators}
\label{sec:bitwise-operators}

\subsubsection{Notation}

The bitwise operators will also be used in this text. We will use
notation introduced in the C programming language\cite{kernighan1988c}
to represent them in pseudocode:

\begin{description}
\item[$\BitAnd$] Bitwise and
\item[$\BitOr$] Bitwise or
\item[$\BitXor$] Bitwise xor
\item[$\BitNeg$] Bitwise not
\item[$\ShiftLeft$] Left bit shift
\item[$\ShiftRight$] Right bit shift
\end{description}

Notice that we are using $\BitXor$ for representing bitwise
xor rather than the traditional C notation $\AND$. This is due to
the fact that we would otherwise confuse it with logical and, $\AND$.

What follows is a short introduction to the very simple bitwise
operators.

\subsubsection{Bitwise and, or, and xor}

Bitwise and is just like logical and, except for the fact that it
operates on the on the bit level. Let us for demonstrative consider
the result of the expression $22 \BitAnd 12$. Since bitwise and
operates on the bit level we first must convert the two numbers to
binary:\mbox{ $\bin{10110} \BitAnd \bin{01100}$}. Then the calculation is simply done
like this:

\begin{center}
  \begin{tabular}{lr}
    & 10110  \\
    $\BitAnd$ & 01100 \\
    \hline
    & 00100 \\
  \end{tabular}
\end{center}

So $22 \BitAnd 12 = \bin{00100} = 4$. So as you can see, the bitwise
operators do boolean logic on the bit level by considering a toggled
bit truth, and a cleared bit falsity

\begin{Exercise}[label={bitand}]

  \begin{enumerate}[(a)]
  \item $2 \BitAnd 1$
  \item $255 \BitAnd 23$
  \item $26 \BitAnd 12$
  \end{enumerate}
\end{Exercise}

Bitwise or is in the same way logical or on the bit level. Let us
perform the former calculation using bitwise or:

\begin{center}
  \begin{tabular}{lr}
    & 10110  \\
    $\BitOr$ & 01100 \\
    \hline
    & 11110 \\
  \end{tabular}
\end{center}

\begin{Exercise}[label={bitor}]
  \begin{enumerate}[(a)]
  \item $172 \BitOr 52$
  \item $3 \BitOr 3$
  \item $240 \BitOr 15$
  \end{enumerate}
\end{Exercise}

\begin{table}
  \centering
  \begin{tabular}{|c|c|c|}
    \hline
    $\mathbf{\BitXor}$ & $\mathbf{\False}$ & $\mathbf{\True}$ \\ \hline
    $\mathbf{\False}$ & $\mathbf{\False}$ & $\mathbf{\True}$ \\ \hline
    $\mathbf{\True}$ & $\mathbf{\True}$ & $\mathbf{\False}$ \\ \hline
  \end{tabular}
  \caption{Logical exclusive or truth table}
  \label{tab:log-exlusive-or-table}
\end{table}

Bitwise xor on the other hand, operates on bits by using logical
exclusive or. The truth table of logical exclusive or is given in
table \ref{tab:log-exlusive-or-table}. Using this table, we can easily
understand how bitwise xor works and calculate the value of the
expression $\bin{10110} \BitXor \bin{01100}$:

\begin{center}
  \begin{tabular}{lr}
    & 10110  \\
    $\BitXor$ & 01100 \\
    \hline
    & 11010 \\
  \end{tabular}
\end{center}

\begin{Exercise}[label={bitxor}]
  \begin{enumerate}[(a)]
  \item $10 \BitXor 10$
  \item $12 \BitXor 7$
  \item $48 \BitXor 16$
  \end{enumerate}
\end{Exercise}

\subsubsection{Bitwise not}

When dealing with bitwise not, it is important that we consider the
size of the numbers that we are performing the operation on. If for
example $b=10$ and the variable $b$ is of type byte, then it
\textit{must} be of length 8 bits: $b=\bin{0000\ 1010}$. What bitwise not
does, is that it inverts the number so that at all toggled bits get
cleared, and all cleared bits get toggled, so $\BitNeg b = \bin{1111\
0101}$.

Now you should see why it was important that we considered the size of
the number. Had the variable $b$ been of size 4 bits, then $b = 1010$
and then the end result of the operation $\BitNeg b$ would have been
$0101$ instead of $1111\ 0101$.

\begin{Exercise}[label={bitnot}]
  What are the values of the following expressions, if all the numbers
  are bytes?

  \begin{enumerate}[(a)]
  \item $\BitNeg 11$
  \item $(\BitNeg 4) \BitXor 4$
  \item $(\BitNeg b) \BitXor b$, for any byte $b$
  \item $(\BitNeg b) \BitOr b$, for any byte $b$
  \item $(\BitNeg b) \BitAnd b$, for any byte $b$
  \end{enumerate}

\end{Exercise}

\subsubsection{Bitwise shifting}

It is also in bitwise shifting important that we consider the size of
the numbers. Bitwise shifting is actually very simple: all the
operation $b \ShiftLeft n$ really does, is that it shifts the bit
pattern in the number $b$ $n$ steps to the left. For the 4-bit number
$0011$, this means that $0011 \ShiftLeft 2 = 1100$. But what would
have happened if the bit pattern was shifted 3 steps? Then one bit is
going fall of the bit boundary and disappear, so $0011 \ShiftLeft 3 =
1000$.

And bitwise right shifting works in pretty much the same way, expect
for the fact that the bit shifting are done to the right instead of
the left, so that for the 4-bit number $0110$: $0110 \ShiftRight 2 =
0001$.

\begin{Exercise}[label={bit-shiting}]
  Assuming that we are dealing with bytes, what is the value of the
  following expression:

  \begin{equation*}
    120 \BitAnd (3 \ShiftLeft 4)
  \end{equation*}

  (important!) What does this expression really do?

  Hint: what kind of value does it extract from the byte $120$?

\end{Exercise}

\begin{Exercise}[label={bit-equiv}]

  What is

  \begin{enumerate}[(a)]
  \item $1 \ShiftLeft 0$
  \item $1 \ShiftLeft 1$
  \item $1 \ShiftLeft 2$
  \item $3 \ShiftLeft 1$
  \item $3 \ShiftLeft 2$
  \item $3 \ShiftLeft 3$
  \end{enumerate}

  Can you express the operation $N \ShiftLeft S$, if the condition $S
  \geq 0$ holds, in terms of the arithmetic operators? Exponentiation
  counts as an arithmetic operator in this exercise.

\end{Exercise}


\subsection{Syntax}

In this section we will discuss the basic syntax of the pseudocode.

The start of a comment is indicated by the symbol \commentsymbol.

To assign the value $n$ to the variable $var$, the notation shown in
algorithm \ref{alg:assign} is used.

\begin{algorithm}[H]
  \caption{Syntax for assigning the value $n$ to the variable $var$}
  \label{alg:assign}
  \begin{algorithmic}[1]
    \Let{$\var{var}$}{$\var{n}$}
  \end{algorithmic}
\end{algorithm}

To store a sequence of values we will use arrays. If for example the
array $a$ contains the values the $3,1,2$ then to access the first
value of this array, $3$, the syntax $a[0]$ is used. In general, to
access the n:th value of an array you do $a[n-1]$, since the indexes
of arrays are zero-based.

Strings are also considered arrays. If the string ``eric'' was
stored in the variable \texttt{str}, then the letter ``e'' can be
accessed with the syntax $str[0]$

To to go through each value in the array $a$, the syntax demonstrated
in algorithm \ref{alg:for-each} is used.

\begin{algorithm}[H]
  \caption{The for each control structure}
  \label{alg:for-each}
  \begin{algorithmic}[1]
    \linecomment{Go through every value $v$ in the
      array $a$. The variable  $v$ is assigned every element in the
      array $a$ in order.}
    \ForEach{$\var{v}$}{$\var{a}$}
    \linecomment{Do something with $a$ here.}
    \EndForEach
  \end{algorithmic}
\end{algorithm}

In algorithm \ref{alg:repeat} the control structure repeat is
demonstrated.

\begin{algorithm}[H]
  \caption{The repeat control structure}
  \label{alg:repeat}
  \begin{algorithmic}[1]
    \Repeatn{$n$}
    \State $\var{actions}$ \Comment{$actions$ are repeated $n$ times.}
    \EndRepeatn
  \end{algorithmic}
\end{algorithm}

For functions, we will be using the traditional syntax;
\Call{Func}{$a,b,c$} means that we are calling the function
\textproc{Func} with the argument $a$,$b$ and $c$ and that the value
of this expression is the return value of the function. To return a
value from a function the \textbf{return} statement is used.

The function syntax is demonstrated in algorithm \ref{alg:euclid}. In
this function, Euclid's algorithm is used to calculate the greatest
common divisor of the two given numbers.
\cite{cormen2009introduction_to_algo,weisstein:_euclid_algor}.

\begin{algorithm}
  \caption{Euclid's algorithm}
  \label{alg:euclid}
  \begin{algorithmic}[1]
    \Procedure{Euclid}{$a,b$}
    \State $r\gets a\bmod b$
    \While{$r\not=0$}
    \State $a\gets b$
    \State $b\gets r$
    \State $r\gets a\bmod b$
    \EndWhile
    \State \textbf{return} $b$
    \EndProcedure
  \end{algorithmic}
\end{algorithm}

The for loop allows for iteration with an explicit counter variable
that is given in the for loop statement. In algorithm \ref{alg:forto}
the for to loop is used to count from $1$ to $9$. Note that it in
reality counts up to $10$, then stops.

\begin{algorithm}
  \caption{For to loop}\algohack{}
  \label{alg:forto}
  \begin{algorithmic}[1]
    \ForTo{$i$}{$1$}{$10$}
    \linecomment{Print the number 1.}
    \EndFor
  \end{algorithmic}
\end{algorithm}

\subsection{Functions}
\label{sec:pseudocode}

We will be dealing with files in many of these algorithms, so we will need
to introduce several functions for handling file operations.

\begin{description}[font=\normalfont]
\item[\textproc{ReadByte}] It is assumed from the beginning of the
  algorithm that a file has already been opened for reading. This
  function reads a byte from that file.

\item[\Call{WriteByte}{$\var{byte}$}] At the beginning of every algorithm,
  we also assume that there is a file opened for output. This function
  writes a byte to that file.

\item[\textproc{EndOfFileReached}] \True{} if the end the file we are
  reading from have been reached.

\item[\Call{Read}{$\var{size}$}] Just being able to read a byte may not
  always be enough. We may want to read a number of a specific size
  from a file. We will do that with this function. This functions
  reads a number of the size $size$ \textit{bytes}(not bits) from a
  file.

\item[\Call{Inc}{\ensuremath{\var{var},n}}] Increases the value of the
  variable $var$ by $n$.

\item[\Call{Dec}{\ensuremath{\var{var},n}}] Decrease the value of the
  variable $var$ by $n$.

\item[\Call{getbits}{$b, \var{start}, \var{end}$}] is used to extract a bit
  pattern in the given bit range from $b$. This function will be
  derived in exercise \ref{bitcount}

\end{description}

\begin{Exercise}[label={bitcount}]
  Using the bitwise operators, try to devise a function
  \textproc{bitcount} for counting the numbers of toggled bits in a
  byte. For \Call{bitcount}{$7$} it should return $3$ and for
  \Call{bitcount}{$1$} it should return $1$.

\end{Exercise}

\begin{Exercise}[label={getbits}]

  Make a function  \Call{getbits}{$b, \var{start}, \var{end}$} that extracts
  the bit pattern of a number $b$ from a given starting position to a
  given ending position. These positions are be zero-based. Example:
  \Call{getbits}{$80, 4, 6$}$=5$, because $80$ is represented by the
  binary number $0101\ 0000$ and from the positions 4 to 6 there is a
  bit pattern that has a value of $5$($101$), which is what this function is
  supposed to extract.

  It is especially important that you do this exercise, because the
  function \textproc{getbits} is going to be used \textit{a lot}
  throughout this text.

  Hint: The bitwise operators $\ShiftLeft, \ShiftRight, \BitAnd,
  \BitNeg$ and the arithmetic operators $+,-$ is all you really need to
  use to solve this problem.

\end{Exercise}

\answers{}

\begin{Answer}[ref={n-to-bin}]
  \begin{enumerate}[(a)]
  \item $100$
  \item $11\ 0010$
  \item $1\ 1111\ 0100$
  \end{enumerate}
\end{Answer}

\begin{Answer}[ref={bin-to-n}]
  \begin{enumerate}[(a)]
  \item $15$
  \item $128$
  \item $254$
  \end{enumerate}
\end{Answer}

\begin{Answer}[ref={n-bits-max-val}]
  The maximum value of a n-bit number is $2^n - 1$.

  Since every binary digit only has two possible values, an n-bit
  number has $2^n$ possible values.
\end{Answer}

\begin{Answer}[ref={bits-value-order}]
  $1$, $1$ and $0$. Or: toggled, toggled, cleared.

\end{Answer}

\begin{Answer}[ref={ascii-to-num}]

  \begin{enumerate}[(a)]
  \item $65$
  \item $110$
  \item $60$
  \end{enumerate}

\end{Answer}

\begin{Answer}[ref={num-to-ascii}]

  \begin{enumerate}[(a)]
  \item \#
  \item z
  \item ?
  \end{enumerate}

\end{Answer}

\begin{Answer}[ref={ascii-uppercase}]

  The 6:th bit is always toggled. This is because the lowest uppercase
  character, A, has the value 65 and the sixth bit has the value $64$.
\end{Answer}

\begin{Answer}[ref={uppercase-conv}]

  You convert an uppercase ASCII value to lowercase by adding $32$ to
  it, since $a - A = 97 - 65 = 32$. To do the reverse transformation,
  you subtract $32$ from the value.

\end{Answer}

\begin{Answer}[ref={hex-to-n}]

  \begin{enumerate}[(a)]
  \item $35$
  \item $255$
  \item $170$
  \end{enumerate}

\end{Answer}

\begin{Answer}[ref={n-to-hex}]

  \begin{enumerate}[(a)]
  \item \hex{03}
  \item \hex{2E}
  \item \hex{BD}
  \end{enumerate}

\end{Answer}

\begin{Answer}[ref={boolean}]

  \begin{enumerate}[(a)]
  \item \False
  \item \True
  \item \False
  \end{enumerate}

\end{Answer}

\begin{Answer}[ref={bitand}]
  \begin{enumerate}[(a)]
  \item $0$
  \item $23$
  \item $8$
  \end{enumerate}
\end{Answer}

\begin{Answer}[ref={bitor}]
  \begin{enumerate}[(a)]
  \item $52$
  \item $3$
  \item $255$
  \end{enumerate}
\end{Answer}

\begin{Answer}[ref={bitxor}]
  \begin{enumerate}[(a)]
  \item $0$
  \item $11$
  \item $32$
  \end{enumerate}
\end{Answer}

\begin{Answer}[ref={bitnot}]
  \begin{enumerate}[(a)]
  \item $244$
  \item $255$
  \item $255$
  \item $255$
  \item $0$
  \end{enumerate}
\end{Answer}

\begin{Answer}[ref={bit-shiting}]

  The value of the expression is $3$. What this expression does, is
  that it extracts the values of the 5:th and 6:th bits from the
  number $120$.

\end{Answer}

\begin{Answer}[ref={bit-equiv}]

  \begin{enumerate}[(a)]
  \item $1$
  \item $2$
  \item $4$
  \item $6$
  \item $12$
  \item $24$
  \end{enumerate}

  The operation $N \ShiftLeft S$ is equivalent to $N \cdot 2^S$.

\end{Answer}

\begin{Answer}[ref={bitcount}]

  \begin{algorithmic}[1]
    \Procedure{bitcount}{$n$}
    \Let{$\var{count}$}{$0$}
    \linecomment{Once the value $8$ is reached the for loop terminates.}
    \ForTo{$\var{shift}$}{$0$}{$8$}
    \If{$n \BitAnd (1 \ShiftLeft \var{shift})$}
    \State \Call{Inc}{$\var{count}$}
    \EndIf
    \EndFor
    \State \Return{$\var{count}$}
    \EndProcedure
  \end{algorithmic}

\end{Answer}

\begin{Answer}[ref={getbits}]

  \begin{algorithmic}[1]
    \Procedure{getbits}{$n$}
    \linecomment{Calculate the length of the bit pattern}
    \Let{$\var{len}$}{$\var{end} - \var{start} + 1$}
    \State \textbf{return} $(b \ShiftRight \var{start}) \BitAnd (\BitNeg
    (\BitNeg 0 \ShiftLeft \var{len}))$
    \EndProcedure
  \end{algorithmic}

  The answer is given above. We will now explain this function.

  If given the input \Call{getbits}{$80, 4, 6$} how may we calculate
  the value $5$ from this?

  $80$ is represented by the bit pattern $0101\ 0000$. First, we will
  right shift down the pattern $start=4$ steps, $0101\ 0000 \ShiftRight
  start$, resulting in the bit pattern $0000\ 0101$. Now all that
  remains to be done is that we need to figure out how construct the
  bit pattern $0000\ 0111$ from the input values. Once we have figured
  out how to make this pattern, we can calculate the proper result
  like this: $0000\ 0101 \BitAnd 0000\ 0111 = 101 = 5$

  The bit pattern we want to construct is $0000\ 0111$. We want a
  sequence of $3$ toggled bits from the lowest bit. We can trivially
  calculate this length like this: $end - start + 1 = 6 - 4 + 1 = 3$.
  $ + 1$ is necessary because the bit position are zero based.

  We now have the length of the pattern. Now we need to figure out how
  to construct it. The operation $\BitNeg 0$, assuming we are dealing
  with bytes, gets you the pattern $1111\ 1111$. Then, by shifting
  this pattern $len$ steps to the left, $3 \ShiftLeft 1111\ 1111$ we
  end up with the value $1111\ 1000$. Now, by simply using the bitwise
  not operation again, $\BitNeg (1111\ 1000)$, we end up with the
  desired pattern $0000\ 0111$. And now we can simply extract the bit
  pattern like this: $0000\ 0101 \BitAnd 0000\ 0111 = 101 = 5$.

\end{Answer}

