\begin{comment}
  \bibliography{project.bib}
\end{comment}

\chapter{Preliminaries}
\label{cha:preliminaries}

Before we are ready to talk about loading image files, we first need
know how images are represented in a computer.

\section{RGB}
\label{sec:rgb}

There several models for representing color in a computer, but the
most prominent and popular one is by no doubt \textbf{RGB} \index{RGB}

But to understand the RGB color model, we first need to understand
color to begin with. Color is light moving at different
wavelengths. Different colors have different wavelengths. In our eyes
there are cells for perceiving three different wavelengths of light:
red, blue and green. There can also be different amounts of red, blue
and green light absorbed by the cells and the colors and also be
mixed, creating new kinds of colors. \cite{neider93:_openg_progr_guide}

\begin{figure}[h]
  \centering
  \colorrow{255}{250}{250}{snow}
\colorrow{248}{248}{255}{ghost white}
\colorrow{248}{248}{255}{GhostWhite}
\colorrow{245}{245}{245}{white smoke}
\colorrow{245}{245}{245}{WhiteSmoke}
\colorrow{220}{220}{220}{gainsboro}
\colorrow{255}{250}{240}{floral white}
\colorrow{255}{250}{240}{FloralWhite}
\colorrow{253}{245}{230}{old lace}
\colorrow{253}{245}{230}{OldLace}
\colorrow{250}{240}{230}{linen}
\colorrow{250}{235}{215}{antique white}
\colorrow{250}{235}{215}{AntiqueWhite}
\colorrow{255}{239}{213}{papaya whip}
\colorrow{255}{239}{213}{PapayaWhip}
\colorrow{255}{235}{205}{blanched almond}
\colorrow{255}{235}{205}{BlanchedAlmond}
\colorrow{255}{228}{196}{bisque}
\colorrow{255}{218}{185}{peach puff}
\colorrow{255}{218}{185}{PeachPuff}
\colorrow{255}{222}{173}{navajo white}
\colorrow{255}{222}{173}{NavajoWhite}
\colorrow{255}{228}{181}{moccasin}
\colorrow{255}{248}{220}{cornsilk}
\colorrow{255}{255}{240}{ivory}
\colorrow{255}{250}{205}{lemon chiffon}
\colorrow{255}{250}{205}{LemonChiffon}
\colorrow{255}{245}{238}{seashell}
\colorrow{240}{255}{240}{honeydew}
\colorrow{245}{255}{250}{mint cream}
\colorrow{245}{255}{250}{MintCream}
\colorrow{240}{255}{255}{azure}
\colorrow{240}{248}{255}{alice blue}
\colorrow{240}{248}{255}{AliceBlue}
\colorrow{230}{230}{250}{lavender}
\colorrow{255}{240}{245}{lavender blush}
\colorrow{255}{240}{245}{LavenderBlush}
\colorrow{255}{228}{225}{misty rose}
\colorrow{255}{228}{225}{MistyRose}
\colorrow{255}{255}{255}{white}
\colorrow{0}{0}{0}{black}
\colorrow{47}{79}{79}{dark slate gray}
\colorrow{47}{79}{79}{DarkSlateGray}
\colorrow{47}{79}{79}{dark slate grey}
\colorrow{47}{79}{79}{DarkSlateGrey}
\colorrow{105}{105}{105}{dim gray}
\colorrow{105}{105}{105}{DimGray}
\colorrow{105}{105}{105}{dim grey}
\colorrow{105}{105}{105}{DimGrey}
\colorrow{112}{128}{144}{slate gray}
\colorrow{112}{128}{144}{SlateGray}
\colorrow{112}{128}{144}{slate grey}
\colorrow{112}{128}{144}{SlateGrey}
\colorrow{119}{136}{153}{light slate gray}
\colorrow{119}{136}{153}{LightSlateGray}
\colorrow{119}{136}{153}{light slate grey}
\colorrow{119}{136}{153}{LightSlateGrey}
\colorrow{190}{190}{190}{gray}
\colorrow{190}{190}{190}{grey}
\colorrow{211}{211}{211}{light grey}
\colorrow{211}{211}{211}{LightGrey}
\colorrow{211}{211}{211}{light gray}
\colorrow{211}{211}{211}{LightGray}
\colorrow{25}{25}{112}{midnight blue}
\colorrow{25}{25}{112}{MidnightBlue}
\colorrow{0}{0}{128}{navy}
\colorrow{0}{0}{128}{navy blue}
\colorrow{0}{0}{128}{NavyBlue}
\colorrow{100}{149}{237}{cornflower blue}
\colorrow{100}{149}{237}{CornflowerBlue}
\colorrow{72}{61}{139}{dark slate blue}
\colorrow{72}{61}{139}{DarkSlateBlue}
\colorrow{106}{90}{205}{slate blue}
\colorrow{106}{90}{205}{SlateBlue}
\colorrow{123}{104}{238}{medium slate blue}
\colorrow{123}{104}{238}{MediumSlateBlue}
\colorrow{132}{112}{255}{light slate blue}
\colorrow{132}{112}{255}{LightSlateBlue}
\colorrow{0}{0}{205}{medium blue}
\colorrow{0}{0}{205}{MediumBlue}
\colorrow{65}{105}{225}{royal blue}
\colorrow{65}{105}{225}{RoyalBlue}
\colorrow{0}{0}{255}{blue}
\colorrow{30}{144}{255}{dodger blue}
\colorrow{30}{144}{255}{DodgerBlue}
\colorrow{0}{191}{255}{deep sky blue}
\colorrow{0}{191}{255}{DeepSkyBlue}
\colorrow{135}{206}{235}{sky blue}
\colorrow{135}{206}{235}{SkyBlue}
\colorrow{135}{206}{250}{light sky blue}
\colorrow{135}{206}{250}{LightSkyBlue}
\colorrow{70}{130}{180}{steel blue}
\colorrow{70}{130}{180}{SteelBlue}
\colorrow{176}{196}{222}{light steel blue}
\colorrow{176}{196}{222}{LightSteelBlue}
\colorrow{173}{216}{230}{light blue}
\colorrow{173}{216}{230}{LightBlue}
\colorrow{176}{224}{230}{powder blue}
\colorrow{176}{224}{230}{PowderBlue}
\colorrow{175}{238}{238}{pale turquoise}
\colorrow{175}{238}{238}{PaleTurquoise}
\colorrow{0}{206}{209}{dark turquoise}
\colorrow{0}{206}{209}{DarkTurquoise}
\colorrow{72}{209}{204}{medium turquoise}
\colorrow{72}{209}{204}{MediumTurquoise}
\colorrow{64}{224}{208}{turquoise}
\colorrow{0}{255}{255}{cyan}
\colorrow{224}{255}{255}{light cyan}
\colorrow{224}{255}{255}{LightCyan}
\colorrow{95}{158}{160}{cadet blue}
\colorrow{95}{158}{160}{CadetBlue}
\colorrow{102}{205}{170}{medium aquamarine}
\colorrow{102}{205}{170}{MediumAquamarine}
\colorrow{127}{255}{212}{aquamarine}
\colorrow{0}{100}{0}{dark green}
\colorrow{0}{100}{0}{DarkGreen}
\colorrow{85}{107}{47}{dark olive green}
\colorrow{85}{107}{47}{DarkOliveGreen}
\colorrow{143}{188}{143}{dark sea green}
\colorrow{143}{188}{143}{DarkSeaGreen}
\colorrow{46}{139}{87}{sea green}
\colorrow{46}{139}{87}{SeaGreen}
\colorrow{60}{179}{113}{medium sea green}
\colorrow{60}{179}{113}{MediumSeaGreen}
\colorrow{32}{178}{170}{light sea green}
\colorrow{32}{178}{170}{LightSeaGreen}
\colorrow{152}{251}{152}{pale green}
\colorrow{152}{251}{152}{PaleGreen}
\colorrow{0}{255}{127}{spring green}
\colorrow{0}{255}{127}{SpringGreen}
\colorrow{124}{252}{0}{lawn green}
\colorrow{124}{252}{0}{LawnGreen}
\colorrow{0}{255}{0}{green}
\colorrow{127}{255}{0}{chartreuse}
\colorrow{0}{250}{154}{medium spring green}
\colorrow{0}{250}{154}{MediumSpringGreen}
\colorrow{173}{255}{47}{green yellow}
\colorrow{173}{255}{47}{GreenYellow}
\colorrow{50}{205}{50}{lime green}
\colorrow{50}{205}{50}{LimeGreen}
\colorrow{154}{205}{50}{yellow green}
\colorrow{154}{205}{50}{YellowGreen}
\colorrow{34}{139}{34}{forest green}
\colorrow{34}{139}{34}{ForestGreen}
\colorrow{107}{142}{35}{olive drab}
\colorrow{107}{142}{35}{OliveDrab}
\colorrow{189}{183}{107}{dark khaki}
\colorrow{189}{183}{107}{DarkKhaki}
\colorrow{240}{230}{140}{khaki}
\colorrow{238}{232}{170}{pale goldenrod}
\colorrow{238}{232}{170}{PaleGoldenrod}
\colorrow{250}{250}{210}{light goldenrod yellow}
\colorrow{250}{250}{210}{LightGoldenrodYellow}
\colorrow{255}{255}{224}{light yellow}
\colorrow{255}{255}{224}{LightYellow}
\colorrow{255}{255}{0}{yellow}
\colorrow{255}{215}{0}{ gold}
\colorrow{238}{221}{130}{light goldenrod}
\colorrow{238}{221}{130}{LightGoldenrod}
\colorrow{218}{165}{32}{goldenrod}
\colorrow{184}{134}{11}{dark goldenrod}
\colorrow{184}{134}{11}{DarkGoldenrod}
\colorrow{188}{143}{143}{rosy brown}
\colorrow{188}{143}{143}{RosyBrown}
\colorrow{205}{92}{92}{indian red}
\colorrow{205}{92}{92}{IndianRed}
\colorrow{139}{69}{19}{saddle brown}
\colorrow{139}{69}{19}{SaddleBrown}
\colorrow{160}{82}{45}{sienna}
\colorrow{205}{133}{63}{peru}
\colorrow{222}{184}{135}{burlywood}
\colorrow{245}{245}{220}{beige}
\colorrow{245}{222}{179}{wheat}
\colorrow{244}{164}{96}{sandy brown}
\colorrow{244}{164}{96}{SandyBrown}
\colorrow{210}{180}{140}{tan}
\colorrow{210}{105}{30}{chocolate}
\colorrow{178}{34}{34}{firebrick}
\colorrow{165}{42}{42}{brown}
\colorrow{233}{150}{122}{dark salmon}
\colorrow{233}{150}{122}{DarkSalmon}
\colorrow{250}{128}{114}{salmon}
\colorrow{255}{160}{122}{light salmon}
\colorrow{255}{160}{122}{LightSalmon}
\colorrow{255}{165}{0}{orange}
\colorrow{255}{140}{0}{dark orange}
\colorrow{255}{140}{0}{DarkOrange}
\colorrow{255}{127}{80}{coral}
\colorrow{240}{128}{128}{light coral}
\colorrow{240}{128}{128}{LightCoral}
\colorrow{255}{99}{71}{tomato}
\colorrow{255}{69}{0}{orange red}
\colorrow{255}{69}{0}{OrangeRed}
\colorrow{255}{0}{0}{red}
\colorrow{255}{105}{180}{hot pink}
\colorrow{255}{105}{180}{HotPink}
\colorrow{255}{20}{147}{deep pink}
\colorrow{255}{20}{147}{DeepPink}
\colorrow{255}{192}{203}{pink}
\colorrow{255}{182}{193}{light pink}
\colorrow{255}{182}{193}{LightPink}
\colorrow{219}{112}{147}{pale violet red}
\colorrow{219}{112}{147}{PaleVioletRed}
\colorrow{176}{48}{96}{maroon}
\colorrow{199}{21}{133}{medium violet red}
\colorrow{199}{21}{133}{MediumVioletRed}
\colorrow{208}{32}{144}{violet red}
\colorrow{208}{32}{144}{VioletRed}
\colorrow{255}{0}{255}{magenta}
\colorrow{238}{130}{238}{violet}
\colorrow{221}{160}{221}{plum}
\colorrow{218}{112}{214}{orchid}
\colorrow{186}{85}{211}{medium orchid}
\colorrow{186}{85}{211}{MediumOrchid}
\colorrow{153}{50}{204}{dark orchid}
\colorrow{153}{50}{204}{DarkOrchid}
\colorrow{148}{0}{211}{dark violet}
\colorrow{148}{0}{211}{DarkViolet}
\colorrow{138}{43}{226}{blue violet}
\colorrow{138}{43}{226}{BlueViolet}
\colorrow{160}{32}{240}{purple}
\colorrow{147}{112}{219}{medium purple}
\colorrow{147}{112}{219}{MediumPurple}
\colorrow{216}{191}{216}{thistle}
\colorrow{255}{250}{250}{snow1}
\colorrow{238}{233}{233}{snow2}
\colorrow{205}{201}{201}{snow3}
\colorrow{139}{137}{137}{snow4}
\colorrow{255}{245}{238}{seashell1}
\colorrow{238}{229}{222}{seashell2}
\colorrow{205}{197}{191}{seashell3}
\colorrow{139}{134}{130}{seashell4}
\colorrow{255}{239}{219}{AntiqueWhite1}
\colorrow{238}{223}{204}{AntiqueWhite2}
\colorrow{205}{192}{176}{AntiqueWhite3}
\colorrow{139}{131}{120}{AntiqueWhite4}
\colorrow{255}{228}{196}{bisque1}
\colorrow{238}{213}{183}{bisque2}
\colorrow{205}{183}{158}{bisque3}
\colorrow{139}{125}{107}{bisque4}
\colorrow{255}{218}{185}{PeachPuff1}
\colorrow{238}{203}{173}{PeachPuff2}
\colorrow{205}{175}{149}{PeachPuff3}
\colorrow{139}{119}{101}{PeachPuff4}
\colorrow{255}{222}{173}{NavajoWhite1}
\colorrow{238}{207}{161}{NavajoWhite2}
\colorrow{205}{179}{139}{NavajoWhite3}
\colorrow{139}{121}{94}{NavajoWhite4}
\colorrow{255}{250}{205}{LemonChiffon1}
\colorrow{238}{233}{191}{LemonChiffon2}
\colorrow{205}{201}{165}{LemonChiffon3}
\colorrow{139}{137}{112}{LemonChiffon4}
\colorrow{255}{248}{220}{cornsilk1}
\colorrow{238}{232}{205}{cornsilk2}
\colorrow{205}{200}{177}{cornsilk3}
\colorrow{139}{136}{120}{cornsilk4}
\colorrow{255}{255}{240}{ivory1}
\colorrow{238}{238}{224}{ivory2}
\colorrow{205}{205}{193}{ivory3}
\colorrow{139}{139}{131}{ivory4}
\colorrow{240}{255}{240}{honeydew1}
\colorrow{224}{238}{224}{honeydew2}
\colorrow{193}{205}{193}{honeydew3}
\colorrow{131}{139}{131}{honeydew4}
\colorrow{255}{240}{245}{LavenderBlush1}
\colorrow{238}{224}{229}{LavenderBlush2}
\colorrow{205}{193}{197}{LavenderBlush3}
\colorrow{139}{131}{134}{LavenderBlush4}
\colorrow{255}{228}{225}{MistyRose1}
\colorrow{238}{213}{210}{MistyRose2}
\colorrow{205}{183}{181}{MistyRose3}
\colorrow{139}{125}{123}{MistyRose4}
\colorrow{240}{255}{255}{azure1}
\colorrow{224}{238}{238}{azure2}
\colorrow{193}{205}{205}{azure3}
\colorrow{131}{139}{139}{azure4}
\colorrow{131}{111}{255}{SlateBlue1}
\colorrow{122}{103}{238}{SlateBlue2}
\colorrow{105}{89}{205}{SlateBlue3}
\colorrow{71}{60}{139}{SlateBlue4}
\colorrow{72}{118}{255}{RoyalBlue1}
\colorrow{67}{110}{238}{RoyalBlue2}
\colorrow{58}{95}{205}{RoyalBlue3}
\colorrow{39}{64}{139}{RoyalBlue4}
\colorrow{0}{0}{255}{blue1}
\colorrow{0}{0}{238}{blue2}
\colorrow{0}{0}{205}{blue3}
\colorrow{0}{0}{139}{blue4}
\colorrow{30}{144}{255}{DodgerBlue1}
\colorrow{28}{134}{238}{DodgerBlue2}
\colorrow{24}{116}{205}{DodgerBlue3}
\colorrow{16}{78}{139}{DodgerBlue4}
\colorrow{99}{184}{255}{SteelBlue1}
\colorrow{92}{172}{238}{SteelBlue2}
\colorrow{79}{148}{205}{SteelBlue3}
\colorrow{54}{100}{139}{SteelBlue4}
\colorrow{0}{191}{255}{DeepSkyBlue1}
\colorrow{0}{178}{238}{DeepSkyBlue2}
\colorrow{0}{154}{205}{DeepSkyBlue3}
\colorrow{0}{104}{139}{DeepSkyBlue4}
\colorrow{135}{206}{255}{SkyBlue1}
\colorrow{126}{192}{238}{SkyBlue2}
\colorrow{108}{166}{205}{SkyBlue3}
\colorrow{74}{112}{139}{SkyBlue4}
\colorrow{176}{226}{255}{LightSkyBlue1}
\colorrow{164}{211}{238}{LightSkyBlue2}
\colorrow{141}{182}{205}{LightSkyBlue3}
\colorrow{96}{123}{139}{LightSkyBlue4}
\colorrow{198}{226}{255}{SlateGray1}
\colorrow{185}{211}{238}{SlateGray2}
\colorrow{159}{182}{205}{SlateGray3}
\colorrow{108}{123}{139}{SlateGray4}
\colorrow{202}{225}{255}{LightSteelBlue1}
\colorrow{188}{210}{238}{LightSteelBlue2}
\colorrow{162}{181}{205}{LightSteelBlue3}
\colorrow{110}{123}{139}{LightSteelBlue4}
\colorrow{191}{239}{255}{LightBlue1}
\colorrow{178}{223}{238}{LightBlue2}
\colorrow{154}{192}{205}{LightBlue3}
\colorrow{104}{131}{139}{LightBlue4}
\colorrow{224}{255}{255}{LightCyan1}
\colorrow{209}{238}{238}{LightCyan2}
\colorrow{180}{205}{205}{LightCyan3}
\colorrow{122}{139}{139}{LightCyan4}
\colorrow{187}{255}{255}{PaleTurquoise1}
\colorrow{174}{238}{238}{PaleTurquoise2}
\colorrow{150}{205}{205}{PaleTurquoise3}
\colorrow{102}{139}{139}{PaleTurquoise4}
\colorrow{152}{245}{255}{CadetBlue1}
\colorrow{142}{229}{238}{CadetBlue2}
\colorrow{122}{197}{205}{CadetBlue3}
\colorrow{83}{134}{139}{CadetBlue4}
\colorrow{0}{245}{255}{turquoise1}
\colorrow{0}{229}{238}{turquoise2}
\colorrow{0}{197}{205}{turquoise3}
\colorrow{0}{134}{139}{turquoise4}
\colorrow{0}{255}{255}{cyan1}
\colorrow{0}{238}{238}{cyan2}
\colorrow{0}{205}{205}{cyan3}
\colorrow{0}{139}{139}{cyan4}
\colorrow{151}{255}{255}{DarkSlateGray1}
\colorrow{141}{238}{238}{DarkSlateGray2}
\colorrow{121}{205}{205}{DarkSlateGray3}
\colorrow{82}{139}{139}{DarkSlateGray4}
\colorrow{127}{255}{212}{aquamarine1}
\colorrow{118}{238}{198}{aquamarine2}
\colorrow{102}{205}{170}{aquamarine3}
\colorrow{69}{139}{116}{aquamarine4}
\colorrow{193}{255}{193}{DarkSeaGreen1}
\colorrow{180}{238}{180}{DarkSeaGreen2}
\colorrow{155}{205}{155}{DarkSeaGreen3}
\colorrow{105}{139}{105}{DarkSeaGreen4}
\colorrow{84}{255}{159}{SeaGreen1}
\colorrow{78}{238}{148}{SeaGreen2}
\colorrow{67}{205}{128}{SeaGreen3}
\colorrow{46}{139}{87}{SeaGreen4}
\colorrow{154}{255}{154}{PaleGreen1}
\colorrow{144}{238}{144}{PaleGreen2}
\colorrow{124}{205}{124}{PaleGreen3}
\colorrow{84}{139}{84}{PaleGreen4}
\colorrow{0}{255}{127}{SpringGreen1}
\colorrow{0}{238}{118}{SpringGreen2}
\colorrow{0}{205}{102}{SpringGreen3}
\colorrow{0}{139}{69}{SpringGreen4}
\colorrow{0}{255}{0}{green1}
\colorrow{0}{238}{0}{green2}
\colorrow{0}{205}{0}{green3}
\colorrow{0}{139}{0}{green4}
\colorrow{127}{255}{0}{chartreuse1}
\colorrow{118}{238}{0}{chartreuse2}
\colorrow{102}{205}{0}{chartreuse3}
\colorrow{69}{139}{0}{chartreuse4}
\colorrow{192}{255}{62}{OliveDrab1}
\colorrow{179}{238}{58}{OliveDrab2}
\colorrow{154}{205}{50}{OliveDrab3}
\colorrow{105}{139}{34}{OliveDrab4}
\colorrow{202}{255}{112}{DarkOliveGreen1}
\colorrow{188}{238}{104}{DarkOliveGreen2}
\colorrow{162}{205}{90}{DarkOliveGreen3}
\colorrow{110}{139}{61}{DarkOliveGreen4}
\colorrow{255}{246}{143}{khaki1}
\colorrow{238}{230}{133}{khaki2}
\colorrow{205}{198}{115}{khaki3}
\colorrow{139}{134}{78}{khaki4}
\colorrow{255}{236}{139}{LightGoldenrod1}
\colorrow{238}{220}{130}{LightGoldenrod2}
\colorrow{205}{190}{112}{LightGoldenrod3}
\colorrow{139}{129}{76}{LightGoldenrod4}
\colorrow{255}{255}{224}{LightYellow1}
\colorrow{238}{238}{209}{LightYellow2}
\colorrow{205}{205}{180}{LightYellow3}
\colorrow{139}{139}{122}{LightYellow4}
\colorrow{255}{255}{0}{yellow1}
\colorrow{238}{238}{0}{yellow2}
\colorrow{205}{205}{0}{yellow3}
\colorrow{139}{139}{0}{yellow4}
\colorrow{255}{215}{0}{gold1}
\colorrow{238}{201}{0}{gold2}
\colorrow{205}{173}{0}{gold3}
\colorrow{139}{117}{0}{gold4}
\colorrow{255}{193}{37}{goldenrod1}
\colorrow{238}{180}{34}{goldenrod2}
\colorrow{205}{155}{29}{goldenrod3}
\colorrow{139}{105}{20}{goldenrod4}
\colorrow{255}{185}{15}{DarkGoldenrod1}
\colorrow{238}{173}{14}{DarkGoldenrod2}
\colorrow{205}{149}{12}{DarkGoldenrod3}
\colorrow{139}{101}{8}{DarkGoldenrod4}
\colorrow{255}{193}{193}{RosyBrown1}
\colorrow{238}{180}{180}{RosyBrown2}
\colorrow{205}{155}{155}{RosyBrown3}
\colorrow{139}{105}{105}{RosyBrown4}
\colorrow{255}{106}{106}{IndianRed1}
\colorrow{238}{99}{99}{IndianRed2}
\colorrow{205}{85}{85}{IndianRed3}
\colorrow{139}{58}{58}{IndianRed4}
\colorrow{255}{130}{71}{sienna1}
\colorrow{238}{121}{66}{sienna2}
\colorrow{205}{104}{57}{sienna3}
\colorrow{139}{71}{38}{sienna4}
\colorrow{255}{211}{155}{burlywood1}
\colorrow{238}{197}{145}{burlywood2}
\colorrow{205}{170}{125}{burlywood3}
\colorrow{139}{115}{85}{burlywood4}
\colorrow{255}{231}{186}{wheat1}
\colorrow{238}{216}{174}{wheat2}
\colorrow{205}{186}{150}{wheat3}
\colorrow{139}{126}{102}{wheat4}
\colorrow{255}{165}{79}{tan1}
\colorrow{238}{154}{73}{tan2}
\colorrow{205}{133}{63}{tan3}
\colorrow{139}{90}{43}{tan4}
\colorrow{255}{127}{36}{chocolate1}
\colorrow{238}{118}{33}{chocolate2}
\colorrow{205}{102}{29}{chocolate3}
\colorrow{139}{69}{19}{chocolate4}
\colorrow{255}{48}{48}{firebrick1}
\colorrow{238}{44}{44}{firebrick2}
\colorrow{205}{38}{38}{firebrick3}
\colorrow{139}{26}{26}{firebrick4}
\colorrow{255}{64}{64}{brown1}
\colorrow{238}{59}{59}{brown2}
\colorrow{205}{51}{51}{brown3}
\colorrow{139}{35}{35}{brown4}
\colorrow{255}{140}{105}{salmon1}
\colorrow{238}{130}{98}{salmon2}
\colorrow{205}{112}{84}{salmon3}
\colorrow{139}{76}{57}{salmon4}
\colorrow{255}{160}{122}{LightSalmon1}
\colorrow{238}{149}{114}{LightSalmon2}
\colorrow{205}{129}{98}{LightSalmon3}
\colorrow{139}{87}{66}{LightSalmon4}
\colorrow{255}{165}{0}{orange1}
\colorrow{238}{154}{0}{orange2}
\colorrow{205}{133}{0}{orange3}
\colorrow{139}{90}{0}{orange4}
\colorrow{255}{127}{0}{DarkOrange1}
\colorrow{238}{118}{0}{DarkOrange2}
\colorrow{205}{102}{0}{DarkOrange3}
\colorrow{139}{69}{0}{DarkOrange4}
\colorrow{255}{114}{86}{coral1}
\colorrow{238}{106}{80}{coral2}
\colorrow{205}{91}{69}{coral3}
\colorrow{139}{62}{47}{coral4}
\colorrow{255}{99}{71}{tomato1}
\colorrow{238}{92}{66}{tomato2}
\colorrow{205}{79}{57}{tomato3}
\colorrow{139}{54}{38}{tomato4}
\colorrow{255}{69}{0}{OrangeRed1}
\colorrow{238}{64}{0}{OrangeRed2}
\colorrow{205}{55}{0}{OrangeRed3}
\colorrow{139}{37}{0}{OrangeRed4}
\colorrow{255}{0}{0}{red1}
\colorrow{238}{0}{0}{red2}
\colorrow{205}{0}{0}{red3}
\colorrow{139}{0}{0}{red4}
\colorrow{255}{20}{147}{DeepPink1}
\colorrow{238}{18}{137}{DeepPink2}
\colorrow{205}{16}{118}{DeepPink3}
\colorrow{139}{10}{80}{DeepPink4}
\colorrow{255}{110}{180}{HotPink1}
\colorrow{238}{106}{167}{HotPink2}
\colorrow{205}{96}{144}{HotPink3}
\colorrow{139}{58}{98}{HotPink4}
\colorrow{255}{181}{197}{pink1}
\colorrow{238}{169}{184}{pink2}
\colorrow{205}{145}{158}{pink3}
\colorrow{139}{99}{108}{pink4}
\colorrow{255}{174}{185}{LightPink1}
\colorrow{238}{162}{173}{LightPink2}
\colorrow{205}{140}{149}{LightPink3}
\colorrow{139}{95}{101}{LightPink4}
\colorrow{255}{130}{171}{PaleVioletRed1}
\colorrow{238}{121}{159}{PaleVioletRed2}
\colorrow{205}{104}{137}{PaleVioletRed3}
\colorrow{139}{71}{93}{PaleVioletRed4}
\colorrow{255}{52}{179}{maroon1}
\colorrow{238}{48}{167}{maroon2}
\colorrow{205}{41}{144}{maroon3}
\colorrow{139}{28}{98}{maroon4}
\colorrow{255}{62}{150}{VioletRed1}
\colorrow{238}{58}{140}{VioletRed2}
\colorrow{205}{50}{120}{VioletRed3}
\colorrow{139}{34}{82}{VioletRed4}
\colorrow{255}{0}{255}{magenta1}
\colorrow{238}{0}{238}{magenta2}
\colorrow{205}{0}{205}{magenta3}
\colorrow{139}{0}{139}{magenta4}
\colorrow{255}{131}{250}{orchid1}
\colorrow{238}{122}{233}{orchid2}
\colorrow{205}{105}{201}{orchid3}
\colorrow{139}{71}{137}{orchid4}
\colorrow{255}{187}{255}{plum1}
\colorrow{238}{174}{238}{plum2}
\colorrow{205}{150}{205}{plum3}
\colorrow{139}{102}{139}{plum4}
\colorrow{224}{102}{255}{MediumOrchid1}
\colorrow{209}{95}{238}{MediumOrchid2}
\colorrow{180}{82}{205}{MediumOrchid3}
\colorrow{122}{55}{139}{MediumOrchid4}
\colorrow{191}{62}{255}{DarkOrchid1}
\colorrow{178}{58}{238}{DarkOrchid2}
\colorrow{154}{50}{205}{DarkOrchid3}
\colorrow{104}{34}{139}{DarkOrchid4}
\colorrow{155}{48}{255}{purple1}
\colorrow{145}{44}{238}{purple2}
\colorrow{125}{38}{205}{purple3}
\colorrow{85}{26}{139}{purple4}
\colorrow{171}{130}{255}{MediumPurple1}
\colorrow{159}{121}{238}{MediumPurple2}
\colorrow{137}{104}{205}{MediumPurple3}
\colorrow{93}{71}{139}{MediumPurple4}
\colorrow{255}{225}{255}{thistle1}
\colorrow{238}{210}{238}{thistle2}
\colorrow{205}{181}{205}{thistle3}
\colorrow{139}{123}{139}{thistle4}
\colorrow{0}{0}{0}{gray0}
\colorrow{0}{0}{0}{grey0}
\colorrow{3}{3}{3}{gray1}
\colorrow{3}{3}{3}{grey1}
\colorrow{5}{5}{5}{gray2}
\colorrow{5}{5}{5}{grey2}
\colorrow{8}{8}{8}{gray3}
\colorrow{8}{8}{8}{grey3}
\colorrow{10}{10}{10}{ gray4}
\colorrow{10}{10}{10}{ grey4}
\colorrow{13}{13}{13}{ gray5}
\colorrow{13}{13}{13}{ grey5}
\colorrow{15}{15}{15}{ gray6}
\colorrow{15}{15}{15}{ grey6}
\colorrow{18}{18}{18}{ gray7}
\colorrow{18}{18}{18}{ grey7}
\colorrow{20}{20}{20}{ gray8}
\colorrow{20}{20}{20}{ grey8}
\colorrow{23}{23}{23}{ gray9}
\colorrow{23}{23}{23}{ grey9}
\colorrow{26}{26}{26}{ gray10}
\colorrow{26}{26}{26}{ grey10}
\colorrow{28}{28}{28}{ gray11}
\colorrow{28}{28}{28}{ grey11}
\colorrow{31}{31}{31}{ gray12}
\colorrow{31}{31}{31}{ grey12}
\colorrow{33}{33}{33}{ gray13}
\colorrow{33}{33}{33}{ grey13}
\colorrow{36}{36}{36}{ gray14}
\colorrow{36}{36}{36}{ grey14}
\colorrow{38}{38}{38}{ gray15}
\colorrow{38}{38}{38}{ grey15}
\colorrow{41}{41}{41}{ gray16}
\colorrow{41}{41}{41}{ grey16}
\colorrow{43}{43}{43}{ gray17}
\colorrow{43}{43}{43}{ grey17}
\colorrow{46}{46}{46}{ gray18}
\colorrow{46}{46}{46}{ grey18}
\colorrow{48}{48}{48}{ gray19}
\colorrow{48}{48}{48}{ grey19}
\colorrow{51}{51}{51}{ gray20}
\colorrow{51}{51}{51}{ grey20}
\colorrow{54}{54}{54}{ gray21}
\colorrow{54}{54}{54}{ grey21}
\colorrow{56}{56}{56}{ gray22}
\colorrow{56}{56}{56}{ grey22}
\colorrow{59}{59}{59}{ gray23}
\colorrow{59}{59}{59}{ grey23}
\colorrow{61}{61}{61}{ gray24}
\colorrow{61}{61}{61}{ grey24}
\colorrow{64}{64}{64}{ gray25}
\colorrow{64}{64}{64}{ grey25}
\colorrow{66}{66}{66}{ gray26}
\colorrow{66}{66}{66}{ grey26}
\colorrow{69}{69}{69}{ gray27}
\colorrow{69}{69}{69}{ grey27}
\colorrow{71}{71}{71}{ gray28}
\colorrow{71}{71}{71}{ grey28}
\colorrow{74}{74}{74}{ gray29}
\colorrow{74}{74}{74}{ grey29}
\colorrow{77}{77}{77}{ gray30}
\colorrow{77}{77}{77}{ grey30}
\colorrow{79}{79}{79}{ gray31}
\colorrow{79}{79}{79}{ grey31}
\colorrow{82}{82}{82}{ gray32}
\colorrow{82}{82}{82}{ grey32}
\colorrow{84}{84}{84}{ gray33}
\colorrow{84}{84}{84}{ grey33}
\colorrow{87}{87}{87}{ gray34}
\colorrow{87}{87}{87}{ grey34}
\colorrow{89}{89}{89}{ gray35}
\colorrow{89}{89}{89}{ grey35}
\colorrow{92}{92}{92}{ gray36}
\colorrow{92}{92}{92}{ grey36}
\colorrow{94}{94}{94}{ gray37}
\colorrow{94}{94}{94}{ grey37}
\colorrow{97}{97}{97}{ gray38}
\colorrow{97}{97}{97}{ grey38}
\colorrow{99}{99}{99}{ gray39}
\colorrow{99}{99}{99}{ grey39}
\colorrow{102}{102}{102}{ gray40}
\colorrow{102}{102}{102}{ grey40}
\colorrow{105}{105}{105}{ gray41}
\colorrow{105}{105}{105}{ grey41}
\colorrow{107}{107}{107}{ gray42}
\colorrow{107}{107}{107}{ grey42}
\colorrow{110}{110}{110}{ gray43}
\colorrow{110}{110}{110}{ grey43}
\colorrow{112}{112}{112}{ gray44}
\colorrow{112}{112}{112}{ grey44}
\colorrow{115}{115}{115}{ gray45}
\colorrow{115}{115}{115}{ grey45}
\colorrow{117}{117}{117}{ gray46}
\colorrow{117}{117}{117}{ grey46}
\colorrow{120}{120}{120}{ gray47}
\colorrow{120}{120}{120}{ grey47}
\colorrow{122}{122}{122}{ gray48}
\colorrow{122}{122}{122}{ grey48}
\colorrow{125}{125}{125}{ gray49}
\colorrow{125}{125}{125}{ grey49}
\colorrow{127}{127}{127}{ gray50}
\colorrow{127}{127}{127}{ grey50}
\colorrow{130}{130}{130}{ gray51}
\colorrow{130}{130}{130}{ grey51}
\colorrow{133}{133}{133}{ gray52}
\colorrow{133}{133}{133}{ grey52}
\colorrow{135}{135}{135}{ gray53}
\colorrow{135}{135}{135}{ grey53}
\colorrow{138}{138}{138}{ gray54}
\colorrow{138}{138}{138}{ grey54}
\colorrow{140}{140}{140}{ gray55}
\colorrow{140}{140}{140}{ grey55}
\colorrow{143}{143}{143}{ gray56}
\colorrow{143}{143}{143}{ grey56}
\colorrow{145}{145}{145}{ gray57}
\colorrow{145}{145}{145}{ grey57}
\colorrow{148}{148}{148}{ gray58}
\colorrow{148}{148}{148}{ grey58}
\colorrow{150}{150}{150}{ gray59}
\colorrow{150}{150}{150}{ grey59}
\colorrow{153}{153}{153}{ gray60}
\colorrow{153}{153}{153}{ grey60}
\colorrow{156}{156}{156}{ gray61}
\colorrow{156}{156}{156}{ grey61}
\colorrow{158}{158}{158}{ gray62}
\colorrow{158}{158}{158}{ grey62}
\colorrow{161}{161}{161}{ gray63}
\colorrow{161}{161}{161}{ grey63}
\colorrow{163}{163}{163}{ gray64}
\colorrow{163}{163}{163}{ grey64}
\colorrow{166}{166}{166}{ gray65}
\colorrow{166}{166}{166}{ grey65}
\colorrow{168}{168}{168}{ gray66}
\colorrow{168}{168}{168}{ grey66}
\colorrow{171}{171}{171}{ gray67}
\colorrow{171}{171}{171}{ grey67}
\colorrow{173}{173}{173}{ gray68}
\colorrow{173}{173}{173}{ grey68}
\colorrow{176}{176}{176}{ gray69}
\colorrow{176}{176}{176}{ grey69}
\colorrow{179}{179}{179}{ gray70}
\colorrow{179}{179}{179}{ grey70}
\colorrow{181}{181}{181}{ gray71}
\colorrow{181}{181}{181}{ grey71}
\colorrow{184}{184}{184}{ gray72}
\colorrow{184}{184}{184}{ grey72}
\colorrow{186}{186}{186}{ gray73}
\colorrow{186}{186}{186}{ grey73}
\colorrow{189}{189}{189}{ gray74}
\colorrow{189}{189}{189}{ grey74}
\colorrow{191}{191}{191}{ gray75}
\colorrow{191}{191}{191}{ grey75}
\colorrow{194}{194}{194}{ gray76}
\colorrow{194}{194}{194}{ grey76}
\colorrow{196}{196}{196}{ gray77}
\colorrow{196}{196}{196}{ grey77}
\colorrow{199}{199}{199}{ gray78}
\colorrow{199}{199}{199}{ grey78}
\colorrow{201}{201}{201}{ gray79}
\colorrow{201}{201}{201}{ grey79}
\colorrow{204}{204}{204}{ gray80}
\colorrow{204}{204}{204}{ grey80}
\colorrow{207}{207}{207}{ gray81}
\colorrow{207}{207}{207}{ grey81}
\colorrow{209}{209}{209}{ gray82}
\colorrow{209}{209}{209}{ grey82}
\colorrow{212}{212}{212}{ gray83}
\colorrow{212}{212}{212}{ grey83}
\colorrow{214}{214}{214}{ gray84}
\colorrow{214}{214}{214}{ grey84}
\colorrow{217}{217}{217}{ gray85}
\colorrow{217}{217}{217}{ grey85}
\colorrow{219}{219}{219}{ gray86}
\colorrow{219}{219}{219}{ grey86}
\colorrow{222}{222}{222}{ gray87}
\colorrow{222}{222}{222}{ grey87}
\colorrow{224}{224}{224}{ gray88}
\colorrow{224}{224}{224}{ grey88}
\colorrow{227}{227}{227}{ gray89}
\colorrow{227}{227}{227}{ grey89}
\colorrow{229}{229}{229}{ gray90}
\colorrow{229}{229}{229}{ grey90}
\colorrow{232}{232}{232}{ gray91}
\colorrow{232}{232}{232}{ grey91}
\colorrow{235}{235}{235}{ gray92}
\colorrow{235}{235}{235}{ grey92}
\colorrow{237}{237}{237}{ gray93}
\colorrow{237}{237}{237}{ grey93}
\colorrow{240}{240}{240}{ gray94}
\colorrow{240}{240}{240}{ grey94}
\colorrow{242}{242}{242}{ gray95}
\colorrow{242}{242}{242}{ grey95}
\colorrow{245}{245}{245}{ gray96}
\colorrow{245}{245}{245}{ grey96}
\colorrow{247}{247}{247}{ gray97}
\colorrow{247}{247}{247}{ grey97}
\colorrow{250}{250}{250}{ gray98}
\colorrow{250}{250}{250}{ grey98}
\colorrow{252}{252}{252}{ gray99}
\colorrow{252}{252}{252}{ grey99}
\colorrow{255}{255}{255}{ gray100}
\colorrow{255}{255}{255}{ grey100}
\colorrow{169}{169}{169}{dark grey}
\colorrow{169}{169}{169}{DarkGrey}
\colorrow{169}{169}{169}{dark gray}
\colorrow{169}{169}{169}{DarkGray}
\colorrow{0}{0}{139}{dark blue}
\colorrow{0}{0}{139}{DarkBlue}
\colorrow{0}{139}{139}{dark cyan}
\colorrow{0}{139}{139}{DarkCyan}
\colorrow{139}{0}{139}{dark magenta}
\colorrow{139}{0}{139}{DarkMagenta}
\colorrow{139}{0}{0}{dark red}
\colorrow{139}{0}{0}{DarkRed}
\colorrow{144}{238}{144}{light green}
\colorrow{144}{238}{144}{LightGreen}

  \caption{RGB color model}
  \label{fig:rgb}
\end{figure}

So the RGB color model relies on the fact that all colors are a
mixture of red,blue and green, as it is demonstrated in figure
\ref{fig:rgb}. White is a mixture of red, blue and green, yellow is
mixture of red and green, and so on. The mixing of different amounts
of colors can conveniently enough be specified in numbers very
easily. Which brings us to:

\section{Color depth}
\label{sec:bit-depth}

\section{Digtal image}
\label{sec:digtal-image}

% Images can most easily be thought of as an

\printbibliography[heading=subbibliography]
