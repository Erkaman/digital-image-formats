\begin{comment}
  \bibliography{project.bib}
\end{comment}

\chapter{IM -- The General Image Format}
\label{cha:im}

\section{Introduction}

\begin{figure}
  \centering
  \inputtikz{im_format.tex}
  \caption{IM image format}
  \label{fig:im}
\end{figure}

In this chapter, we are to explore an imaginary, made-up, image format
called \textit{IM}\index{IM}. By studying this extremely simple image
format, we will get a good \textit{general} view of how image formats
are structured.

But before we dive in, let me clarify one thing for you: all data that
is stored in a computer is just a sequence of numbers. How this
sequence of numbers is to be parsed is defined by it's \textit{file
  format} \index{file format}. The entirety of a file format is
specified in its \textit{specification}\index{specification}.

\section{Specification for the Imaginary Format}

Observe figure \ref{fig:im}. It demonstrates a picture of a pretty
flower in the imaginary file format IM\index{IM}. It nicely
demonstrates the elements common to all image formats. Let's go
through them one by one:

\subsection{Image Header}

The first numbers of an image is always its image header\index{image
  header}. How many numbers it consists of is specific to the image
format in question. The image header gives general information about
an image that greatly helps the loading process. It usually includes
information like the width, height and color depth of the image.

\subsubsection{Magic Numbers}

Let's look at our example image. The first two numbers are the two
letters IM. These numbers are known as magic numbers \index{magic
  numbers}. You see, at the beginning of every image file, there a
couple bytes called magic numbers that are common to all images of
that type. They are sort like file name extensions, meaning that they
help you identify the type of the file you are currently reading.

\subsubsection{Measurements}

And following the magic number is the width and the height of the
image. In practically every image format this information is
included.

\subsection{Color Palette}

Color palettes \index{color palette} are also very common, but not
all images out in the wild use them, but as good all image formats support
them, hence why they are included in this imaginary format. A color
palette is basically an array or list of colors that is to be used
by the image.

\subsection{The Color Data}

Following the header and the optional color palette is the actual
color data. Now let us talk about how the colors are stored as an
image.

An image is simply a grid of colors. As demonstrated in figure
\ref{fig:im}. As you can
see, an image really only consists of a lot of tiny,tiny little
colored squares. We usually refer to these squares as pixels
\index{pixel}. (todo{cite sources})

So as we can see from our sample image format, IM, the flower image
is just a couple of colored pixels. But in this case a color palette
is used. This means the image indexes the color palette like an
array or list. So instead of the image containing its actual color,
it in this case just contains a bunch of index numbers.

If a color palette were not to be used in this case, the image would
just contain a grid of raw colors.

The color stored in an image and/or palette uses a certain bit depth,
as we discussed in chapter \ref{cha:color}. This information is
\textit{always} included in the image format specification. I didn't
specify a color depth for this format to keep the example simple and
understandable.

\section{Done\dots?}

But it's not really that simple. A majority of all image format have
had some sort of compression applied to their color data. And
performing the decompression of that image data is usually the
most difficult part. So most time of this book will be dedicateed on
understanding how these compression algorithms work.

\todo{fix the outro}

