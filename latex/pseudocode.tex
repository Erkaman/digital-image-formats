\chapter{Pseudocode Conventions}
\label{cha:pseudocode-convent}

Instead of showing code examples in some random programming language,
we'll be using pseudocode to explain the algorithms in this book. This
will keep things as general as possible, and not force the reader into
knowing a specific programming language before reading this book.

But on the other hand, sample code in the programming language \C{} is
provided. It can be found in the \path|/code| directory. But the code
that we will discuss and analyze in the book is the pseudocode and not
the \C{}-code.

The pseudocode will be kept as traditional as possible, but we'll
still need to establish several conventions for it, which is what
we're going to do for the rest of the chapter.

\section{Typographical Conventions}

This is how we'll use fonts in the pseudocode:

\textbf{keywords}, \textproc{functions}, and $variables$!

\section{Boolean Algebra}
\label{sec:boolean-operators}

For the boolean operators, we'll be using the following
symbols:

\begin{description}
\item[$\NOT$] logical \textit{not}
\item[$\AND$] logical \textit{and}
\item[$\OR$] logical \textit{or}
\end{description}

Boolean truth is represented by \True{}, and falseness is represented by
\False{}.

\section{Bitwise Operators}
\label{sec:bitwise-operators}

It is very rare that bitwise operators are ever used in
pseudocode. However, digital image och compression algorithms are
areas where they are almost mandatory to understand the algorithms. If
you are not familiar with bitwise operators, look up them and study
them on your own, as I assumed from the beginning of this book that such
knowledge should be obvious to you. We will be using the notation used
in \C{} to represent them in pseudocode:

\begin{description}
\item[$\BitNeg$] Bitwise \textit{NOT}
\item[$\BitAnd$] Bitwise \textit{AND}
\item[$\BitOr$] Bitwise \textit{OR}
\item[$\BitXor$] Bitwise \textit{XOR}
\item[$\ShiftLeft$] Left bit shift
\item[$\ShiftRight$] Right bit shift
\end{description}

Notice that we are using $\BitXor$ for representing bitwise
\textit{XOR} rather than the traditional \C{} notation. This is due to
the fact that we'd otherwise confuse it with logical and, $\AND$.

\section{Functions}
\label{sec:pseudocode}

We'll be dealing with files in many of these algorithms, so we'll need
to introduce several functions for file handling operations.

\begin{description}[font=\normalfont]
\item[\textproc{ReadByte}] It is assumed from the beginning of the
  algorithm that a file has already been opened for reading. This
  function reads a byte from that file.

\item[\Call{WriteByte}{$byte$}] At the beginning of every algorithm,
  we also assume that there is a file opened for output. This function
  writes a byte to that file.

\item[\textproc{EndOfFileReached}] \True{} if the end the file we're
  reading from have been reached.

\item[\Call{Read}{$size$}] Just being able to read a byte may not
  always be enough. We may want to read a number of a specific size
  from a file. We will do that with this function. This functions
  reads a number of the size $size$ \textit{bytes}(not bits) from a
  file.

\item[\Call{Seek}{$offset,origin$}] Move the from current position in
  the file offset bytes forward. The current position in the file is
  specified by the argument $origin$. If it is $START$, we move from the
  beginning of the file, if it is $END$ we move from the end of the
  file, and if it is $CURRENT$ we move from the current position in
  the file.

\end{description}

\section{Syntax}
\label{sec:other-conventions}

As a decently competent programmer, you should be familiar with most
of the control structures we'll be using. One control structure that
is probably unfamiliar to you is the custom control structure called
\algorithmicrepeat, which is demonstrated in algorithm \ref{alg:repeat}.

\todo{Cover all the control structures?}

\begin{algorithm}[H]
  \caption{The repeat control structure.}
  \label{alg:repeat}
  \begin{algorithmic}[1]
    \Repeatn{$n$}
      \State $actions\ldots$ \Comment{Repeats $actions\ldots n$ times}
    \EndRepeatn
  \end{algorithmic}
\end{algorithm}

The start of a comment is indicated by the symbol \commentsymbol.

To assign the value $n$ to the variable $var$, the notation shown in
algorithm \ref{alg:assign} is used.

\begin{algorithm}[H]
  \caption{Assigning the value $n$ to $var$..}
  \label{alg:assign}
  \begin{algorithmic}[1]
    \Let{$var$}{$n$}
  \end{algorithmic}
\end{algorithm}

\section{An example}

In this brief section, we'll be showing how all of the algorithms will
be typeset in this book. We will use binary search as an example
algorithm. It is shown in algorithm \ref{alg:binary-search}.

\begin{algorithm}[H]
  \caption{Binary Search.}
  \label{alg:binary-search}
  \begin{algorithmic}[1]
    \Function{BinarySearch}{$list,length,x$}
      \Let{$min$}{$1$}
      \Let{$max$}{$length$}
      \Repeat
        \Let{$min$}{$\frac{min + max}{2}$}
        \If{$x > list[mid]$}
          \Let{$min$}{$mid + 1$}
        \Else
          \Let{$max$}{$mid - 1$}
        \EndIf
      \Until{$list[mid] = x \OR min > max$}
    \EndFunction
  \end{algorithmic}
\end{algorithm}