\begin{comment}
  \bibliography{project.bib}
\end{comment}

\chapter{Conventions}
\label{cha:pseudocode-convent}

In this chapter, We'll establish all the conventions and glossary that
will be used throughout his text.

\section{Glossary}

\begin{description}
\item[Binary numbers] A binary number is simply a numbers of base
  2. For example, the number $139$ is represented by the binary number
  $1000\ 1011$ since $1 \cdot  2^7 + 0 \cdot 2^6 + 0 \cdot 2^5 + 0
  \cdot 2^4 + 1 \cdot 2^3 + 0 \cdot 2^2 + 1 \cdot 2^1 + 1 \cdot 2^0 =
  2^7 + 2^3 + 2^1 + 2^0 = 128 + 8 + 2 + 1 = 139 $

\item[bit] The separate digits of a binary numbers are reffered to as
  bits. If we say that a bit is \textit{toggled}, we mean that the bit has a
  value of $1$. Controversially, if a bit has the value $0$, then we say
  that the bit is \textit{cleared}.
\item[Highest and Lowest bit] In the numbers $1100$ the highest bit is
  $1$ and has a value of $2^3 = 8$. In order words, the highest bit is
  the bit that has the highest digit position in the number. The
  lowest bit in this number is $0$ and has a value of $0$. We may also
  can the highest bit the 4:th bit, and the lowest bit the 1:st
  bit. So we start counting the bits from one to the position of the
  highest bit.
\item[Most sigficant bit] Alternative name for the highest bit.
\item[Least sigficant bit] Alternative name for the lowest bit.

\item[Byte] A byte is simply a binary number with 8 digits/bits. The
  max value of a byte is $2^8 - 1 = 255$ and there are $2^8$ different
  values that a byte can have. We'll often be referring to bytes
  throughout the entire text, since, as we shall soon see, an image
  represented digitally is just a sequence of bytes.


\item[Hexadecimal] We'll also be using the hexadecimal numeral system
  in this text a lot. Here the numbers 0-9 are given their usual
  values, while the letters A-F are assigned to the values 10-15, so
  that the hexadecimal number \hex{D3} has the value $13 \cdot 16^1 + 3 \cdot
  16^0 ? 13 \cdot 16 + 3 = 208 + 3 =  211$.

\end{description}

\begin{itemize}

\item What is the values of the 2:nd, 4:th and highest bits in the
  number $01001$?

  1, 1 and 0

\item If a binary number has $n$ bits, then what is its the maximum value?

   $2^n - 1$

\item Given a number $x$ find a function $c(x)$ that will count the
  number of bits in that number if it is rewritten on binary form. So
  that, for example, the following holds true: $c(23) = 4$

  (Hint: use the binary logarithm, $\log_2$)

  Now find the another such a function for the hexadecimal system, and
  then find it for \textit{any} number is base n.

\end{itemize}

\section{Pseudocode Conventions}

Instead of showing code examples in some random programming language,
we'll be using pseudocode to explain the algorithms in this book. This
will keep things as general as possible, and not force the reader into
knowing a specific programming language before reading this text.

The pseudocode will be kept as traditional as possible, but we'll
still need to establish several conventions for it, which is what
we're going to do for the rest of the chapter.

\section{Typographical Conventions}

\begin{description}
\item[Keywords] will use \textbf{bold} font.
\item[functions] will be signified by \textbf{uppercase}.
\item[variables] can noticed by their $cursive\ slant$.
\end{description}

\section{Boolean Operators}
\label{sec:boolean-operators}

To signify the boolean operators, or logical operators as we'll often
also refer them to, we'll be using the following symbols:

\begin{description}
\item[$\NOT$] logical \textit{not}
\item[$\AND$] logical \textit{and}
\item[$\OR$] logical \textit{or}
\end{description}

Logical truth is represented by \True, and falseness is represented by
\False.

The truth tables for these operators are presented in the
tables,~\ref{tab:log-and-table},~\ref{tab:log-or-table} and
\ref{tab:log-not-table}.

\begin{table}
  \centering
    \begin{tabular}{|c|c|c|}
      \hline
      $\mathbf{\AND}$ & $\mathbf{\False}$ & $\mathbf{\True}$ \\ \hline
      $\mathbf{\False}$ & $\mathbf{\False}$ & $\mathbf{\False}$ \\ \hline
      $\mathbf{\True}$ & $\mathbf{\False}$ & $\mathbf{\True}$ \\ \hline
    \end{tabular}
  \caption{Logical and truth table.}
  \label{tab:log-and-table}
\end{table}

\begin{table}
  \centering
    \begin{tabular}{|c|c|c|}
      \hline
      $\mathbf{\OR}$ & $\mathbf{\False}$ & $\mathbf{\True}$ \\ \hline
      $\mathbf{\False}$ & $\False$ & $\True$ \\ \hline
      $\mathbf{\True}$ & $\True$ & $\True$ \\ \hline
    \end{tabular}
  \caption{Logical or truth table.}
  \label{tab:log-or-table}
\end{table}

\begin{table}
  \centering
    \begin{tabular}{|c|c|}
      \hline
      $\mathbf{p}$ & $\mathbf{\NOT p}$ \\ \hline
      $\True$ & $\False$ \\ \hline
      $\False$ & $\True$ \\ \hline
    \end{tabular}
  \caption{Logical not truth table.}
  \label{tab:log-not-table}
\end{table}

\section{Bitwise Operators}
\label{sec:bitwise-operators}

\subsection{Notation}

Bitwise operators will be used a lot in this text. We will be using
the notation used in \C{}\cite{kernighan1988c} to represent them in pseudocode:

\begin{description}
\item[$\BitAnd$] Bitwise \textit{AND}
\item[$\BitOr$] Bitwise \textit{OR}
\item[$\BitXor$] Bitwise \textit{XOR}
\item[$\BitNeg$] Bitwise \textit{NOT}
\item[$\ShiftLeft$] Left bit shift
\item[$\ShiftRight$] Right bit shift
\end{description}

Notice that we are using $\BitXor$ for representing bitwise
\textit{XOR} rather than the traditional \C notation $\AND$. This is due to
the fact that we'd otherwise confuse it with logical and, $\AND$.

What follows is a short introduction to the very simple bitwise
operators.

\subsection{Bitwise and, or and xor}

Bitwise AND is just like logical and, except for the fact that it
operates on the on the bit level. Let us for demonstrative consider
the result of the expression $23 \BitAnd 12$. Since the bitwise and
operates on the bit level we first must convert the two numbers to the
binary numeral system: $10110 \BitAnd 01100$. Then the calculation of
now simply done like this:

\begin{center}
  \begin{tabular}{lr}
    & 10110  \\
    $\BitAnd$ & 01100 \\
    \hline
    & 00100 \\
  \end{tabular}
\end{center}

And so the result of the operation $23 \BitAnd 12$ is $00100=4$. So as
you can see, the bitwise operators do boolean logic on the bit level
by considering a toggled bit as truth, and a cleared as falsity

Bitwise OR is in the same way logical OR on the bit level. Let us
perform the former calculation using bitwise or instead of and to
illustrate the difference between them:

\begin{center}
  \begin{tabular}{lr}
             & 10110  \\
    $\BitOr$ & 01100 \\
    \hline
    & 11110 \\
  \end{tabular}
\end{center}

\begin{table}
  \centering
    \begin{tabular}{|c|c|c|}
      \hline
      $\mathbf{\BitXor}$ & $\mathbf{\False}$ & $\mathbf{\True}$ \\ \hline
      $\mathbf{\False}$ & $\mathbf{\False}$ & $\mathbf{\True}$ \\ \hline
      $\mathbf{\True}$ & $\mathbf{\True}$ & $\mathbf{\False}$ \\ \hline
    \end{tabular}
  \caption{Logical exclusive or truth table.}
  \label{tab:log-exlusive-or-table}
\end{table}

Bitwise xor on the other hand, operates on bits by using logical
exclusive or. The truth table of logical exclusive or is given in
table \ref{tab:log-exlusive-or-table}. Using this table, we can easily
understand how bitwise xor works and calculate the value of the
expression $10110 \BitXor 01100$:

\begin{center}
  \begin{tabular}{lr}
              & 10110  \\
    $\BitXor$ & 01100 \\
    \hline
              & 11010 \\
  \end{tabular}
\end{center}

\subsection{Bitwise not}

When dealing with bitwise not, it is important that we consider the
size of the variable that we are storing the number we are to perform
the operation on. If for example $b=10$ and $b$ the variable $b$ is of
type byte, then $b=0000\ 1010$. What bitwise not does is that it
inverts the number, so that at all toggled bits gets cleared, and all
cleared bits gets toggled, so $\BitNeg b = 1111\ 0101$. Now you should
see it was important that we considered the size of the variable data
type. Since if for example the variable $b$ instead was of size 4-bits
isntead of 8-bits, then $b = 1010$ and  the end result of the operation $\BitNeg$
would have been $0101$ instead of $1111\ 0101$.

\subsection{Bitwise shifting}

It is also in bitwise shifting important that we consider the size of
the numbers. Bitwise shifting is actually very simple. All the $b
\ShiftLeft n$ really does, is that it shifts the bit pattern in the
variable $b$ $n$ steps to the right. For the 4-bit number $0011$, this
means that $0011 \ShiftLeft 2 = 1100$. But what would have happened if
the bit pattern was shifted 3 steps. Then one bit is going fall of the
bit boundary and disappear, so $0011 \ShiftLeft 3 = 1000$.

And bitwise left shifting works in pretty much the same way, expect
for the fact that the bit shifting are done to the right instead of
left, so that for the 4-bit number $0110 \ShiftRight 2 = 0001$.

\begin{enumerate}
\item For the byte $b = 1101\ 0001$ calculate the value of the
  expression

  \begin{equation*}
    ((b \ShiftLeft 3) \BitAnd b) \BitXor ((b \ShiftRight 4) \BitOr b)
  \end{equation*}

\item Using the bitwise operators, try to devise an algorithm for
  counting the numbers of toggled bits in a byte. It should for
  example give the end result of $4$ for the numbers $147 = 1001\
  0011$

\item Try to make an algorithm for reversing the bits in a byte. For
  $32=0000\ 0111$ is should give the result $224 = 1110\ 0000$.

\end{enumerate}

\section{Functions}
\label{sec:pseudocode}

We'll be dealing with files in many of these algorithms, so we'll need
to introduce several functions for handling file operations.

\begin{description}[font=\normalfont]
\item[\textproc{ReadByte}] It is assumed from the beginning of the
  algorithm that a file has already been opened for reading. This
  function reads a byte from that file.

\item[\Call{WriteByte}{$byte$}] At the beginning of every algorithm,
  we also assume that there is a file opened for output. This function
  writes a byte to that file.

\item[\textproc{EndOfFileReached}] \True{} if the end the file we're
  reading from have been reached.

\item[\Call{Read}{$size$}] Just being able to read a byte may not
  always be enough. We may want to read a number of a specific size
  from a file. We will do that with this function. This functions
  reads a number of the size $size$ \textit{bytes}(not bits) from a
  file.

\item[\Call{Seek}{$offset,origin$}] Move the from current position in
  the file offset bytes forward. The current position in the file is
  specified by the argument $origin$. If it is $START$, we move from the
  beginning of the file, if it is $END$ we move from the end of the
  file, and if it is $CURRENT$ we move from the current position in
  the file.

\item[\Call{Inc}{\ensuremath{var,n}}] Increases the value of the
  variable $var$ by $n$.

\item[\Call{Dec}{\ensuremath{var,n}}] Decrease the value of the
  variable $var$ by $n$.
\end{description}

\section{Syntax}

As a decently competent programmer, you should be familiar with most
of the control structures we'll be using. But one control structure
that will be used that is probably unfamiliar to you is the custom
control structure called \algorithmicrepeat\index{Repeat Control
  Structure}, which is demonstrated in algorithm \ref{alg:repeat}.

\begin{algorithm}[H]
  \caption{The repeat control structure.}
  \label{alg:repeat}
  \begin{algorithmic}[1]
    \Repeatn{$n$}
    \State $actions$ \Comment{$actions$ are repeated $n$ times.}
    \EndRepeatn
  \end{algorithmic}
\end{algorithm}

and by the way, The start of a comment is indicated by the symbol \commentsymbol.

To assign the value $n$ to the variable $var$, the notation shown in
algorithm \ref{alg:assign} is used.

\begin{algorithm}[H]
  \caption{Assigning the value $n$ to $var$..}
  \label{alg:assign}
  \begin{algorithmic}[1]
    \Let{$var$}{$n$}
  \end{algorithmic}
\end{algorithm}