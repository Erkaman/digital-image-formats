\begin{comment}
  \bibliography{project.bib}
\end{comment}

\chapter{Conventions}
\label{cha:pseudocode-convent}

In this chapter, we'll establish all the conventions that will be used
throughout his text.

\section{Glossary}

\begin{description}
\item[binary number] A binary number is simply a number of base 2. For
  example, the number $139$ is represented by the binary number $1000\
  1011$, since $1 \cdot 2^7 + 0 \cdot 2^6 + 0 \cdot 2^5 + 0 \cdot 2^4
  + 1 \cdot 2^3 + 0 \cdot 2^2 + 1 \cdot 2^1 + 1 \cdot 2^0 = 2^7 + 2^3
  + 2^1 + 2^0 = 128 + 8 + 2 + 1 = 139 $

  \begin{Exercise}[label={n-to-bin}]
    Rewrite the following numbers on binary form:

    \begin{enumerate}[(a)]
    \item 4
    \item 50
    \item 500
    \end{enumerate}

  \end{Exercise}

  \begin{Exercise}[label={bin-to-n}]
    Convert the following binary numbers to ordinary numbers of base
    10:

    \begin{enumerate}[(a)]
    \item $1111$
    \item $1000\ 0000$
    \item $1111\ 1110$
    \end{enumerate}

  \end{Exercise}

  \begin{Exercise}[label={n-bits-max-val}]
    If you know that a binary number has $n$ bits, then what is its
    maximum possible value? Also, how many different possible values can
    such a number have?
  \end{Exercise}

  \begin{Exercise}[label={count-bits-func}]
    Given a number $x$, find a function $c(x)$ that will count the
    number of bits in that number, if it is rewritten on binary
    form. The following should, for example, hold true for this
    function: $c(23) = 4$

    (Hint: use the binary logarithm, $\log_2$, and the floor function,
    $\lfloor x \rfloor$)

  \end{Exercise}

\item[bit] The separate digits of a binary numbers are often referred
  to as bits.

\item[toggled bit] If we say that a bit is \textit{toggled}, we mean that
  the bit has a value of $1$.

\item[cleared bit] And controversially, if a bit has the value
  $0$, then we say that the bit is \textit{cleared}.

\item[Bit counting order] The bit that has the lowest value is
  referred to as bit 0\footnote{This convention is mainly used because
    us programmers like to start counting from $0$ rather than $1$}.
  In the binary number $0100$, bit 0 is cleared( it has the value
  $0$), bit 2 is toggled (it has the value $1$) and the last bit 3 is
  cleared.

\item[Lowest bit] Bit 0 will also commonly be referred to as the
  lowest bit.

\item[Highest bit] Bit $n - 1$ of a binary number of length $n$ bits;
  in other words, the last bit of a binary number. In the number $100$
  the highest bit toggled and the 2 lowest bits are cleared.

\item[Most significant bit] Alternative name for the highest bit.

\item[Least sigficant bit] Alternative name for the lowest bit.

  \begin{Exercise}[label={bits-value-order}]
    What are the values of bits 1,3 and the most significant bit in the
    number $01001$?

  \end{Exercise}

\item[Byte] A byte is simply a binary number with 8 digits/bits. The
  max value of a byte is $2^8 - 1 = 255$ and there are $2^8 = 256$
  different values that a byte can have. We'll often be referring to
  bytes throughout the entire text, since, as we soon shall see, an
  image represented digitally is just a sequence of bytes.

\item[ASCII] A very common text encoding that will be used a lot in
  this text. The ASCII encoding only covers 128 different characters
  and a single ASCII character can therefore always fit in a byte. To
  convert between ASCII characters and byte values you can use the
  ASCII table as it was originally specified at \cite{rfc20}. This
  table is however a bit of a pain to use, and I therefore recommend
  you to use the easier to read version \cite{ascii_table_descr}.

  \begin{Exercise}[label={ascii-to-num}]
    Convert the following characters to their corresponding ASCII
    values:

    \begin{enumerate}[(a)]
    \item A
    \item n
    \item <
    \end{enumerate}
  \end{Exercise}

  \begin{Exercise}[label={num-to-ascii}]
    Convert the following ASCII values to their corresponding characters:
    \begin{enumerate}[(a)]
    \item $35$
    \item $122$
    \item $63$
    \end{enumerate}
  \end{Exercise}

\item[Hexadecimal] We'll also be using the hexadecimal numeral system
  in this text a lot. To signify that a number is hexadecimal, it is
  prefixed by the characters \hex{}. In the hexadecimal system the
  numbers 0-9 are given their usual values, while the letters A-F are
  assigned to the values 10-15, so that the hexadecimal number
  \hex{D3} has the value $13 \cdot 16^1 + 3 \cdot 16^0 = 13 \cdot 16 +
  3 = 208 + 3 = 211$.

  \begin{Exercise}[label={hex-to-n}]
    Convert the following hexadecimal numbers to ordinary numbers of
    base-10:

    \begin{enumerate}[(a)]
    \item \hex{23}
    \item \hex{FF}
    \item \hex{AA}
    \end{enumerate}

  \end{Exercise}

  \begin{Exercise}[label={n-to-hex}]
    Convert the following numbers to hexadecimal:

    \begin{enumerate}[(a)]
    \item 3
    \item 46
    \item 189
    \end{enumerate}

  \end{Exercise}


\item[String] A string is simply a sequence of letters in some
  encoding. The most commonly used encoding in this text will be
  ASCII \cite{rfc20} and we will also most typically use C-strings.

  If the string ``eric'' is stored in the variable \texttt{str}, then

\item[C string] String as they are represented in the C programming
  language. In this language, strings are always terminated by a so
  called NULL character. The NULL character has a value of
  $0$\cite{kernighan1988c}. This means that the string ``eric'' will
  be represented by the sequence of bytes $101,114,105,99, 0$ in the C
  programming language. This is important to know, because in a lot of
  image formats strings are stored as C-strings.

  Why the NULL character at the end of the string even is necessary is
  for rather complex reasons that I will not treat in this text.

\item[File] We are going to be talking a lot about files in this text,
  so it is important that we as early as possible establish a strict
  definition for what a file is. A file is just a sequence of bytes;
  it is in other word just a sequence of numbers. A perfectly valid
  file could for example consist of the numbers $101,114,105,99$. This
  a file that consists of the single string ``eric'', where the
  letters use ASCII encoding. However, if you opened this file in a
  text editor, say notepad, you would only see the letters ``eric''
  and not the numbers that represent the letters. This is because a
  text editor is programmed to see all the bytes in a file as text. If
  you on the other hand opened this file in a hex viewer, you would
  see the file for what it truly is: a sequence of numbers\footnote{Do
    note that in a hex viewer these numbers are, as is implied by the
    name, shown as hexadecimal numbers}.


\item[Offset] When we are talking about an offset we are referring to
  a position in a file. The offset is zero based. When we are talking
  about the number at offset 0 in the file $13,2,1$ we are talking
  about the number $13$. In the same file at the offset $2$, the
  number $1$ can be found.

\end{description}

\section{Pseudocode Conventions}

Instead of showing code examples in some random programming language,
we'll be using pseudocode to explain the algorithms in this book. This
will keep things as general as possible, and not force the reader into
knowing a specific programming language before reading this text.

The pseudocode will be kept as traditional as possible, but we'll
still need to establish several conventions for it, which is what
we're going to do for the rest of the chapter.

\subsection{Typographical Conventions}

\begin{description}
\item[Keywords] will use \textbf{bold} font.
\item[functions] will be signified by \textbf{uppercase}.
\item[variables] can noticed by their $cursive\ slant$.
\end{description}

\subsection{Boolean Operators}
\label{sec:boolean-operators}

To signify the boolean operators, or logical operators as we'll often
also refer them to, we'll be using the following symbols:

\begin{description}
\item[$\NOT$] logical \textit{not}
\item[$\AND$] logical \textit{and}
\item[$\OR$] logical \textit{or}
\end{description}

Logical truth is represented by \True, and falseness is represented by
\False.

The truth tables for these operators are presented in the
tables,~\ref{tab:log-and-table},~\ref{tab:log-or-table} and
\ref{tab:log-not-table}.

\begin{table}
  \centering
  \begin{tabular}{|c|c|c|}
    \hline
    $\mathbf{\AND}$ & $\mathbf{\False}$ & $\mathbf{\True}$ \\ \hline
    $\mathbf{\False}$ & $\mathbf{\False}$ & $\mathbf{\False}$ \\ \hline
    $\mathbf{\True}$ & $\mathbf{\False}$ & $\mathbf{\True}$ \\ \hline
  \end{tabular}
  \caption{Logical and truth table.}
  \label{tab:log-and-table}
\end{table}

\begin{table}
  \centering
  \begin{tabular}{|c|c|c|}
    \hline
    $\mathbf{\OR}$ & $\mathbf{\False}$ & $\mathbf{\True}$ \\ \hline
    $\mathbf{\False}$ & $\False$ & $\True$ \\ \hline
    $\mathbf{\True}$ & $\True$ & $\True$ \\ \hline
  \end{tabular}
  \caption{Logical or truth table.}
  \label{tab:log-or-table}
\end{table}

\begin{table}
  \centering
  \begin{tabular}{|c|c|}
    \hline
    $\mathbf{p}$ & $\mathbf{\NOT p}$ \\ \hline
    $\True$ & $\False$ \\ \hline
    $\False$ & $\True$ \\ \hline
  \end{tabular}
  \caption{Logical not truth table.}
  \label{tab:log-not-table}
\end{table}

\begin{Exercise}[label={boolean}]
  Are the following expressions \True or \False?

  \begin{enumerate}[(a)]
  \item $\NOT \True \AND \True$
  \item $\NOT \True \OR (\True \AND \NOT \False)$

  \item $(\True \AND \True) \AND (\True \AND \NOT(\False \OR \NOT \True))$

  \end{enumerate}
\end{Exercise}

\subsection{Bitwise Operators}
\label{sec:bitwise-operators}

\subsubsection{Notation}

Bitwise operators will also be used a lot in this text. We will use
notation introduced in \C{}\cite{kernighan1988c} to represent them in
pseudocode:

\begin{description}
\item[$\BitAnd$] Bitwise \textit{AND}
\item[$\BitOr$] Bitwise \textit{OR}
\item[$\BitXor$] Bitwise \textit{XOR}
\item[$\BitNeg$] Bitwise \textit{NOT}
\item[$\ShiftLeft$] Left bit shift
\item[$\ShiftRight$] Right bit shift
\end{description}

Notice that we are using $\BitXor$ for representing bitwise
\textit{XOR} rather than the traditional \C notation $\AND$. This is due to
the fact that we'd otherwise confuse it with logical and, $\AND$.

What follows is a short introduction to the very simple bitwise
operators.

\subsubsection{Bitwise and, or and xor}

Bitwise AND is just like logical and, except for the fact that it
operates on the on the bit level. Let us for demonstrative consider
the result of the expression $22 \BitAnd 12$. Since bitwise and
operates on the bit level we first must convert the two numbers to
binary : $10110 \BitAnd 01100$. Then the calculation is simply done
like this:

\begin{center}
  \begin{tabular}{lr}
    & 10110  \\
    $\BitAnd$ & 01100 \\
    \hline
    & 00100 \\
  \end{tabular}
\end{center}

So $22 \BitAnd 12$ is $00100=4$. So as you can see, the bitwise
operators do boolean logic on the bit level by considering a toggled
bit truth, and a cleared bit falsity

\begin{Exercise}[label={bitand}]
  \begin{enumerate}[(a)]
  \item $2 \BitAnd 1$
  \item $255 \BitAnd 23$
  \item $26 \BitAnd 12$
  \end{enumerate}
\end{Exercise}

Bitwise or is in the same way logical or on the bit level. Let us
perform the former calculation using bitwise or instead of bitwise and
to illustrate the difference between them:

\begin{center}
  \begin{tabular}{lr}
    & 10110  \\
    $\BitOr$ & 01100 \\
    \hline
    & 11110 \\
  \end{tabular}
\end{center}

\begin{Exercise}[label={bitor}]
  \begin{enumerate}[(a)]
  \item $172 \BitOr 52$
  \item $3 \ BitOr 3$
  \item $240 \BitOr 15$
  \end{enumerate}
\end{Exercise}

\begin{table}
  \centering
  \begin{tabular}{|c|c|c|}
    \hline
    $\mathbf{\BitXor}$ & $\mathbf{\False}$ & $\mathbf{\True}$ \\ \hline
    $\mathbf{\False}$ & $\mathbf{\False}$ & $\mathbf{\True}$ \\ \hline
    $\mathbf{\True}$ & $\mathbf{\True}$ & $\mathbf{\False}$ \\ \hline
  \end{tabular}
  \caption{Logical exclusive or truth table.}
  \label{tab:log-exlusive-or-table}
\end{table}

Bitwise xor on the other hand, operates on bits by using logical
exclusive or. The truth table of logical exclusive or is given in
table \ref{tab:log-exlusive-or-table}. Using this table, we can easily
understand how bitwise xor works and calculate the value of the
expression $10110 \BitXor 01100$:

\begin{center}
  \begin{tabular}{lr}
    & 10110  \\
    $\BitXor$ & 01100 \\
    \hline
    & 11010 \\
  \end{tabular}
\end{center}

\begin{Exercise}[label={bitxor}]
  \begin{enumerate}[(a)]
  \item $10 \BitXor 10$
  \item $12 \BitXor 7$
  \item $48 \BitXor 16$
  \end{enumerate}
\end{Exercise}

\subsubsection{Bitwise not}

When dealing with bitwise not, it is important that we consider the
size of the numbers that we are performing the operation on. If for
example $b=10$ and the variable $b$ is of type byte, then it
\textit{must} be of length 8 bits: $b=0000\ 1010$. What bitwise not
does, is that it inverts the number, so that at all toggled bits gets
cleared, and all cleared bits gets toggled, so $\BitNeg b = 1111\
0101$.

Now you should see why it was important that we considered the size of
the number. Had the variable $b$ been of size 4 bits, then $b = 1010$
and then the end result of the operation $\BitNeg b$ would have been
$0101$ instead of $1111\ 0101$.

\begin{Exercise}[label={bitnot}]
  What are the values of the following expressions, if all the numbers
  are bytes?

  \begin{enumerate}[(a)]
  \item $\BitNeg 11$
  \item $(\BitNeg 4) \BitXor 4$
  \item $(\BitNeg b) \BitXor b$, for any byte $b$
  \item $(\BitNeg b) \BitOr b$, for any byte $b$
  \item $(\BitNeg b) \BitAnd b$, for any byte $b$
  \end{enumerate}

\end{Exercise}

\subsubsection{Bitwise shifting}

It is also in bitwise shifting important that we consider the size of
the numbers. Bitwise shifting is actually very simple: all the $b
\ShiftLeft n$ really does, is that it shifts the bit pattern in the
number $b$ $n$ steps to the right. For the 4-bit number $0011$, this
means that $0011 \ShiftLeft 2 = 1100$. But what would have happened if
the bit pattern was shifted 3 steps? Then one bit is going fall of the
bit boundary and disappear, so $0011 \ShiftLeft 3 = 1000$.

And bitwise right shifting works in pretty much the same way, expect
for the fact that the bit shifting are done to the right instead of
the left, so that for the 4-bit number $0110$: $0110 \ShiftRight 2 =
0001$.

\begin{Exercise}[label={bit-shiting}]
  Assuming that we are dealing with bytes, what is the value of the
  following expression:

  \begin{equation*}
    120 \BitAnd (3 \ShiftLeft 4)
  \end{equation*}

  (important!) What does this expression really do?

  Hint: what kind of value does it extract from the byte $120$?

\end{Exercise}

\begin{Exercise}[label={bit-equiv}]

  What is

  \begin{enumerate}[(a)]
  \item $1 \ShiftLeft 0$
  \item $1 \ShiftLeft 1$
  \item $1 \ShiftLeft 2$
  \item $3 \ShiftLeft 1$
  \item $3 \ShiftLeft 2$
  \item $3 \ShiftLeft 3$
  \end{enumerate}

  Can you express the operation $N \ShiftLeft S$, if the condition $S
  \geq 0$ holds, in terms of the arithmetic operators? Exponentiation
  counts as an arithmetic operator in this exercise.

\end{Exercise}

\begin{Exercise}[]

\end{Exercise}


\subsection{Syntax}

In this section we will discuss the basic syntax of the pseudocode.

The start of a comment is indicated by the symbol \commentsymbol.

To assign the value $n$ to the variable $var$, the notation shown in
algorithm \ref{alg:assign} is used.



\begin{algorithm}[H]
  \caption{Syntax for assigning the value $n$ to the variable $var$.}
  \label{alg:assign}
  \begin{algorithmic}[1]
    \Let{$var$}{$n$}
  \end{algorithmic}
\end{algorithm}

To store a sequence of values we'll use arrays. If for example the
array $a$ contains the values the $3,1,2$ then to access the first
value of this array, $3$, the syntax $a[0]$ is used. In general, to
access the n:th value of an array you do $a[n-1]$, since arrays are 0
indexed.

Strings are also considered arrays. If the string ``eric'' was
stored in the variable \texttt{str}, then the letter ``e'' can be
accessed with the syntax $str[0]$

To then to go through each value in the array $a$, the syntax
demonstrated in algorithm \ref{alg:for-each} is used.

\begin{algorithm}[H]
  \caption{The for each control structure.}
  \label{alg:for-each}
  \begin{algorithmic}[1]
    \linecomment{Go through every value $v$ in the
      array $a$. The variable  $v$ is here to each element in the
      array in turn}
    \ForEach{$v$}{$a$}
    \linecomment{Do something with $a$ here.}
    \EndForEach
  \end{algorithmic}
\end{algorithm}

In algorithm \ref{alg:repeat} the control structure repeat is
demonstrated.

\begin{algorithm}[H]
  \caption{The repeat control structure.}
  \label{alg:repeat}
  \begin{algorithmic}[1]
    \Repeatn{$n$}
    \State $actions$ \Comment{$actions$ are repeated $n$ times.}
    \EndRepeatn
  \end{algorithmic}
\end{algorithm}

For functions, we'll be using the traditional syntax;
\Call{Func}{$a,b,c$} means that we are calling the function
\textproc{Func} with the argument $a$,$b$ and $c$ and that the value
of this expression is the return value of the function. To return a
value from a function the \textbf{return} statement is used.

The function syntax is demonstrated in algorithm \ref{alg:euclid}. In
this function, Euclid's algorithm is used to calculate the greatest
common divisor of the two given numbers.
\cite{cormen2009introduction_to_algo,weisstein:_euclid_algor}.

\begin{algorithm}
  \caption{Euclid's algorithm for computing the greatest common
    divisor of two numbers.}
  \label{alg:euclid}
  \begin{algorithmic}[1]
    \Procedure{Euclid}{$a,b$}
    \State $r\gets a\bmod b$
    \While{$r\not=0$}
    \State $a\gets b$
    \State $b\gets r$
    \State $r\gets a\bmod b$
    \EndWhile
    \State \textbf{return} $b$
    \EndProcedure
  \end{algorithmic}
\end{algorithm}

The for loop allows for iteration with an explicit counter variable
that is given in the for loop statement. In algorithm \ref{alg:forto}
the for to loop is used to count from $1$ to $9$. Note that it in
reality counts up to $10$, then stops.

\begin{algorithm}
  \caption{For to loop.}
  \label{alg:forto}
  \begin{algorithmic}[1]
    \ForTo{$i$}{$1$}{$10$}
    \linecomment{Print the number 1.}
    \EndFor
  \end{algorithmic}
\end{algorithm}

\subsection{Functions}
\label{sec:pseudocode}

We'll be dealing with files in many of these algorithms, so we'll need
to introduce several functions for handling file operations.

\begin{description}[font=\normalfont]
\item[\textproc{ReadByte}] It is assumed from the beginning of the
  algorithm that a file has already been opened for reading. This
  function reads a byte from that file.

\item[\Call{WriteByte}{$byte$}] At the beginning of every algorithm,
  we also assume that there is a file opened for output. This function
  writes a byte to that file.

\item[\textproc{EndOfFileReached}] \True{} if the end the file we're
  reading from have been reached.

\item[\Call{Read}{$size$}] Just being able to read a byte may not
  always be enough. We may want to read a number of a specific size
  from a file. We will do that with this function. This functions
  reads a number of the size $size$ \textit{bytes}(not bits) from a
  file.

\item[\Call{Seek}{$offset,origin$}] Move the from current position in
  the file offset bytes forward. The current position in the file is
  specified by the argument $origin$. If it is $START$, we move from the
  beginning of the file, if it is $END$ we move from the end of the
  file, and if it is $CURRENT$ we move from the current position in
  the file.

\item[\Call{Inc}{\ensuremath{var,n}}] Increases the value of the
  variable $var$ by $n$.

\item[\Call{Dec}{\ensuremath{var,n}}] Decrease the value of the
  variable $var$ by $n$.


\item[\Call{getbits}{$b, start, end$}] is used to extract a bit
  pattern in the given range. \Call{getbits}{$80, 4, en5$} returns
  $5$ because $80 = 0101\ 0000$. This function will be derived in
  exercise \ref{bitcount}

\end{description}

\begin{Exercise}[label={bitcount}]
  Using the bitwise operators, try to devise a function
  \textproc{bitcount} for counting the numbers of toggled bits in a
  byte. For \Call{bitcount}{$7$} it should return $3$ and for
  \Call{bitcount}{$1$} it should return $1$.

\end{Exercise}

\begin{Exercise}[label={getbits}]

  Make a function named \Call{getbits}{$b, start, end$} that extracts
  the bit pattern of a number $b$ from a given starting position to a
  given ending position. These positions are be zero-based. Example:
  \Call{getbits}{$80, 4, 6$}$=5$, because $80$ is represented by the
  binary number $0101\ 0000$ and from the positions 4 to 6 there is a
  bit pattern that has a value of $5$, which is what this function is
  supposed to extract.

  It is especially important that you do this exercise, because the
  function \textproc{getbits} is going to be used \textit{a lot}
  throughout this text.

  Hint: The bitwise operators $\ShiftLeft, \ShiftRight, \BitAnd,
  \BitNeg$ and the arithmetic operators $+,-$ is all you really need to
  use to solve this problem.

\end{Exercise}

\section{Answers to the exercises}

\begin{Answer}[ref={n-to-bin}]
  \begin{enumerate}[(a)]
  \item $100$
  \item $11\ 0010$
  \item $1\ 1111\ 0100$
  \end{enumerate}
\end{Answer}

\begin{Answer}[ref={bin-to-n}]
  \begin{enumerate}[(a)]
  \item $15$
  \item $128$
  \item $254$
  \end{enumerate}
\end{Answer}

\begin{Answer}[ref={n-bits-max-val}]
  The max value is $2^n - 1$ and the number of different possible
  values is $2^n$
\end{Answer}

\begin{Answer}[ref={count-bits-func}]
  This function is $c(x) = \lfloor log_2 x \rfloor + 1$, where $\lfloor
  x \rfloor$ is the floor funciton that calculates the largest whole
  integer not greater than $x$.

  This function is a bit hard to use however, since few calculators
  include a button for the $log_2$ function. But we can define it in
  terms of the $\ln$ function like this:

  \begin{equation*}
    2^x = y \iff x = log_2 y
  \end{equation*}

  If we from this can find a way to calculate $x$ from $y$ we will
  have solved the problem. And it turns out that we can:

  \begin{align*}
    \ln 2^x = & \ln y \\
    x \ln 2 = & \ln y \\
    x = & \frac{\ln y}{\ln 2}
  \end{align*}

  So $log_2 x = \frac{\ln y}{\ln 2}$. So the final solution can also be
  written like this: $c(x) = \left\lfloor \frac{\ln x}{\ln 2} \right\rfloor + 1$

\end{Answer}

\begin{Answer}[ref={bits-value-order}]
  $1$, $1$ and $0$. Or: toggled, toggled, cleared.

\end{Answer}

\begin{Answer}[ref={ascii-to-num}]

  \begin{enumerate}[(a)]
  \item $65$
  \item $110$
  \item $60$
  \end{enumerate}

\end{Answer}

\begin{Answer}[ref={num-to-ascii}]

  \begin{enumerate}[(a)]
  \item \#
  \item z
  \item ?
  \end{enumerate}

\end{Answer}

\begin{Answer}[ref={hex-to-n}]

  \begin{enumerate}[(a)]
  \item $35$
  \item $255$
  \item $170$
  \end{enumerate}

\end{Answer}

\begin{Answer}[ref={n-to-hex}]

  \begin{enumerate}[(a)]
  \item \hex{03}
  \item \hex{2E}
  \item \hex{BD}
  \end{enumerate}

\end{Answer}

\begin{Answer}[ref={boolean}]

  \begin{enumerate}[(a)]
  \item \False
  \item \True
  \item \False
  \end{enumerate}

\end{Answer}

\begin{Answer}[ref={bitand}]
  \begin{enumerate}[(a)]
  \item $0$
  \item $23$
  \item $8$
  \end{enumerate}
\end{Answer}

\begin{Answer}[ref={bitor}]
  \begin{enumerate}[(a)]
  \item $52$
  \item $3$
  \item $255$
  \end{enumerate}
\end{Answer}

\begin{Answer}[ref={bitxor}]
  \begin{enumerate}[(a)]
  \item $0$
  \item $11$
  \item $32$
  \end{enumerate}
\end{Answer}

\begin{Answer}[ref={bitnot}]
  \begin{enumerate}[(a)]
  \item $244$
  \item $256$
  \item $256$
  \item $256$
  \item $0$
  \end{enumerate}
\end{Answer}

\begin{Answer}[ref={bit-shiting}]

  The value of the expression is $3$. What this expression does, is
  that it extracts the values of the 5:th and 6:th bits from the
  number $120$.

\end{Answer}

\begin{Answer}[ref={bit-equiv}]

  \begin{enumerate}[(a)]
  \item $1$
  \item $2$
  \item $4$
  \item $6$
  \item $12$
  \item $24$
  \end{enumerate}

  The operation $N \ShiftLeft S$ is equivalent to $N \cdot 2^S$.

\end{Answer}

\begin{Answer}[ref={bitcount}]

  \begin{algorithmic}[1]
    \Procedure{bitcount}{$n$}
    \Let{$count$}{$0$}

    \ForTo{$shift$}{$0$}{$8$}
    \If{$n \BitAnd (1 \ShiftLeft shift)$}
    \State \Call{Inc}{$count$}
    \EndIf
    \EndFor

    \EndProcedure
  \end{algorithmic}

\end{Answer}

\begin{Answer}[ref={getbits}]

  \begin{algorithmic}[1]
    \Procedure{bitcount}{$n$}
    \linecomment{Calculate the length of the bit pattern}
    \Let{$len$}{$end - start + 1$}
    \State \textbf{return} $(b \ShiftRight start) \BitAnd (\BitNeg
    (\BitNeg 0 \ShiftLeft len))$
    \EndProcedure
  \end{algorithmic}

  The answer is given above. I'll now explain this function, since it
  will be used a lot throughout this text.

  If given the input \Call{getbits}{$80, 4, 6$} how may we calculate
  the value $5$ from this?

  $80$ is represented by the bit pattern $0101\ 0000$. First, we'll
  right shift down the pattern $start=4$ steps, $0101\ 0000 \ShiftRight
  start$ ,resulting in the bit pattern $0000\ 0101$. Now all that
  remains to be done is that we need to figure out how construct the
  bit pattern $0000\ 0111$ from the input values. Once we have figured
  out how to make this pattern, we can calculate the proper result
  like this: $0000\ 0101 \BitAnd 0000\ 0111 = 101 = 5$

  The bit pattern we want to construct is $0000\ 0111$. We want a
  sequence of $3$ toggled bits from the lowest bit. We can trivially
  calculate this length like this: $end - start + 1 = 6 - 4 + 1 = 3$,
  $ + 1$ is necessary because the bit position are zero based.

  We now have the length of the pattern. Now we need to figure out how
  to construct it. From the operation $\BitXor 0$ you get the bit
  pattern $1111\ 1111$. Then, by shifting this pattern $len$ steps to
  the left, $3 \ShiftLeft 1111\ 1111$ we end up with the value $1111\
  1000$. Now, by simply using the bitwise not operation again,
  $\BitNeg (1111\ 1000)$ ,we end up with the desired pattern $0000\
  0111$. And now we can simply extract the bit pattern like this:
  $0000\ 0101 \BitAnd 0000\ 0111 = 101 = 5$

  This method so general that it not only works on 8-bit numbers, but
  also on $n-bit$ numbers! The fact that we're using the bitwise
  shiftings and not operations doesn't in this case affect the fact
  that this is a very general algorithm. The reader should verify that
  this method also works for 16-bit number by using this algorithm to
  extract the pattern $1100$ from the 16-bit number $1100\ 0011\ 1010\
  1111$.

\end{Answer}
