\begin{comment}
  \bibliography{project.bib}
\end{comment}

% png iso spec?
% http://www.y-adagio.com/public/confs/itsig/rep_itsgxml4/docs/html15948/C029581e%20HTML/png-fdis-png-screen.htm

% http://www.sno.phy.queensu.ca/~phil/exiftool/TagNames/PNG.html
\chapter{Portable Network Graphics}
\label{cha:png}

\section{History}

As explained in section \ref{sec:gif-history}, the creation of the PNG
format was was primarily motivated by the patents encumbering the
already popular GIF format. To solve this problem, a bunch of
developers in the graphics developer community decided to cooperate on
creating a new format. \dt{1}{7}{1996}, version 1.0 of the PNG
specification was released. Then version 1.1 was released on new years
eve of 1998. And the latest version of the PNG specification is
version 1.2, which was released in August 1999. After this it was
finally standardized by ISO as ISO/IEC standard 15948:2004
\cite{roelofs09:_histor_portab_networ_graph_png_format,roelofs99:_png,roelofs:_portab_networ_graph_main}.

What follows is a technical description of the PNG format. It is based
on the references
\cite{boutel:_png_portab_networ_graph_specif_version12,roelofs99:_png}.

  \cite{sivonen:_sad_story_png_gamma_correc}

\section{The Chunk Type}

PNG images are fundamentally built up of chunks.

\begin{figure}
  \centering
  \inputtikz{png_chunk.tex}
  \caption{The PNG chunk datatype}
  \label{fig:png-chunk}
\end{figure}

\section{Critical Chunks}

\section{Color Data Storage}

\subsection{Uninterlaced Color Data}

\subsection{Interlaced Color Data}

\section{Ancillary Chunks of interests }

% quickly summarize the ones that won't get in depth coverage.
