\begin{comment}
  \bibliography{project.bib}
\end{comment}

\chapter{Huffman Coding}
\label{cha:crc}

\section{History}

In 1951, professor Robert Fano gave his students in a course on
information theory two choices: to complete the course they would
either have to write a term paper or take a final exam. In the term
paper the students were asked to solve a seemingly simple problem. In
order to not discourage his students, Fano did not tell them that the
problem that he gave them was a problem that he and Claude Shannon,
the father of information theory, had been unable to
solve\cite{stix91:_profil}.

One of Fano's students was David Huffman. For months he tried to solve
the problem and he was in the end very close to giving up. But then,
in a moment of insight, he suddenly realized how to solve the
problem. He then published his solution to the rest of the world in
\cite{huf52}. The technique he described in this paper is known as
\textit{Huffman Coding}, and this is the compression method that we
will be describing in this chapter.

And luckily, Huffman decided not to patent the algorithm. And because
of this, the Huffman Coding proved to be a very popular compression
technique for a long time. It is for example widely used in
high-definition televisions and modems.

\section{The Algorithm}

The following description of Huffman Coding is mainly based of
Huffman's original description of the algorithm, \cite{huf52}, but
also of
\cite{Salomon:2004:DCC,mark1996data_compression_book,mcfadden92:_hackin_data_compr}.

The key observation behind Huffman Coding is this: in any non-random
data there will almost always be some characters that are more
frequent than others. For example, in the English text the by far most
common letter is ``e''
\cite{lewand2000cryptological,Shannon:2001:MTC:584091.584093}. So if a
random letter is picked from an English language text then it is most
likely that that letter will be an ``e''. So, what Huffman Coding does
is that it assigns more frequent characters to binary numbers with
less digits. These numbers are referred to as \textit{codes}. The
collection of all these codes computed, each assigned to a character,
will henceforth be known as a code \textit{list}.

All values stored in a file are 8-bit numbers. Out of a compression
performance perspective this system is catastrophic, because for
practically all non-random data some byte-values will always be more
frequent than others. For English ASCII text the ASCII code for the
letter ``e'', 101, will be one of the most common characters, so it
would make sense to represent this number by a binary number of a
small length. However, do take in mind that the 8-bit system is used
because that 8-bit numbers are one heck of a lot easier to deal with
than variably sized codes. To read all the contents of a normal file
all you have to do is read it 8 bits for value. Reading values whose
numbers are of different bit lengths is on the other hand much harder
and error prone for the programmer.

\subsection{The Prefix Property}

Before we consider how the Huffman Algorithm chooses these codes, we
also have to consider what properties the code list should
satisfy. Are all possible code lists acceptable?

\begin{table}
  \centering
  \begin{tabular}{llll}
    \toprule
    Letter & Probability & Code List 1 & Code List 2 \\
    \midrule
    $A$ & $0.35$ & \bin{01} & \bin{01} \\
    $B$ & $0.35$ & \bin{11} & \bin{00} \\
    $C$ & $0.15$ & \bin{001} & \bin{010} \\
    $D$ & $0.15$ & \bin{000} & \bin{101} \\
    \bottomrule
  \end{tabular}
  \caption{}
  \label{tab:codes-ex}
\end{table}

Let us say that we dealing with the four-length alphabet $A$, $B$,
$C$, $D$ and that for the sake of argument the probabilities of these
letters are the ones given in table \ref{tab:codes-ex}. Two possible
alternatives(but not all possible) of code lists that could be
assigned to this alphabet are also given in the former table. Let
first consider the first of these code lists.  Character that are more
frequent are in this code list assigned shorter codes, which is just
the property that we are after. Using this code the string ``BABACD''
would get encoded as

\begin{equation*}
  11\ 01\ 11\ 01\ 001\ 000
\end{equation*}

This encoded can data can then be decoded perfectly fine by
translating the codes to their corresponding letters. The list codes
can be transmitted along with the Huffman compressed file. However,
what would have happened if had we used code list 2 instead?

\begin{equation*}
  00\ 01\ 00\ 01\ 010\ 101
\end{equation*}

The first four codes can get decoded perfectly fine, but what about
the fifth? After reading the first two bits of this code it ends up
with the code \bin{01}, which is the code for ``A''. However, how
would the encoder know that it is supposded to read the next bit and
then parse it as the code for ``C''? It can't, basically. The fifth
code in combination with the sixth, \bin{010101} can be parsed a ``C''
followed by an ``D'', but an equally valid way to decode this data
would be as three ``A'':s. So the second set of codes is
\textit{ambiguous} while the first is not. And a file encoded using an
ambiguous code list cannot be losslessly decompressed, since the
encoded data can be parsed in several ways, and such a compression is
therefore not lossless in the first place!

The reason that the first code list is not ambiguous is because it has
the \textit{prefix property}. This means that no code in the code list
is the prefix of another code. On the other hand, in the code list 2,
the code \bin{01} is a prefix of the code \bin{010}, and therefore the
code list was ambiguous in the first place.

\begin{Exercise}[label={prefix-prop}]

  Which of the following code lists has the prefix property?  Also,
  suggest of a way to fix the sets of codes that do not obey by the
  prefix property.

  \begin{enumerate}[(a)]
  \item \bin{011}, \bin{0101},\bin{11},\bin{001}

  \item \bin{01}, \bin{00}, \bin{10}, \bin{101}

  \item \bin{1010101}, \bin{00001}, \bin{001}, \bin{1011}

  \end{enumerate}

\end{Exercise}

% http://books.google.se/books?id=CyCcRAm7eQMC&pg=PA36&redir_esc=y#v=onepage&q&f=false

\newenvironment{huffman}
{\begin{tikzpicture}
    [level distance=10mm,
    % every node/.style={circle,inner sep=1pt,draw=black},
    hnode/.style={circle,inner sep=1pt,draw=black},
    level 1/.style={sibling distance=20mm},
    level 2/.style={sibling distance=10mm},
    level 3/.style={sibling distance=5mm},
    text height=1.5ex,text depth=.25ex]}{\end{tikzpicture}}

\newenvironment{huffmanc}
{\begin{center}\begin{huffman}}
    {\end{huffman}\end{center}}

\newcommand{\charnodeoffset}{1.0cm}

\newcommand{\nodechar}[2]{\node[below=of #1, yshift=\charnodeoffset] {#2};}

\newcommand{\firstcharnode}[3]{  \node (#1) [hnode] {#2};
  \nodechar{#1}{#3}}

\newcommand{\restcharnode}[4]{
  \node (#1) [hnode,right=of #2] {#3};
  \nodechar{#1}{#4}
}

\section{Trees}

A tree is a recursive data structure that is used to implement Huffman
Coding. A tree consists of a collection of nodes with values, and
every node can have two left and right child
nodes. Two example trees:

\begin{huffmanc}
  \node (first)[hnode]{7};
  \node[hnode,right=of first,xshift=10mm]{2}
  child{node[hnode] {3}}
  child{node[hnode] {1}};
\end{huffmanc}

The first of these two trees is a single node with a value of $2$. It
has no left or right child nodes. The node with the value $2$ has two
child nodes. The left and right child have to values $3$ and $1$
respectively. You can also say that the node with the value $2$ is
\textit{parent} of the two nodes with the values $3$ and $1$.

\subsection{Finding The Codes}

Let us consider finding the Huffman codes for the letters of the
string $ababaacdd$ for the alphabet $a$,$b$,$c$, and $d$. First the
number of times every letter occurs in the data, the frequency of the
letter, is assigned to a node of its own:

\begin{huffmanc}
  \firstcharnode{ctree}{1}{c}
  \restcharnode{btree}{ctree}{2}{b}
  \restcharnode{dtree}{btree}{2}{d}
  \restcharnode{atree}{dtree}{4}{a}
\end{huffmanc}

We now have forest of single nodes. The frequencies of the two nodes
with minimum frequencies are now summed. The sum of this operation is
made to a node and this node is made the parent of the two former
minimum frequency nodes:

\begin{huffmanc}
  \node[hnode]{3}
  child{node(cnode)[hnode] {1}}
  child{node(bnode)[hnode] {2}};
  \nodechar{cnode}{c}
  \nodechar{bnode}{b}
  \restcharnode{bnode}{dtree,xshift=-30}{2}{d}
  \restcharnode{dtree}{atree}{4}{a}
\end{huffmanc}

Note that order in which these two nodes are added does not matter. It
is perfectly acceptable to switch the positions of $b$ and $c$, and
assigning $b$ to the left node and $c$ to the right node.

Another alternative for the second minimum frequency to node $b$ was
$d$, neither but this choice will affect the validity of the resulting
code list.

The last step is now repeated, and the two minimum trees are now added
to form an even bigger tree:

\begin{huffmanc}
  \node[hnode]{5}
  child{node[hnode]{3}
    child{node(cnode)[hnode] {1}}
    child{node(bnode)[hnode] {2}}}
  child{node (dnode) [hnode] {2}};
  \restcharnode{dtree}{atree,xshift=-80}{4}{a}

  \nodechar{cnode}{c}
  \nodechar{bnode}{b}
  \nodechar{dnode}{d}

\end{huffmanc}

And the same thing is done for the two remaining trees:

\begin{huffmanc}

  \node[hnode] {9}
  child {node[hnode]{5}
    child{node[hnode]{3}
      child{node(cnode)[hnode] {1}}
      child{node(bnode)[hnode] {2}}}
    child{node (dnode) [hnode] {2}}}
  child{node (anode) [hnode] {4}};

  \nodechar{cnode}{c}
  \nodechar{bnode}{b}
  \nodechar{dnode}{d}
  \nodechar{anode}{a}

\end{huffmanc}

The tree we end up with is known as the Huffman tree of the input
data. But it is important to realize that this is \textit{not} by any
means unique. This is because there will always be the possibility
that these are more than two nodes with a minimum frequency.

To finally create the code list, the following is done: starting from
the root of the tree, it is followed along its branches to reach its
leaves. If a left branch is followed then a cleared bit($0$) is added
to the resulting code, else a toggled bit($1$) is added. So to get the
resulting code for $b$, we follow the tree from its to the leaf that
has character $b$. To reach $b$ we go left,left and then right, so $b$
has the code \bin{001}. The final huffman tree and the rest of the
codes are given in table ?.

\begin{figure}
  \centering
  \begin{huffmanc}

    \node[hnode] {9}
    child {node[hnode]{5}
      child{node[hnode]{3}
        child{node(cnode)[hnode] {1}}
        child{node(bnode)[hnode] {2}}}
      child{node (dnode) [hnode] {2}}}
    child{node (anode) [hnode] {4}};
    \nodechar{cnode}{c}
    \nodechar{bnode}{b}
    \nodechar{dnode}{d}
    \nodechar{anode}{a}

  \end{huffmanc}

  \caption{Computed Huffman Tree for the input data ababaacdd}
  \label{fig:huff}
\end{figure}

\begin{Exercise}[label={nuther}]
  Give an example of another valid Huffman that could have been
  constructed using the above input data. What final code list does
  this tree instead result in?
\end{Exercise}

And Huffman Coding will in general always give valid a code list that
has the prefix property. A tree \textit{not} satisfying that property
would have to look something like this:

\begin{huffmanc}

  \node[hnode] {}
  child {node[hnode]{c}
    child{node[hnode]{b}}
    child{node (dnode) [hnode] {d}}}
  child{node (anode) [hnode] {a}};
\end{huffmanc}

Since the $c$ node is has the parent nodes $b$ and $d$, its code will
also be a prefix of the codes of $b$ and $d$. However, since the leaf
branches are assigned only frequency values and the leaf nodes are
assigned characters, it is impossible for such a tree to result from
Huffman Coding.

It is easy to see how the algorithm assigns shorter codes to more
frequent character. Since frequent characters will get assigned the
highest frequencies, they will also get added to the tree at the later
parts of the algorithm, and so they get placed higher in the tree, and
therefore they will get assigned the shorter codes. And similarly,
since the least occurring codes will get assigned small values, they
also get added into the tree early, so will get placed deep down in
the tree, and they will therefore get assigned the longer codes.

\subsection{Transmitting The Huffman Tree}

A file compressed by Huffman Coding not only needs the encoded data,
but also the set of codes in order to decode the compressed
data. There are many different ways of doing this. In chapter ?, we
will discuss how this is odne in a very pculiaiar way in the DEFLATE method.

\section{Answers to the exercises}

\begin{Answer}[ref={prefix-prop}]

  \begin{enumerate}[(a)]
  \item Has the prefix property.
  \item Does not have the prefix property. If the codes are on the
    other hand changed to \bin{01}, \bin{00}, \bin{101} and \bin{100}
    they will obey the prefix property.
  \item Has the prefix property.

  \end{enumerate}

\end{Answer}

\begin{Answer}[ref={nuther}]
  \begin{huffmanc}

    \node[hnode] {9}
    child {node[hnode]{5}
      child{node[hnode]{3}
        child{node(cnode)[hnode] {1}}
        child{node(bnode)[hnode] {2}}}
      child{node (dnode) [hnode] {2}}}
    child{node (anode) [hnode] {4}};

    \nodechar{cnode}{c}
    \nodechar{dnode}{b}
    \nodechar{bnode}{d}
    \nodechar{anode}{a}

  \end{huffmanc}

  This tree results in the codes $a=$\bin{1}, $b=$\bin{01},
  $c=$\bin{000}, $d=$\bin{001}

\end{Answer}