\begin{comment}
  \bibliography{project.bib}
\end{comment}

\chapter{\gif -- Graphics Interchange Format}
\label{cha:gif}

\begin{refsection}

  \section{History}

  \subsection{Troublesome Patents}

  AS you may rememerber from the last chapter, the \lzw was
  inventented by Terry Welch. However, he also filed a software patent
  for the algorithm. This patent was U.S. Patent 4,558,302
  \cite{welch85:_u,roelofs09:_histor_portab_networ_graph_png_format}.

  \newcommand{\compus}{CompuServe\xspace}

  And in 1987, \compus started designing a new \textit{portable and
    compressed} image format called \textit{Graphics Interchange
    Format}, abbreviated \gif. To perform the compression in this new
  format, \compus chose a small variation of the compression algorithm
  \lzw. The next thing that \compus did, was that they released this fresh
  image format to the general public.

  But when they released it, they and the rest of the world had no idea
  that Unisys where pursuing a patent for the \lzw compression
  algorithm. And once Unisys had gained the patent and informed the rest
  of the world of this, the GIF format had already been released!

  But it wasn't until 1993 that Unisys actually started pursuing
  companies that were selling software using the \lzw algorithm. And one
  of those companies that Unisys started attacking because of this was
  of course \compus. And at christmas day 1994 Unisys and \compus
  finally came to an agreement: all programmers writing software that
  creates or reads images will have to have a license from Unisys.

  This of course causes an uproar in the graphics developer community
  at the time. Some of these developers formed that would develop a new
  \textit{free} graphics format that would replace \gif, because
  \gif was very popular at the time
  . This
  new graphisc format was \png, which we will discuss in chapter
  \ref{cha:png} \cite{roelofs09:_histor_portab_networ_graph_png_format,caie:_sad}.

  \subsection{Freedom at last}

  But since 2006 all \lzw related patents have expired. Software
  developers are in other words finally free to use the \lzw algorithm
  just as they please. And this also means that the \gif format is
  finally free from the shackles of Unisys. So programmers are free to
  write and develop software that deals with \gif images \cite{caie:_sad}.

  \section{Introduction to the \gif format}

  So, in this chapter we'll the discussing the inner workings of \gif
  images. Because the \gif format also supports animation, it is
  slightly more advanced than the \tga format we discussed in chapter
  \ref{cha:tga}. Understanding the general structure of the image
  format is actually the easiest part, the hardest part of the format
  is the compression algorithm that it uses, which is a small
  variation of the \lzw algorithm. But since already discussed the \lzw
  algorithm in chapter \ref{cha:lzw}, this will only be marginally
  difficult. 
  
  \section{Color Data}

  \printbibliography[heading=  subbibliography]

\end{refsection}