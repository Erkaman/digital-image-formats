\begin{comment}
  \bibliography{project.bib}
\end{comment}

\chapter{Graphics Interchange Format}
\label{cha:gif}

\begin{refsection}

  \section{History}

  % http://netghost.narod.ru/gff/graphics/book.htm

  \subsection{Troublesome Patents}

  As you may remember from chapter \ref{cha:lzw}, the LZW algorithm
  was invented by Terry Welch. He also filed a software patent for the
  algorithm. This patent was U.S. Patent 4,558,302
  \cite{welch85:_u,roelofs09:_histor_portab_networ_graph_png_format}.

  \newcommand{\compus}{CompuServe\xspace}

  In 1987 \compus started designing a new \textit{portable and
    compressed} image format called \textit{Graphics Interchange
    Format}, abbreviated GIF. To do all the data-compression in this
  new format, \compus chose to use a small variation of the
  compression algorithm LZW. Then \compus released this fresh image
  format to the general public
  \cite{roelofs09:_histor_portab_networ_graph_png_format}.

  But at the time they released it, they and the rest of the world had
  no idea that Unisys was pursuing a patent for the LZW compression
  algorithm. And once Unisys had gained the patent and informed the
  rest of the world of this, the GIF format had already been released
  \cite{roelofs09:_histor_portab_networ_graph_png_format}!

  But it wasn't until 1993 that Unisys seriously started pursuing
  companies that were selling software using the LZW algorithm. And
  one of those companies that Unisys started attacking because of this
  was, of course, \compus. And at 28 December 1994, Unisys and \compus
  finally came to an agreement: all companies writing software that
  creates or reads GIF images will have to purchase a license from
  Unisys for using the LZW algorithm
  \cite{roelofs09:_histor_portab_networ_graph_png_format,caie:_sad}.

  This of course caused an uproar in the graphics developer community
  at the time. Some of these upset developers formed a group that
  would develop a new, \textit{free} graphics format that would replace
  GIF. This new graphics format was PNG, which we shall discuss in
  chapter \ref{cha:png}
  \cite{roelofs09:_histor_portab_networ_graph_png_format,caie:_sad}.

  \subsection{Freedom at last!}

  But since 2006 all LZW related patents have expired. Software
  developers are in other words finally free to use the LZW algorithm
  just as they please without having to request a license for doing so
  from Unisys. And this also means that the GIF format is finally free
  from the shackles of Unisys. \cite{caie:_sad}.

  \section{Introduction to the GIF format}

  So in this chapter we'll be discussing the inner workings of GIF
  images. Because the GIF format also supports animation, it is
  slightly more advanced than the TGA format we discussed in chapter
  \ref{cha:tga}. Understanding the general structure of the image
  format is actually the easiest part, the hardest part of the format
  is the compression algorithm that it uses, which is a small
  variation of the LZW algorithm. But since we have already discussed
  the LZW algorithm in chapter \ref{cha:lzw}, this will only be
  marginally difficult.

  There are actually two version of the GIF format: version 87a and
  89a. Version 89a is obviously the newer of these two versions and
  was released in 1989. GIF images that are of version 87a don't
  support all of the features of version 89a and they are nowadays
  very rare. And since version 89a is backwards compatible with
  version 87a, we'll only be discussing version 89a in this chapter
  \cite{gif89a,gif87a}.

  The entire version 89a of the GIF format is specified in this
  document: \cite{gif89a}.

  \section{The building blocks of the GIF format}

  \subsection{Blocks}

  The GIF format is organized into so called blocks blocks. There are three kinds of
  blocks:

  \begin{enumerate}
  \item control blocks
  \item graphic-rendering blocks
  \item special purpose blocks
  \end{enumerate}

  Controls blocks are used to specify how an image should be displayed
  and gives information about how the image data should be read.

  Graphic rendering blocks are used to contain the color data that
  should be rendered, displayed, on the screen. Control blocks always
  occur before graphics rendering blocks and they often have a limited
  \textit{scope}, meaning that they almost always only affect the
  graphics rendering block that follow them. You'll understand better
  what this means later.

  And miscellaneous things that aren't handled by the two former
  block types are done by the special purpose blocks.

  The data that blocks consists of is specified in two kinds number of
  types called \textit{byte} and \textit{unsigned} respectively. They
  are defined like this:

  \begin{description}
  \item[byte] An 8-bit unsigned integer.
  \item[unsigned] A 16-bit unsigned integer.
  \end{description}

  \subsection{Data Sub-blocks}
  \label{sec:gif-data-sub-blocks}

  Sub-blocks are used to store data in GIF images that by definition
  is sequential. An example of such data is the compressed color data,
  which is just long sequence codes that were outputted in the GIF
  variation of the LZW algorithm. Sub-blocks is just sequence of
  smaller blocks so let's go through the parts of such a sub-block:

  \subsubsection*{Block Size(Byte)}

  At the beginning of every sub-block is its size. This value is put
  into a single byte, meaning that the maximum possible size of a
  sub-block is 255.

  \subsubsection*{Block Data Values(Bytes)}

  Following the block size is sequence of bytes. These bytes are the
  data of the sub-block. The number of bytes in the sub-block is the
  size of the block size field that was specified earlier.

  The the sub-blocks are read one by one until the block terminator is
  found:

  \subsubsection*{Block Terminator(Byte)}

  If after reading a sub-block the next byte found is 0, that means
  that you have found the empty sub-block. This sub-block is also
  known as the block-terminator. Should such a block be discovered, you
  have reached the end of a sub-blocks sequence.

  Figure \ref{fig:gif-data-sub-blocks} demonstrates how the string
  ``ERICA'' could be stored using data sub-blocks. As can be seen, it
  is perfectly acceptable that the data is split over several
  sub-blocks, because when reading the sub-blocks all of this data is
  concatenated into one large string. But this example may seem a bit
  silly, because the string ``ERICA'' can easily be fit into a single
  block instead of splitting it into two. But this was only done for
  demonstrative purposes and a more efficent way of storing the string
  would be putting it all into one sub-block of length $5$.

  Because the data is split up over several sub-blocks, reading it may
  be a bit tricky. Algorithm \ref{alg:gif-sub-blocks} demonstrates one way of reading all the
  data of a sequence of sub-blocks into one array.

  \begin{figure}
    \centering
    \inputtikz{gif_subblocks}
    \caption{GIF Data Sub-Blocks storing the string ``ERICA''}
    \label{fig:gif-data-sub-blocks}
  \end{figure}

  \begin{algorithm}[H]
    \caption{Reading all the data of sequence of sub-blocks into one array.}
    \label{alg:gif-sub-blocks}
    \begin{algorithmic}[1]
      \Let{$blockSize$}{\VoidCall{ReadByte}}
      \Let{$i$}{$0$}
      \Repeat
      \Repeatn{$blockSize$}
      \Let{$data[i]$}{\VoidCall{ReadByte}}
      \Inc{i}{1}
      \EndRepeatn

      \Let{$blockSize$}{\VoidCall{ReadByte}}

      \Until{$blockSize == 0$}
    \end{algorithmic}
  \end{algorithm}

  \subsection{Packed Fields}

  Another kind of data storage mechanism are packed fields. In packed
  fields the fields values are stored in the bits of a single byte. To
  make things easier we'll be calling the top bit, the one usually
  called bit 7 or 8, bit 1, and we'll call the lowest bit, the bit
  that is usually called bit 1 or 0, bit 8. See also figure
  \ref{fig:gif-packed-fields}.

  \begin{figure}
    \centering
    \inputtikz{gif_packed_fields}
    \caption{The order of the bits we'll be describing in packed fields.}
    \label{fig:gif-packed-fields}
  \end{figure}

  \subsection{Extensions}

  Extension block are used to modify the behavior of other blocks. You
  can say that they \textit{extend} their behavior. To make them
  easier to identify, the extension blocks always begin with the value
  \hex{21}. This value is also known as the extension
  introducer\index{extension introducer}. Following the extension introducer is the extension
  label. This field identifies the kind of an extension.

  After these two labels you can find a sequence of sub-blocks
  containing all of the data of the extension block. The data is
  stored like this to make it easier to skip over the extension block,
  because the data of some extension blocks is not at all necessary to
  properly render the file.

  \section{Header Blocks}

  Now we will be describing all the possible kinds of blocks that
  can occur in a GIF file. We'll be describing them in the order they
  usually appear. I will also be telling which blocks are required and
  which blocks are optional.

  Coming first in the GIF file format are the header blocks.

  \subsection{Header(Required)}

  The header contains the magic numbers\index{magic number} of the GIF format and is always
  first in a GIF-file.

  \subsubsection*{Signature(3 bytes)}

  Identifies the file as a GIF-file. This field always has the value
  ``GIF''.

  \subsubsection{Version(3 bytes)}

  Gives the version of the GIF-file. Remember, there are two version
  of the GIF format: ``87a'' and ``89a'', and these two are the two
  only possible values of this field.

  \subsection{Logical Screen Descriptor(Required)}

  Following the header block is always the logical screen descriptor
  block. This block contains information about the global color
  table(we'll discuss this block very soon) and the sizes of the
  screen the image will be rendered on.

  \subsubsection{Logical Screen Width(Unsigned)}
  \subsubsection{Logical Screen Height(Unsigned)}

  These give the size of the logical screen where the image will be
  rendered. So these fields are basically the image
  measurements. However, do note that the GIF format allows several
  images to be stored in a GIF-file to allow for such things as
  animation. These images can have measurements that are smaller than
  but not greater than the logical screen measurements.

  \subsubsection{Packed Fields}

  This packed field contains data about the global color table of the
  file. This field will always be found here, even if there is no
  global color table in the image to be found.

  \paragraph{Bit 1 --  Global Color Table Flag}

  This flag indicates whether there is a global color table in the
  image.

  \paragraph{Bit 2-4 -- Color resolution}

  Now, this field has a very misleading name. You may think this field
  is simply used to indicate the color depth of the image, but it is
  not. This field indicates the color resolution of the
  \textit{original image}! You see, the GIF has a very limited range
  of colors. It only permits for a palette of maximum length 256,
  where every color has color depth of 24 bits.

  Since this field gives no rendering information about the image,
  this field is pretty much useless and you can safely ignore it.

  \paragraph{Bit 5 -- Sort Flag}

  If this flag is true, the colors in the global color table are
  sorted in the order decreading importance. This means that the most
  frequently used colors are sorted first in the color
  table. According to \cite{gif89a}:

  \begin{quote}
    This assists a decoder, with fewer available colors, in choosing
    the best subset of colors; the decoder may use an initial segment
    of the table to render the graphic.
  \end{quote}

  What this basically means is that the decoder can get a away with
  only loading the first few colors of the color table and still get a
  decently well-rendered image. if this is done there will be some
  gaps in the rendered image, because certain rarely used colors will
  not be rendered at all. the reason that you would want to do this is
  that it would make the loading and rendering process quicker, i
  think.

  but because of how powerful the computers of today are, this field
  is as good never used nowadays and you can safely ignore of it
  without harming the loading process.

  \paragraph{Bit 6-8 -- Color Table Size}

  This field is used to calculate the size of the global color
  table. If we let the variable $n$ signify the value of this field,
  then the \textit{real} size of the global color table can be
  calculated with the formula:

  \begin{equation}
    \label{eq:calc-global-color-table-size}
    \textrm{Global Color Table Size} = 2^{n + 1}
  \end{equation}

  \subsubsection{Background Color Index(Byte)}

  This field's value is the index of the color table that should not
  be rendered at all. You see, in the GIF format images may be
  rendered in sequende after each other. Pixel that have this index
  render pixels of the image ``below'', or before the current image.


  \subsubsection{Pixel Aspect Ratio(Byte)}

  The pixel aspect is ratio is defined as

  \begin{equation*}
    \textrm{Pixel Aspect Ratio} = \frac{\text{Pixel Width}}{\text{Pixel Height}}.
  \end{equation*}

  todo: fix

  \section{Color Tables}

  \subsection{Global Color Table(Optional)}

  After the logical screen descriptor comes the global color
  table. However, if the Global Color Table flag is set to
  \textbf{false} the global color table will not occur at all within
  the image!

  The global color table is just as in any other image format long
  array of colors. Every color is given 24 bits and every primary
  color is given 8-bits. So the length of this field is $3 \cdot
  \textrm{Global Color Table Size}$. In which $\textrm{Global Color
    Table Size}$ is calculated according to equation
  (\ref{eq:calc-global-color-table-size}).

  Figure \ref{fig:gif-color-table} shows an example of a global color
  table in your typical GIF image. The size of the global table size
  given in the packed field of the logical screen descriptor will in
  the case of this color table have the value $0$. The number of bytes
  in this color table can then be calculated like so:

  \todo{fix this explanation }

  \begin{figure}
    \centering
    \inputtikz{gif_color_table}
    \caption{Example of GIF color table }
    \label{fig:gif-color-table}
  \end{figure}

  \begin{equation*}
    3 \cdot 2^{0 +1} = 3 \cdot 2^1 = 3 \cdot 2 = 6
  \end{equation*}

  \subsection{Local Color Tables(Optional)}

  If there is a not a global color table in the GIF image, a local
  color table is instead used. The local color table is always right
  before the image data. This block laid out just like the global
  color table block. It's size is however given by a block we will discuses
  later.

  \section{Preimage Data Blocks}

  The preimage data blocks always occur before the color data and help
  the decoder in loading the data and described how the image is to be
  rendered.

  \subsection{Image Descriptor(Required)}

  Before the image data there will \textit{always} be a Image
  Descriptor. It contains the data you need to know to properly render
  and process the image data.

  \subsubsection{Image Separator(Byte)}

  This field identifies the block as an image descriptor and it always
  has the value \hex{2C}.

  \subsubsection{Image Left Position(Unsigned)}
  \subsubsection{Image Top Position(Unsigned)}

  These fields give the position of the image on the screen in respect
  to the logical screen descriptor specified in the logical screen
  descriptor. If these two fields both have the value zero, the image
  is rendered in the top-left of the logical screen. If the values are
  for instance 4 and 2, the image is shifted 4 pixels to the right and
  2 pixels down before being rendered.

  \subsubsection{Image Width(Unsigned)}
  \subsubsection{Image Height(Unsigned)}

  These fields allows you to specify the size of the image.

  \subsubsection{Packed Fields(Byte)}

  \paragraph{Bit 1 -- Local Color Table Flag}

  This flag tells whether a local color table is found after this
  block.

  A separate image in a GIF-file can have a specific color table,
  local to only itself. An image \textit{needs} to have a local color
  table if no global color table is in the image. If no global table
  can be found in the file \textit{every} image in the GIF-file must
  have a local color table.

  However, if an image do have a global color table, but the existence
  of a local color table is indicated by this field, that means that
  the local color table will be used instead of the global color
  table.

  \paragraph{Bit 2 -- Interlace Flag}

  This flag indicates whether the image is interlaced.

  \todo{explain interlacing}

  \paragraph{Bit 3 -- Sort Flag}

  This indicates whether the color data in the local color table is
  sorted. This works in the exactly same as it does in global color
  table.

  \paragraph{Bit 4-5 -- Reserved}

  These bits are reserved, meaning that their both values are set to
  $0$ and that they are \textit{unused}.

  \paragraph{Bit 5-8 -- Size of Local Color Table}

  Helps calculate the size of the local color table. The actual size
  of the local color table is calculated in the same way as it is in
  the global color table:

  \begin{equation*}
    \textrm{Local Color Table Size} = 2^{n + 1}
  \end{equation*}

  \subsection{Graphic Control Extension(Optional)}

  The Graphic Control block is an extension block that mostly controls
  how the animation is done in an animated GIF image.

  \subsubsection{Extension Introducer}

  This field Identifies the block as an extension block and it always
  contains the value \hex{21}.

  \subsubsection{Graphic Control Label}

  Identifier the extension block as a graphic control extension
  block. Contains the fixed value \hex{F9}.

  \subsubsection{Block Size}

  The data in the graphic control extension block is contained in one
  single sub-block. This is the size of that one sub-block, which is
  \textit{always} $4$.

  \subsubsection{Bits 1-3 -- Reserved}

  \textit{Unused}.

  \subsubsection{Bits 4-6 -- Disposal Method}

  Now this is one interesting field. When a GIF image is animated the
  file essentially contains a sequence images. In the animation these
  images are displayed one after one. After each image has been
  displayed, you can do different things to dispose of the previous
  image when current image is to be displayed. The action that will be
  performed in order to dispose of the former image is specified by
  this field. The different possible actions are:

  \begin{description}

  \item[0] No disposal is specified. This basically means that
    the decoder doesn't need to do anything at all.

  \item[1] Do not dispose the previous image. The current image is
    rendered over the previous one. This is very useful for when you
    in an animation only want change small parts of an image.

  \item[2] After the former image has been displayed the area used by
    it must be restored to the background color when the next image is
    to be rendered. The background color was, as you should know,
    specified by the Background Color Index field of the logical
    screen descriptor.

  \item[3] Restore to previous. Meaning that the decoder must restore
    the area taken up by the former and replace it with what was
    rendered there before the former image. \todo{explain this one a
      bit better.}

  \item[4-7] These actions are undefined and they should never ever be
    found in a valid GIF-file.
  \end{description}

  \subsubsection{Bit 7  -- User Input Flag}

  If this flag is true, the loading of the next field will continue
  only after user has entered some input determined by the application
  loading the image.

  But in practice this field is never used and practically no applications
  in existence support it.

  \subsubsection{Bit 8 -- Transparency Flag }

  This flag whether the Transparency Index field should be used, if
  the value of this is \textit{false} then the value in that field must be
  ignored.

  \subsection{Delay Time(Unsigned)}

  This specifies the number of centiseconds the decoder should wait after
  having rendered the image. This is very useful for when making
  animated GIFs. A centisecond is a hundredth of a
  second:

  \begin{equation*}
    \SI{1}{\centi\second} = \frac{1}{100}\si{\second}
  \end{equation*}

  \subsection{Transparency Index(Byte)}

  If this field is to be used(as is specified in the Transparency
  Flag.), it contains the color index that should not be rendered at
  all when encountered.

  So as you can see from this field, the GIF format supports a very
  simple form of transparency. But it doesn't support any form of the
  alpha color channel. Either a pixel is colored or it is not, there is
  no state in between.

  \subsection{Block Terminator(Byte)}

  This field terminates the sub-block that this extension block
  consists of.

  \section{Rendering blocks}

  For storing the data to be rendered from a GIF image these blocks
  are used.

  \subsection{Table Based Image Data}

  Now we have finally reached the most important kind of block that
  you can find in a GIF file: the table based image data block. Since this
  block is rather complex we'll explain it in two steps: first we'll
  explain the fields of this blocks, and we'll explain how to extract
  the image data from the data given in the fields of this block.

  \subsubsection{LZW Minimum Code Size(Byte)}

  The GIF format uses a variation of the LZW compression algorithm
  for storing it's color data. The huge difference between the
  original LZW algorithm and the version used in GIF is that the
  code sizes of the individual codes increase as bigger codes are
  needed in the image data. We'll explain this a lot more in depth
  soon.

  \subsubsection{Compressed Image Data(Sub-blocks)}

  The compressed image data is stored in a sequence of sub-blocks. If
  you want to extract the full sequence of data values of these
  sub-blocks I refer you to algorithm \ref{alg:gif-sub-blocks}.

  \subsubsection{The Compression Algorithm}

  Now we are to discuss how to decompress the color using the
  information given in the two previous fields. The full decompression
  algorithm is given in algorithm \ref{alg:gif-decompression}. As you
  can see, it is extremely to the original LZW algorithm. Let us now
  go through the most important differences.

  At the beginning the algorithm a table is filled all of the colors
  of the color table local to the data. However, two additional codes
  are added to the color table: the clear and the end codes. If for
  example of our color table consisted of the colors black and white,
  the resulting color that would be used in the decompression
  algorithm is then given in table
  \ref{tab:example-gif-color-table}. These codes only occur in the GIF
  version of the LZW algorithm. If the clear code is found the color
  table is reset and restored to it's initial state. And if the End
  Code is found that means that the end of the compressed data has
  been reached and that the algorithm thus should terminate.

\begin{table}
    \centering
    \begin{tabular}{ll}
      \toprule
      Color & Code \\
      \midrule
      Black & 0 \\
      White & 1 \\
      Clear Code & 2 \\
      End Code & 3 \\
      \bottomrule
    \end{tabular}
    \caption{Example of an initial GIF color table.}
    \label{tab:example-gif-color-table}
  \end{table}

  And the second huge difference in the GIF version of the LZW
  algorithm, is code sizes used increases. The minimum code size given
  at the beginning of this block is used to specify the number that is
  one less than the code size we should be starting with. So this is
  one less than the size of the codes that we should be inputting.

  Every time a new code is added to the table we check if the size of
  the current code is also equal to the max value of the current code
  size. If so, and the code is not equal to it's maximum allowable
  value in the GIF format -- $12$ -- then the value of the code size
  is increased by one.

  And these are the greatest differences between the original LZW
  algorithm and the GIF version.

  \begin{algorithm}[H]
    \caption{GIF Decompression algorithm}
    \label{alg:gif-decompression}

    \newcommand{\InputCodeSize}{\Call{inputCode}{$codeSize$}}

    \begin{algorithmic}[1]

      \Let{$ClearCode$}{$colorTableSize$}
      \Let{$EndCode$}{$colorTableSize + 1$}

      \Let{$IntitalCodeSize$}{\VoidCall{ReadByte} $+ 1$} \Comment{
        Read the initial code size}
      \Let{$codeSize$}{$IntitalCodeSize$}

      \State \InputCodeSize \Comment{Skip the first
        clear code}

      \State \VoidCall{resetCompressionTable}
      \Let{$nextCode$}{$colorTableSize + 2$}

      \Let{$oldCode$}{\InputCodeSize}
      \State \Call{outputCode}{oldCode}

      \Let{$character$}{$oldCode$}
      \Let{$newCode$}{\InputCodeSize}

      \While{$newCode != EndCode$}

        \If{$newCode = ClearCode$}

        \linecomment{Reset the compression table if a clear code is
          found}

          \State \VoidCall{resetCompressionTable}
          \Let{$nextCode$}{$colorTableSize + 2$}
          \Let{$codeSize$}{$IntitalCodeSize$}

          \Let{$oldCode$}{\InputCodeSize}
          \Call{outputCode}{oldCode}

          \Let{$newCode$}{\InputCodeSize}

        \EndIf

        \If{$\NOT$ \Call{IsInTable}{$newCode$}}
          \Let{$string$}{\Call{translate}{oldCode}}
          \Let{$string$}{$string + character$}
        \Else
          \Let{$string$}{\Call{translate}{$newCode$}}
        \EndIf

        \State \Call{outputString}{$string$}

        \Let{$character$}{$string[0]$}

       \If{$nextCode \leq (2^{12} - 1)$}

          \State \Call{$AddToStringTable$}{$nextCode, oldCode, character$}

          \If{$nextCode == 2^{codeSize} - 1 \And nextCode \ne 2^{12} -
            1$}
            \Inc{codeSize}{1}
          \EndIf
          \Inc{nextCode}{1}
      \EndIf

      \Let{$oldCode$}{$newCode$}

      \Let{$newCode$}{\VoidCall{$ReadByte$}}

      \EndWhile

    \end{algorithmic}
  \end{algorithm}

  \subsection{Plain Text Extension}

  The second way of storing rendering data is through the plain text
  extension. This extension basically allows you to render text as
  graphic in a GIF image. While this might have been seen as a good
  idea at the time it was thought up, nowadays practically no programs
  implement support for this block. In writing a GIF decoder you may
  very safely assume that you will never ever find this block, but if
  you do indeed find it just ignore it.

  \section{Miscellaneous Blocks}

  \subsection{Trailer(Required)}

  Once we have reached the trailer we have also reached the end of the
  image file. Once the trailer block has been reached, the decoder's
  job is done. It is a single field block containing only one value:

  \subsubsection{GIF trailer(Byte)}

  Identifies the block as the trailer block. Always has the value
  \hex{3B}.

  \subsection{Comment Extension(Optional)}

  The comment extension blocks allows you to embed a comment into a
  GIF-file. This block is very rare in GIF files nowadays and if you
  only want to load the image data you can safely ignore it.

  \subsubsection{Extension Introducer(Byte)}

  Identifies the block as an extension block. Always contains the
  value \hex{21}.

  \subsubsection{Comment Label(Byte)}

  This label identifies this extension block as a comment extension
  block. Contains the constant value \hex{FE}.

  \subsubsection{Comment Data(Data Sub-block)}

  Containg the comment data is just a long sequence of data
  sub-blocks. I refer you back to section \ref{sec:gif-data-sub-blocks} if
  you want to know how load of these sub-blocks.

  \subsubsection{Block Terminator(Byte)}

  And this zero-length data block terminates the data sub-blocks and hence
  the comment data.

  \subsection{Application Extension}

  The application extension block allows you to give information to a
  specific application about a GIF file. Because this block is
  entirely application specific, it may seem pointless to give a
  thourough disscussion of this block. But since it is this block that
  allows an animated GIF image to be infinitely looped, this is indeed
  a very intesting block.

  \subsubsection{Extension Introducer(Byte)}

  Identifies the block as an extension block. Always has the value
  \hex{21}.

  \subsubsection{Extension Label(Byte)}

  Identifier the extension block as an application extension
  block. Always has the value \hex{FF}.

  \subsubsection{Block Size}

  The block size of the application extension block. Has the constant
  value $11$.

  \subsubsection{Application Identifier(8 Bytes)}

  This identifies the application that this block is supposed to
  affect.

  \subsubsection{Application Authentication Code(3 Bytes)}

  These values are used to show the that Application Identifier is
  authenticated to the application given in the application
  identifier.

  \subsubsection{Application Data(Sub-blocks)}

  This sequence of data sub-blocks contains the actual data that
  passed to the application. Different data given in this sequence of
  data sub-blocks often results in the application reacting
  differently to this block.

  \subsection{Infinite GIFs}

  Using the application extension we can make an animated GIF repeat
  it's animation infinitely long.

  This is done using the Netscape Application Extension. The
  Application Identifier is given the value ``Netscape'' and the
  Application Authentication Code is given the code ``2.0''. The
  application data always consists of a single sub-block of length
  3. Let us go through the values of this single sub-block:
  \cite{frazier97:_all_about_gif89}

  \subsubsection{Constant(Byte)}

  This is field doens't really have a formal name, so I named it
  constant, because it always has the constant value $1$.

  \subsubsection{Iterations(Unsigned)}

  This number indicates the number of times the animation should be
  repeated. If this value is $0$ the animation will loop infinitely.

  The Netscape application extension block was first used by the web
  browser Netscape. Other software developpers then saw that the block
  was a good idea and started implementing support for the block in
  their programs, and that is how the use of this block spread. The
  name Netscape Application Extension has stuck ever since because you
  didn't want to break old GIFs just because you wanted to make a
  silly name change \cite{walte96:_web_scrip_secret_weapon}.

  \section{Wrapping it up}

  So, to make an animated GIF, all you really do is specify a sequence
  of smaller images within a GIF file. But in which order should all
  the blocks of a GIF image occur for the GIF file to be considered
  well formed.

  \printbibliography[heading=subbibliography]

\end{refsection}
