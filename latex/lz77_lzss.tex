\begin{comment}
  \bibliography{project.bib}
\end{comment}

\section{LZ77 and LZSS}
\label{sec:lz77-lzss}

\section{History}

As familiar from section \ref{sec:hist-lzw}, LZ77 was the first of the
two algorithms described by Lempel and Ziv that sparked the family of
dictionary methods. We have partially discussed already LZ78, since
LZW was based on this method so we will in this chapter discuss the
LZ77 method. But we will also discuss LZSS, which is an improvement of
LZ77 that was developed by Storer and Szymanski in 1982.

\section{LZ77}

The following description of the LZ77 algorihtm is based on
\cite{Salomon:2004:DCC,mark1996data_compression_book,mcfadden92:_hackin_data_compr_ziv_lempel}.


The core parts of the LZ77 algorihtm are the so-called window and
lookahead buffers. In the window buffer already compressed data is
stored. The lookahead buffer contains the data that hasn't yet been
compressed and matched. Strings in the lookahead buffer are in the
algorihtm matched for already processed strings in the window
buffer. If it found a matching string for the string in the lookahead
buffer it outputs a so called \textit{token}. The token contains the
values that are necessary for the decoder to locate the matched
string in the window and decode it.

Now let us go though a simple example that demonstrates how LZ77
operates. Let us see how good of a compression LZ77 can perform on the
string \texttt{She sells sea shells on the sea shore}. But we will
first have to decide on the size of the lookahead and window
buffers. Since these buffers have to be stored in memory they
obviously cannot be infinitely big. The size of the window detemines
how way book in the previously compressed data we can look for string
matches. So the bigger window the better the compression tends to
be. The size of the lookahead buffer determines how long strings we
can match for in the window buffer. Usually, this number does not need
to be that big since it is usually rather small strings that get
matched for the window buffer. This size of almost always bigger than
the size of the window buffer. For this example we give the window
buffer a size of 20 characters, while the lookahead buffer is assigned
10 characters.

In the initial part of the algorihtm, the lookahead buffer is filled
to its maximum capability with characters. The window buffer is at
this part completely empty, since no characters has yet been
processed.

\newcommand{\windowsize}{20}
\newcommand{\lookaheadsize}{10}

\lzonedem{%
  \lzonestates{}{She{\spc}sells{\spc}}{S}
}

Since the window buffer is empty, the token \lzonetokens{S} is simply
output. This LZ77:s ways o signaling that match at all was found in
the window buffer.

Now the processed letter moved to the beginning of the window buffer a
new letter is read to the lookahead buffer, replacing the

%  \lzonestate{S}{he{\spc}sells{\spc}s}{0}{0}{h}
%  \lzonestate{Sh}{e{\spc}sells{\spc}se}{0}{0}{e}
%  \lzonestate{She}{{\spc}sells{\spc}sea}{0}{0}{{\spc}}

\section{LZSS}

The following description of the LZSS algorihtm is based on
\cite{Salomon:2004:DCC,mark1996data_compression_book,mcfadden92:_hackin_data_compr_lzss,okumura:_data_compr_algor_larc_lharc}.