\begin{comment}
  \bibliography{project.bib}
\end{comment}

\chapter{LZ77 and LZSS}
\label{cha:lz77-lzss}

\section{History}

As familiar from section \ref{sec:hist-lzw}, \lzone was the first of the
two algorithms described by Lempel and Ziv that sparked the family of
dictionary methods. We have partially discussed already \lztwo, since
\lzw was based on it, and we will in this chapter discuss the \lzone
method. But we will also discuss \lzss, which is an improvement of \lzone
that was developed by Storer and Szymanski in 1982.

\section{LZ77}

The following description of the \lzone algorihtm is based on
\cite{Salomon:2004:DCC,mark1996data_compression_book,mcfadden92:_hackin_data_compr_ziv_lempel}.

\subsection{Compression}

The core parts of the \lzone algorihtm are the so-called window and
lookahead buffers. In the window buffer already compressed data is
stored. The lookahead buffer contains the data that hasn't yet been
compressed and matched. Strings in the lookahead buffer are in the
algorihtm matched for already processed strings in the window
buffer. If it finds a matching string for the string in the lookahead
buffer, it outputs a \textit{token}. The token contains the
information that is necessary for the decoder to locate the matched
string in the window buffer.

Now let us go though a simple example that demonstrates how \lzone
operates. Let us study the \lzone compression of the string \texttt{she
  sells sea shells on the sea shore}. But we will first have to decide
on the size of the lookahead and window buffers. Since these buffers
have to be stored in memory they obviously cannot be infinitely
big. The size of the window determines how long back in the previously
compressed data we can look for string matches. The size of the
lookahead buffer determines how long strings we can match for in the
window buffer. This number does not need to be that big since it is
usually comparatively small strings that get matched for in the window
buffer. For this example, we give the window buffer a size of 20
characters, while the lookahead buffer is given a length of 10
characters.

In the initial part of the algorihtm, the lookahead buffer is filled
to its maximum capacity with characters. The window buffer is at this
point completely empty, since no characters has yet been processed:

\newcommand{\windowsize}{20}
\newcommand{\lookaheadsize}{10}

\lzonedem{%
  \lzonestates{}{she{\spc}sells{\spc}}{s}
}

Since the window buffer is empty no matching string can be found in
it. So the token \lzonetokens{s} is simply output. This is how \lzone
signals that no match at all was found in the window buffer. We will
soon describe what the first two numbers in the token describe. The
compressor then keeps outputting these single character tokens:

\lzonedem{%
  \lzonestates{s}{he{\spc}sells{\spc}s}{h}
  \lzonestatedots{}
  \lzonestates{she{\spc}sell}{s{\spc}sea{\spc}shel}{s}
}

(The \lzone compressor does not bother in matching single character
strings)While characters keeps getting read to the lookahead buffer,
processed characters get added to the window buffer. After the token
for the last character \texttt{s} has been outputted, interesting
things finally start happening; the next three characters to be found
in the lookahead buffer are \texttt{{\spc{}se}}, and this sequence can
also be found in the window buffer. How far back in the window buffer
does this match begin? If we start counting from the end of the window
buffer, then we have to go $6$(this count is one-based) characters
\textit{back} in the buffer to reach the match. This number is formally known
as the \textit{offset} of the token, which is the first number of a
token. The second number in a token represent the \textit{length} of
the matching string. Since the matching string, \texttt{{\spc{}se}},
has length of $3$, this will also be the second number of the
token. The third part of the token will get set to the character
following the matching string. Thus, the resulting token will be:

\lzonedem{%
  \lzonestate{she{\spc}sells}{{\spc}sea{\spc}shell}{6}{3}{a}
}

And after this yet another string can get matched for:

\lzonedem{%
  \lzonestate{she{\spc}sells{\spc}sea}{{\spc}shells on}{4}{2}{h}
}

If there are several matches in the window buffer, then obviously the
longest match will be picked. And in this fashion the rest of the
compression proceeds. Once the end of the input data has been, reached
the lookahead buffer will start shrinking as the remaining characters
are compressed, while the size of the window buffer is managed so that
it never rises above its fixed size value.

It is easy to see why \lzone works. Consider for example the following
extract from ``Alice in Wonderland'':

\begin{quote}
  `Then you should say what you mean,' the March Hare went on.

  `I do,' Alice hastily replied; `at least--\underline{at least} I
  \underline{mean} what I say--that's the same thing, you know.'

  `Not the \underline{same thing} a bit!' said \underline{the} Hatter.
  `You might just as well say that ``I see what I eat'' is the
  \underline{same thing} as ``\underline{I eat} \underline{what}
  \underline{I see}''!'

  `\underline{You might just as well say},' \underline{added the}
  \underline{March Hare}, `that ``I like what I get'' is the same
  thing as ``\underline{I get} what \underline{I like}''!'

  `\underline{You might just as well say},' \underline{added the}
  Dormouse, who seemed to be talking in his sleep, `that ``I breathe
  when I \underline{sleep}'' is the same thing as ``\underline{I
    sleep} \underline{when} \underline{I breathe''}!'

\end{quote}

Underlined in this extract are the longest parts of this text that
will get caught by the \lzone compressor and get represented more
efficiently as tokens. (Not all strings that will get caught by \lzone
are underlined, only the longest ones are.) Human language texts tend
to be chock-full with repeated words, no matter how well written they
are ,so they compress especially well under \lzone. And the same thing
could be said for simple images.

\subsection{Decompression}

A \lzone compressed file will consists of a sequence of \lzone tokens, so
the only thing we really need to discuss is how to decode the separate
tokens.

A token where the values of the offset and length fields are set to
$0$ is represented by the character in its third field.

Decoding a token that represents a matched string is not much
trickier. A token was encoded by finding a matching string in the
window buffer, and then outputting the offset and length of the
matching string. So the decoding step is simple: first, go back the
number of characters in the so far compressed data indicated by the
offset. Now add the character you are currently positioned at to the
decompressed data(this character is in fact the first character of the
matched string). And now, since a new character was added, the size of
the decompressed data will have increased by one. Now you can by again
going back offset steps in the decompressed data reach the second
character of the matched string. This process is then repeated until a
string with the length indicated by the second field of the token has
been decoded. This whole process is outlined in algorithm
\ref{alg:dec-lzone-token}. Once this has been done, the third field of
the match token is also added to the decompressed data.

\begin{algorithm}[H]
  \caption{Decoding a \lzone token}\algohack{}
  \label{alg:dec-lzone-token}
  \begin{algorithmic}[1]
    \linecomment{$\var{length}$ is the length of the so far decompressed data. }
    \Repeatn{$\var{tokenLength}$}
      \Let{$\var{decompressed[length]}$}{$\var{decompressed[length -
          tokenOffet]}$}
      \Let{$\var{length}$}{$\var{length + 1}$}

    \EndRepeatn
  \end{algorithmic}
\end{algorithm}

\subsection{Token Sizes}

But we have up until this point entirely omitted a discussion of the
bit sizes of the three fields of \lzone tokens. The bit size of the
third field is the easiest; since it must store an unmatched byte it
has the bit size 8. Deciding the size of the offset part is a bit
harder. This field must have a bit size such that it has the capacity
to specify the offset to \textit{any} matched string in the window
buffer. A window buffer of size $2^{10} = 1024$ will therefore require
the use of a 10-bit binary number for specifying the offset. You get
this value by taking the binary logarithm, $\log_2$, of the window
size:

\begin{equation*}
  \log_2 2^{10} = 10\log_2 2 = 10
\end{equation*}

Where the $\log_2$ is defined as

\begin{equation*}
  \log_2(x) = \frac{\log_b (x)}{\log_b (2)}
\end{equation*}

for \textit{any} base $b$.

For window sizes that can not be written on the form $2^n$ then the
above reasoning does not hold, but it needs an extension. For a window
size of $20$ 5-bit numbers are required for specifying offsets, since
$2^4 = 16$ and $2^5 = 32$. But $\log_2 20 \approx 4.32$, so if we let
$\lceil x \rceil$ denote the floor operation, then correct bit size is
calculated as

\begin{equation*}
  \lceil\log_2 20\rceil = 5
\end{equation*}

So for a window size of $w$, binary numbers of the bit length

\begin{equation*}
  \lceil\log_2 w\rceil
\end{equation*}

are used for specifying offsets. And the size of the length field is
the same way similarly.

\section{LZSS}

The following brief description of the \lzss algorihtm is based on
\cite{Salomon:2004:DCC,mark1996data_compression_book,mcfadden92:_hackin_data_compr_lzss,okumura:_data_compr_algor_larc_lharc}.

\lzone can perform excellent compression but it is also very wasteful in
how it outputs unmatched characters. For a window size of $2^8$ and a
lookahead size of $2^4$, $4 + 8 + 8 = 20$ whole bits are required for
specifying a number that would otherwise have been stored as 8 bits!

\lzss fixes that problem as follows: unmatched character are
represented by a single toggled bit, 1, followed by the 8-bit value of
the unmatched character. So the unmatched character \texttt{A}, which
has an ASCII code of $65$, would be represented by the bit pattern
\bin{1\ 0100\ 0001}. Match tokens are represented by a cleared bit, 0,
followed by the offset and length values.

