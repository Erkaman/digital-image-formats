\documentclass[a4paper]{article}
\usepackage[T1]{fontenc}
\usepackage[kerning,spacing]{microtype}
\usepackage[charter]{mathdesign}
\usepackage{datetime}
\usepackage{hyperref}
\hyphenpenalty=100000
\title{Log book}
\author{Eric Arneb�ck}
\date{}

\usdate

% day, month,year
\newcommand{\dayheading}[3]{
  \section*{\formatdate{#1}{#2}{#3}}}

\begin{document}

\maketitle

\dayheading{1}{9}{2011}

Today I met my supervisor, Aref.

This project will be about how image file formats work. It is mainly
focused in how color is represented in them and how the color data is
compressed. Only 3 image formats will be covered: TGA, GIF(no
animation will be covered, though), and PNG. If there's time over, I
may also cover JPEG(but I doubt it).

I've started working on the project a bit in advance. I have already
started working on the first chapter of the project: digital color
representation.

A tiny bit of work has been done on the TGA loader. But the program
is far from done and it really only analyzes the image header so far.

The latest days, I've been looking for good sources on how compression
algorithms work. I quickly noticed compression is a hugely complex
field with lots of hairy math. But the compression algorithms for the
format I've chosen are relatively simple, so that shouldn't pose to
big of a problem.

Last night, I started working on a program demonstrating the
extremely simple compression algorithm Run Length Encoding.

And today, I finished the program demonstrating the most simple
version of the RLE algorithm. More information on the algorithm can be read here:
\url{http://www.fileformat.info/mirror/egff/ch09_03.htm}

\dayheading{2}{9}{2011}

Figured out how to use the algorithmx package in \LaTeX. This
basically allows me to very easily write algorithms in documents and
have them beautifully formatted.

Wrote the very simple algorithm for decoding an Run Length encoded
file to get a feel for using the package.

\end{document}