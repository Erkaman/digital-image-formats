\documentclass[a4paper]{article}
\usepackage[T1]{fontenc}
\usepackage[kerning,spacing]{microtype}
\usepackage[charter]{mathdesign}
\usepackage{datetime}
\hyphenpenalty=100000
\title{Log book}
\author{Eric Arneb�ck}
\date{}

\usdate

% day, month,year
\newcommand{\dayheading}[3]{
  \section*{\formatdate{#1}{#2}{#3}}}

\begin{document}


\maketitle

\dayheading{1}{9}{2011}

Today I met my supervisor, Aref.

This project will about how image file formats. It is mainly focused
in how color is represented in them and how the data is
compressed. Only 3 image formats will be covered: TGA, GIF(no
animation will be covered, though), and PNG. If there's time over I
may also cover JPEG.

I've started working on the project a bit in advance. I have already
started working on the first chapter of the project: digital color
representation.

A tiny bit of work has been done the opener for the TGA format. But
that program is far from done and it really only analyzes the image
header so far.

The latest days, I've been looking for sources on how compression
algorithms work. I quickly noticed compression is a hugely complex
field with lots of math. But the compression algorithms for the format
I've chosen are relatively simple, so that shouldn't pose to big of a
problem.

Last night, I started working on a programming demonstrating the
extremely simple compression algorithm, Run Length Encoding.

\end{document}