\documentclass[a4paper]{article}
\usepackage[T1]{fontenc}
\usepackage[kerning,spacing]{microtype}
\usepackage[charter]{mathdesign}
\usepackage{datetime}
\usepackage{hyperref}
\hyphenpenalty=100000
\title{Log book}
\author{Eric Arneb�ck}
\date{}

\usdate

% day, month,year
\newcommand{\dayheading}[3]{
  \section*{\formatdate{#1}{#2}{#3}}}

\begin{document}

\setlength{\parskip}{1ex plus 0.5ex minus 0.2ex}
\setlength{\parindent}{0pt}

\maketitle

\dayheading{1}{9}{2011}

Today I met my supervisor, Aref.

This project will be about how image file formats work. It is mainly
focused in how color is represented in them and how the color data is
compressed. Only 3 image formats will be covered: TGA, GIF(no
animation will be covered, though), and PNG. If there's time over, I
may also cover JPEG(but I doubt it).

I've started working on the project a bit in advance. I have already
started working on the first chapter of the project: digital color
representation.

A tiny bit of work has been done on the TGA loader. But the program
is far from done and it really only analyzes the image header so far.

The latest days, I've been looking for good sources on how compression
algorithms work. I quickly noticed compression is a hugely complex
field with lots of hairy math. But the compression algorithms for the
format I've chosen are relatively simple, so that shouldn't pose to
big of a problem.

Last night, I started working on a program demonstrating the
extremely simple compression algorithm Run Length Encoding.

And today, I finished the program demonstrating the most simple
version of the RLE algorithm. More information on the algorithm can be read here:
\url{http://www.fileformat.info/mirror/egff/ch09_03.htm}

\dayheading{2}{9}{2011}

Figured out how to use the algorithmx package in \LaTeX. This allows
me to very easily write algorithms in documents and have them
beautifully formatted.

Started writing chapter 2 of the project, compression techniques. So
far I've only written a short introduction to RLE. I've also added a
psuedocode version the RLE decoding algorithm to the document, using
the aforementioned algorithmx package.

I've Also started working on the
psuedocode for the RLE encoding version. It's not done yet, though.

Also fixed several small bugs in rle program which demonstrates rle encoding.

\dayheading{3}{9}{2011}

Spent quite a bit of time revising and improving the algorithms for
Run Length encoding and decoding, and making the pseudocode of them
easier to understand.

Implemented the RLE variant TGA in uses in its compression
scheme.

Before I had started writing this log book I had written a couple of
pages on digital color representation. I had even found quite reliable
sources on the subject. Sections on color wavelengths, RGB, alpha
channels, color depth had already been written. Today I revised these
sections and made them even clearer and added a couple of figures and
tables to illustrate the concepts better. I also added an entirely new
section on how grayscale color is represented.

Before starting writing this log book I had already done a lot work on a
demo program that loads dump a TGA. What this program basically will
do is that will it will load TGA and write all the information
associated with a TGA to a text file along with the uncompressed color
data.

After todays work it is able to load and dump uncompressed grayscale
images.

\dayheading{5}{9}{2011}

A image's metadata is basically data describing things like the author
of the image, the date at which it was created, and soon on. The
majority of todays work was spent figuring out how to read the
metadata from TGA image files and then implementing it in code. This
was a very boring and repetitive process, but along the way I got a
good sense on the overall structure of the simple TGA format.

Was today able to code the functionality that loads a Run Length
Encoded grayscale TGA image. Now that I have understood the
compression algorithm of the format, the rest will be trivial to implement. I have
successfully gotten past my first, but certainly not last, hurdle.

\dayheading{7}{9}{2011}

So now the TGA loader program is also capable of correctly loading
24-bit, uncompressed AND compressed, true color images. If you didn't understand
anything of that, read the main document. Figuring out how to do this was
unsurprisingly not very hard at all. Just some elementary bitwise operators...

\dayheading{11}{9}{2011}

After several days of not having any time or energy(been ill) for working on
the project, I've finally returned. Just a very small thing was done
today: I've made it possible for the TGA loader to read 32-bit RGBA
color, both compressed and uncompressed. All that remains to be done
now are color mapped images, but that shouldn't be very difficult.

\dayheading{12}{9}{2011}

I forgot that TGA also supported 16-bit images, so I quickly
implemented support for this.

Because of several bugs implementing support for color
mapped images took longer than expected. But now the most important
parts of the file format has been covered! There are some
miscellaneous parts, and less important parts left, but I'll cover
them later. I'll need to look through the loading code tomorrow to
clean it up(it's gotten quite messy) and fix some potential bugs that
might have crept in. Then I'll start writing and start trying to
explain the TGA format as good as I can.

\end{document}